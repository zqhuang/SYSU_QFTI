\documentclass[CJK]{beamer}
\usepackage{CJKutf8}
\usepackage{beamerthemesplit}
\usetheme{Malmoe}
\useoutertheme[footline=authortitle]{miniframes}
\usepackage{amsmath}
\usepackage{amssymb}
\usepackage{graphicx}
\usepackage{color}
\graphicspath{{figures/}}
\def \bch {\begin{CJK}{UTF8}{gbsn}}
\def \ech {\end{CJK}}
\def \bex {\begin{minipage}{0.3\textwidth}\includegraphics[width=1in]{jugelizi.png}\end{minipage}\begin{minipage}{0.6\textwidth}}
\def \eex {\end{minipage}}
\def \chtitle#1 {\frametitle{\bch #1 \ech}}
\def \skipline { {\vskip 0.1in}}
\def \langr {\mathcal{L}}
\def \hamil {\mathcal{H}}
\def \vecx {\mathbf{x}}
\def \veck {\mathbf{k}}
\def \vecp {\mathbf{p}}
\def \hatphi {\hat{\phi}}
\def \hatq {\hat{q}}
\def \hatpi  {\hat{\pi}}
\def \vel {\upsilon}
\def \Dint {\mathcal{D}}
\def \adag {{\hat{a}^\dagger}}
\def \hata {\hat{a}}
\def \hatN {\hat{N}}
\def \hatH {\hat{H}}
\def \nket { {| n \rangle}}
\def \bran { {\langle n |}}

\title{Quantum Field Theory I \\ Lesson 02 - Action Princple}
  \author{}
  \date{}


\begin{document}

\begin{frame}
 
\begin{center}
\begin{Large}
\bch
量子场论 I 

{\vskip 0.3in}

作用量原理和Noether定理

\ech
\end{Large}
\end{center}

\vskip 0.2in

\bch
课件下载
\ech
https://github.com/zqhuang/SYSU\_QFTI

\end{frame}


\begin{frame}
\frametitle{\bch 课堂互动:函数与稳定点 \ech}
\bch
稳定点:所有一阶导数全为零的点 (极值点一定是稳定点,反之则未必)
\begin{itemize}
\item{$f(\phi) = \phi$存在稳定点吗}
\item{求$f(\phi) = \phi^2$的稳定点}
\item{求$f(\phi) = \frac{1}{3}\phi^3-\phi$的稳定点}
\item{求$f(\phi_1, \phi_2) = \frac{1}{3}\phi_1^3 - \phi_1 + (\phi_2-\phi_1)^2$的稳定点}
\item{求$f(\phi_1, \phi_2, \ldots, \phi_n) = \frac{1}{3}\phi_1^3 - \phi_1 + \sum_{j=2}^n(\phi_j - \phi_1)^2$的稳定点}
\item{求$f(\phi_1, \phi_2, \ldots) = \frac{1}{3}\phi_1^3 - \phi_1 + \sum_{j=2}^\infty(\phi_j - \phi_1)^2$的满足“边界条件”$\phi_1>0$的稳定点。}
\end{itemize}

\skipline

我们看到函数如存在多个稳定点,则可以添加适当的“边界条件”使得稳定点唯一。

\ech
\end{frame}

\begin{frame}
\chtitle{场的函数}
\bch
以自然数标记自由度的变量$(\phi_1, \phi_2, \ldots)$的进一步推广就是用坐标来标记自由度的“场”$\phi_t$ ($t\in (-\infty, \infty)$)。当然,我们更习惯把它记成$\phi(t)$。在学习场论的过程中,我们通常不把场看成坐标的函数,而是把坐标$t$看成自由度的标记,把场看成有无穷多自由度的变量。

\skipline

场既然只是一个(自由度有点多)的变量,当然就可以定义它的函数,数学上叫“泛函”(functional):
\ech
\bex
\bch
$$ S = \int_{-\infty}^\infty \frac{m}{2}\left[(\frac{d\phi}{dt})^2 - \omega^2\phi^2\right] dt$$
把$\phi(t)$映射为一个数。这里$\omega$是具有角速度量纲的常量,$m$是具有质量量纲的常量。
\ech
\eex
\end{frame}

\begin{frame}
\chtitle{四维时空里的场的泛函}
\bch
四维时空里的场(也同样可以看成一个无穷多自由度的变量,其自由度用四维时空坐标来标记)也可以有它的泛函,例如。
\ech
\bex
$$ S = \int \sqrt{-g} d^4x \left[\frac{1}{2}\partial^\mu\phi \partial_\mu\phi - \frac{1}{2}m^2\phi^2\right]$$
\bch
把场$\phi(x^0,x^1, x^2, x^3)$映射为一个数$S$. 其中$g$是度规$g_{\mu\nu}$的行列式的简写。$d^4x$是$dx^0dx^1dx^2dx^3$的简写。$m$是质量量纲的常量。积分范围是整个四维时空。这里我们假设了度规是已知确定的。(允许度规变化的理论参考广义相对论。)
\ech
\eex
\end{frame}

\begin{frame}
\chtitle{作用量}
\bch
泛函

$$ S = \int_{-\infty}^\infty \frac{m}{2}\left[(\frac{d\phi}{dt})^2 - \omega^2\phi^2\right] dt$$

定义了一个物理系统(谐振子)。反过来的等价说法是这个物理系统的“作用量”(Action)是上述泛函。

\skipline
{\bf 作用量的满足边界条件的稳定点给出了物理系统的经典解。}原则上所有经典物理问题都可以归结为求解作用量的稳定点。
\ech

\end{frame}


\begin{frame}

\end{frame}


\end{document}
