\documentclass[CJK]{beamer}
\usepackage{CJKutf8}
\usepackage{beamerthemesplit}
\usetheme{Malmoe}
\useoutertheme[footline=authortitle]{miniframes}
\usepackage{amsmath}
\usepackage{amssymb}
\usepackage{graphicx}
\usepackage{color}
\graphicspath{{figures/}}
\def \bch {\begin{CJK}{UTF8}{gbsn}}
\def \ech {\end{CJK}}
\def \bex {\begin{minipage}{0.3\textwidth}\includegraphics[width=1in]{jugelizi.png}\end{minipage}\begin{minipage}{0.6\textwidth}}
\def \eex {\end{minipage}}
\def \chtitle#1 {\frametitle{\bch #1 \ech}}
\def \skipline { {\vskip 0.1in}}
\def \langr {\mathcal{L}}
\def \hamil {\mathcal{H}}
\def \vecx {\mathbf{x}}
\def \veck {\mathbf{k}}
\def \vecp {\mathbf{p}}
\def \hatphi {\hat{\phi}}
\def \hatq {\hat{q}}
\def \hatpi  {\hat{\pi}}
\def \vel {\upsilon}
\def \Dint {\mathcal{D}}
\def \adag {{\hat{a}^\dagger}}
\def \hata {\hat{a}}
\def \hatN {\hat{N}}
\def \hatH {\hat{H}}
\def \nket { {| n \rangle}}
\def \bran { {\langle n |}}

\title{Quantum Field Theory I \\ Lesson 02 - Action Princple}
  \author{}
  \date{}


\begin{document}

\begin{frame}
 
\begin{center}
\begin{Large}
\bch
量子场论 I 

{\vskip 0.3in}

第二次课后作业 (共八次,每次2.5分)

交作业时间: 10月10日,星期一,13:30pm

\ech
\end{Large}
\end{center}

\vskip 0.2in

\bch
课件下载
\ech
https://github.com/zqhuang/SYSU\_QFTI

\end{frame}

\begin{frame}
\chtitle{第1题(0.5分)}
\bch
对质量为$m$的自由实标量场$\phi$,四维动量$k$满足$k^0 = \omega \equiv \sqrt{\veck^2+m^2}$,其中$\veck\equiv (k^1, k^2, k^3)$为三维动量。对任意洛仑兹变换下不变的函数$f(k)$,证明积分$\int \frac{d^3\veck}{2\omega} f(k)$ 也是洛仑兹变换下的不变量。
\ech
\end{frame}

\begin{frame}
\chtitle{第2题(0.5分)}
\bch
对质量为$m$的自由实标量场$\phi$,证明四维时空下的任意两点场算符的对易$[\hatphi(x),\hatphi(x')]$是洛仑兹变换下的不变量。(提示:利用$\hatphi$的算符表达式和第1题结论。)
\ech
\end{frame}

\begin{frame}
\chtitle{第3题(0.5分)}
\bch
质量为$m$的自由复标量场,
$$\lagr = \partial_\mu\phi^\dagger \partial^\mu \phi - m^2\phi^\dagger\phi\, $$
把$\phi$和$\phi^\dagger$分别看作独立自由度,它们对应的正则动量分别为
$$\pi = \frac{\partial \lagr}{\partial \dot\phi} = \dot\phi^\dagger, \,\pi^\dagger = \frac{\partial \lagr}{\partial \dot\phi^\dagger} = \dot\phi $$
于是Hamilton密度为
$$\hamil = \dot\phi^\dagger \pi^\dagger +\dot\phi \pi - \lagr = \pi^\dagger \pi + \nabla\phi^\dagger \cdot\nabla\phi + m^2\phi^\dagger \phi $$
我们在课堂上推到了量子化后的$\phi$为
$$\hatphi(x) = \frac{1}{(2\pi)^{3/2}} \int \sqrt{\frac{d^3\veck}{2\omega}} \left(\hata_{\veck} e^{-ik_\mu x^\mu} + \bdag_{\veck}e^{ik_\mu x^\mu}\right) $$
试求量子化后的总Hamilton量$\hat{H}$.
\ech
\end{frame}


\begin{frame}
\chtitle{第4题(0.5分)}
\bch
考虑和规范场耦合的复标量场的拉氏密度:
$$\lagr = (D^\mu\phi)^\dagger D_\mu \phi - m^2\phi^\dagger \phi  - \frac{1}{4}\Fup\Fdown$$
利用Euler-Lagrange方程
$$\partial_\mu \Fup = j^\nu$$
证明$j_\mu$是守恒流.
\ech
\end{frame}


\begin{frame}
\chtitle{第5题(0.5分)}
\bch
关于三维空间转动算符
\begin{itemize}
\item{证明相反方向的角动量算符相差一个负号$\hat{J}_{-\vecn} = - \hatJn$}
\item{因为绕固定轴转动$2\pi$的没有任何效果,所以转动算符$e^{-2\pi\ii\hatJn}=1$,由此证明$\hatJn$的本征值一定是整数。}
\item{三个形成右手正交系的方向$\vecn_1$, $\vecn_2$, $\vecn_3$,对应的转动算符分别为$\hatj{1} $, $\hatj{2} $, $\hatj{3}$,证明$\hatj{1}^2+\hatj{2}^2+\hatj{3}^2 = 2$。.}
\end{itemize}
\ech
\end{frame}


\end{document}
