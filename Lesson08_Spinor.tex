\documentclass[CJK]{beamer}
\usepackage{CJKutf8}
\usepackage{beamerthemesplit}
\usetheme{Malmoe}
\useoutertheme[footline=authortitle]{miniframes}
\usepackage{amsmath}
\usepackage{amssymb}
\usepackage{graphicx}
\usepackage{color}
\graphicspath{{figures/}}
\def \bch {\begin{CJK}{UTF8}{gbsn}}
\def \ech {\end{CJK}}
\def \bex {\begin{minipage}{0.3\textwidth}\includegraphics[width=1in]{jugelizi.png}\end{minipage}\begin{minipage}{0.6\textwidth}}
\def \eex {\end{minipage}}
\def \chtitle#1 {\frametitle{\bch #1 \ech}}
\def \skipline { {\vskip 0.1in}}
\def \langr {\mathcal{L}}
\def \hamil {\mathcal{H}}
\def \vecx {\mathbf{x}}
\def \veck {\mathbf{k}}
\def \vecp {\mathbf{p}}
\def \hatphi {\hat{\phi}}
\def \hatq {\hat{q}}
\def \hatpi  {\hat{\pi}}
\def \vel {\upsilon}
\def \Dint {\mathcal{D}}
\def \adag {{\hat{a}^\dagger}}
\def \hata {\hat{a}}
\def \hatN {\hat{N}}
\def \hatH {\hat{H}}
\def \nket { {| n \rangle}}
\def \bran { {\langle n |}}

\title{Quantum Field Theory I \\ Lesson 08 - Spinor}
\author{}
\date{}


\begin{document}

\begin{frame}
 
\begin{center}
\begin{Large}
\bch
量子场论 I 

{\vskip 0.3in}

第七课 旋量

\ech
\end{Large}
\end{center}

\vskip 0.2in

\bch
课件下载
\ech
https://github.com/zqhuang/SYSU\_QFTI

\end{frame}


\begin{frame}
\chtitle{我们仅在平直时空内讨论旋量}
\bch
历史渊源,想把方程$(g^{\alpha\beta}\partial_\alpha\partial_\beta + m^2)\phi = 0$拆成一次算符的乘积:
$$(-\ii\gamma^\nu\partial_\nu - m)(\ii \gamma^\mu\partial_\mu - m)\psi = 0$$
进而考虑满足一次方程$(i\gamma^\mu - m)\psi = 0$的场$\psi$

\skipline
显然,要满足这样的条件必须有
$$\gamma^\mu\gamma^\nu + \gamma^\nu\gamma^\mu = 2g^{\mu\nu}$$
由此引发了一系列的故事……
\ech
\end{frame}


\begin{frame}
\chtitle{我们仅在平直时空内讨论旋量}
\bch
因为对一般度规寻找$\gamma^\mu\gamma^\nu + \gamma^\nu\gamma^\mu = 2g^{\mu\nu}$变得非常困难,在讨论旋量时,我们仅在平直时空中取Minkowski度规:
$g_{\mu\nu} = \mathrm{diag}(1, -1, -1, -1)$
\ech

\end{frame}

\begin{frame}
\chtitle{数学准备:$\gamma$矩阵}
\bch

约定矩阵A, B的反对易符号$\{A, B\}\equiv AB + BA$。已知存在四个$n\times n$复数矩阵$\gamma^\mu$ ($\mu = 0, 1, 2, 3$)满足$\{\gamma^\mu,\gamma^\nu\} = 2g^{\mu\nu}I$,其中I是$n\times n $单位矩阵。另外,我们定义$\gamma^5\equiv \ii\gamma^0\gamma^1\gamma^2\gamma^3$。
\begin{itemize}
\item{证明 $(\gamma^5)^2 = I$和$\{\gamma^5,\gamma^\mu\} = 0$ ($\mu = 0,1,2,3$)}
\item{若$\mu\ne \nu$ (可取$0,1,2,3,5$中任两个),证明$I$和$\gamma^\mu\gamma^\nu$线性独立}
\item{若$\mu,\nu,\lambda$互不相同(可取$0,1,2,3,5$中任三个),证明$I,\gamma^\mu\gamma^\nu,\gamma^\nu\gamma^\lambda,\gamma^\lambda\gamma^\mu$这4个矩阵线性独立}
\item{证明$\gamma^\mu (\mu = 0,1,2,3),\, \gamma^5\gamma^\mu (\mu = 0,1,2,3)$这8个矩阵线性独立}
\item{证明$I,\, \gamma^5,\,  \gamma^\mu\gamma^\nu (0\le \mu<\nu\le 3)$这8个矩阵线性独立}
\item{最后,证明 $I,\, \gamma^5,\, \gamma^\mu(\mu=0,1,2,3),\, \gamma^5\gamma^\mu(\mu=0,1,2,3),\ \gamma^\mu\gamma^\nu (0\le \mu<\nu\le 3)$这16个矩阵线性独立}
\end{itemize}

\ech
\end{frame}

\begin{frame}
\chtitle{数学准备:$\gamma$矩阵}
\bch
因为已经找到了16个线性独立的$n\times n$的矩阵,$n$至少等于4。

Dirac找到了$n=4$的一组具体解:
\begin{equation}
\gamma^0 = \left(\begin{array}{rr} I_{2\times 2} & 0 \\  0 & -I_{2\times 2} \end{array}\right) ,\ \ 
\gamma^i = \left(\begin{array}{rr} 0 & \sigma^i \\ -\sigma^i &  0 \end{array}\right) \nonumber
\end{equation}
其中$\sigma^i$为$2\times 2$的Pauli矩阵,
\begin{equation}
\sigma^1 = \left(\begin{array}{rr} 0 & 1 \\ 1 & 0 \end{array} \right),\, 
\sigma^1 = \left(\begin{array}{rr} 0 & -\ii \\ \ii & 0 \end{array} \right),\, 
\sigma^1 = \left(\begin{array}{rr} 1 & 0 \\ 0 & -1 \end{array} \right) 
\end{equation}
这样可以直接算出:
\begin{equation}
\gamma^5 = \left(\begin{array}{rr} 0 & I_{2\times 2} \\  I_{2\times 2} & 0 \end{array}\right) \nonumber
\end{equation}

对Dirac的$\gamma$矩阵,容易验证$(\gamma^0)^\dagger = \gamma^0$, $(\gamma^i)^\dagger = - \gamma^i$ ($i=1,2,3$), $(\gamma^5)^\dagger = \gamma^5$.
\ech
\end{frame}


\begin{frame}
\chtitle{数学准备:$\gamma$矩阵}
\bch
今后默认$n=4$,并在不致引起混淆时不再写出$I$,例如$\{\gamma^\nu, \gamma^\nu\} = 2 g^{\mu\nu}$, $(\gamma^5)^2 = 1$等。

\skipline

为了进一步简化符号,对任意矢量$A$, Feynman定义了符号:$\slashed{A} \equiv \gamma^\mu A_\mu$。

\skipline

讨论:试证明 $(\slashed{A})^2 = A^\mu A_\mu $

\ech
\end{frame}

\begin{frame}
\chtitle{数学准备:$\gamma$矩阵}
\bch
从$\gamma^5$和$\gamma^0$可以定义
$P_L = \frac{1}{2}\left(1-\gamma^5\right),\, P_R = \frac{1}{2}\left(1+\gamma^5\right)$
证明:
$$P_L^2 = P_L;\, P_R^2 = P_R; P_LP_R = P_R P_L=0$$
今后我们会看到$P_L$, $P_R$分别为左旋投影算符和右旋投影算符

\skipline
假设四维动量$k^\mu = (\omega, \veck)$满足$k^\mu k_\mu = \omega^2 - \veck^2 = m^2$,定义
$$P_+ = \frac{1}{2}\left(1+ \frac{\slashed{k}}{m}\right),\, P_- = \frac{1}{2}\left(1- \frac{\slashed{k}}{m}\right)$$
证明
$$P_+^2 = P+;\, P_-^2 = P_-; P_+P_- = P_-P_+ = 0$$
今后我们会看到$P_+$和$P_-$分别为粒子与反粒子的投影算符。
\ech
\end{frame}

\begin{frame}
\chtitle{数学准备:$\gamma$矩阵}
\bch
矩阵$\gamma^\mu$的洛仑兹变换和原矩阵相似,且相似变换矩阵不依赖于指标$\mu$。也就是说:存在矩阵$\Lambda$,使得对$\mu=0,1,2,3$均有
$$\Lambda^{-1}\gamma^\mu\Lambda = a^\mu_{\ \nu} \gamma^\nu$$
其中$a^\mu_{\ \nu}$为洛仑兹变换矩阵,满足$a^\mu_{\ \lambda} a^\lambda_{\ \nu} = g^\mu_{\ \nu}$。我们称$\Lambda$为该洛仑兹变换下的旋量变换矩阵。

\skipline
非常重要的概念澄清:$\Lambda$为$n\times n$矩阵,它作用于旋量的内部$n$维空间,而$a^\mu_{\ \nu}$作用于普通的坐标空间。只不过刚好在Dirac表示下取$n=4$使得$\Lambda$看起来跟$a^\mu_{\ \nu}$类似。


\skipline
正规洛仑兹变换($\det(a) = 1$)可以看成很多无穷小的正规洛仑兹变换的乘积,试先对无穷小的正规洛仑兹变换求解旋量变换矩阵,并推广到任意的正规洛仑兹变换。

\ech
\end{frame}



\begin{frame}
\chtitle{空间反射下的旋量变换矩阵}
\bch
在空间反射下:
$$\gamma^0 \rightarrow \gamma^0,\  \gamma^i\rightarrow -\gamma^i (i=1,2,3)$$
显然$\Lambda = \gamma^0$满足
$$\Lambda^{-1}\gamma^\mu \Lambda = a^\mu_{\ \nu}\gamma^\nu$$



\skipline

因为任何非正规(行列式为-1)的洛仑兹变换可以看成空间反射和正规洛仑兹变换的乘积,我们就已经相当于求解出了对任意洛仑兹变换的旋量变换矩阵。

\ech

\end{frame}


\begin{frame}
\chtitle{数学准备:$\gamma$矩阵}
\bch
证明
\begin{itemize}
\item{正规洛仑兹变换下的旋量变换矩阵$\Lambda$和$\gamma^5$对易,非正规洛仑兹变换下的旋量变换矩阵和$\gamma^5$反对易。}
\item{如果取$\gamma$矩阵的Dirac表示,任意洛仑兹变换下的旋量变换矩阵$\Lambda$满足$\Lambda^\dagger = \gamma^0 \Lambda^{-1}\gamma^0$}
\end{itemize}

\skipline

提示:先证明对无穷小变换成立,再推广到任意变换。
\ech
\end{frame}


\begin{frame}
\chtitle{数学准备:$\gamma$矩阵}
\bch

总算把$\gamma$矩阵的数学准备讲完了……(退课还来得及吗?)

\ech
\end{frame}




\begin{frame}
\chtitle{旋量的定义}
\bch
设$\Lambda$为洛仑兹变换下的旋量变换矩阵,旋量定义为满足下述坐标变换规则的有$n$个复分量的变量:
$$\psi \rightarrow \Lambda \psi$$

\skipline
注意:这里讲的旋量仅为描述自旋1/2的粒子的旋量,或称Dirac旋量。一般性的自旋为$3/2,5/2,...$的旋量我们暂不讨论。

\skipline
非常重要的概念澄清:因为在Dirac的$\gamma$矩阵表示下$n=4$,所以旋量在Dirac表示下是有四个复分量的变量$(\psi^0,\psi^1, \psi^2, \psi^3)$。但旋量的分量描述的是旋量在内部$n$维空间的投影,千万不要把旋量看成矢量,也不要用普通空间的度规对旋量进行指标的升降。
\ech
\end{frame}


\begin{frame}
\chtitle{Dirac共轭}
\bch
旋量的Dirac共轭在Dirac的$\gamma$矩阵表示下定义为:
$$\bar\psi \equiv \psi^\dagger \gamma^0$$

在洛仑兹变换下证明:
\begin{itemize}
\item{$\bar\psi \psi$是标量}
\item{$\bar\psi\gamma^\mu\psi$和$\bar\psi\gamma^5\psi$是矢量}
\item{$\bar\psi\gamma^\mu\gamma^\nu\psi$是反对称张量}
\item{$\bar\psi\gamma^5\psi$ 是赝标量}
\item{$\bar\psi\gamma^5\gamma^\mu\psi$ 是赝矢量}
\end{itemize}

\ech
\end{frame}

\begin{frame}
\bch
有了标量就可以开始构造拉氏密度了,下节课我们开始讲旋量场的拉氏密度和场方程。(是的,悲伤才刚刚开始……)
\ech
\end{frame}


\end{document}
