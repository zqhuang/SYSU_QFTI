\documentclass[CJK]{beamer}
\usepackage{CJKutf8}
\usepackage{beamerthemesplit}
\usetheme{Malmoe}
\useoutertheme[footline=authortitle]{miniframes}
\usepackage{amsmath}
\usepackage{amssymb}
\usepackage{graphicx}
\usepackage{color}
\graphicspath{{figures/}}
\def \bch {\begin{CJK}{UTF8}{gbsn}}
\def \ech {\end{CJK}}
\def \bex {\begin{minipage}{0.3\textwidth}\includegraphics[width=1in]{jugelizi.png}\end{minipage}\begin{minipage}{0.6\textwidth}}
\def \eex {\end{minipage}}
\def \chtitle#1 {\frametitle{\bch #1 \ech}}
\def \skipline { {\vskip 0.1in}}
\def \langr {\mathcal{L}}
\def \hamil {\mathcal{H}}
\def \vecx {\mathbf{x}}
\def \veck {\mathbf{k}}
\def \vecp {\mathbf{p}}
\def \hatphi {\hat{\phi}}
\def \hatq {\hat{q}}
\def \hatpi  {\hat{\pi}}
\def \vel {\upsilon}
\def \Dint {\mathcal{D}}
\def \adag {{\hat{a}^\dagger}}
\def \hata {\hat{a}}
\def \hatN {\hat{N}}
\def \hatH {\hat{H}}
\def \nket { {| n \rangle}}
\def \bran { {\langle n |}}

\title{Quantum Field Theory I \\ Lesson 03 - Quantization of Scalar Fields}
\author{}
\date{}


\begin{document}

\begin{frame}
 
\begin{center}
\begin{Large}
\bch
量子场论 I 

{\vskip 0.3in}

第四课 标量场的量子化(II)

\ech
\end{Large}
\end{center}

\vskip 0.2in

\bch
课件下载
\ech
https://github.com/zqhuang/SYSU\_QFTI

\end{frame}



\begin{frame}
\chtitle{实标量场的量子化}
\bch
回到傅立叶变换后的$\phi$的拉氏量:
$$L = \int d^3\veck\, \left[\frac{1}{2}|\dot\phi(\veck)|^2 - \frac{\omega^2}{2}|\phi(\veck)|^2\right]$$
\skipline

考虑到$\phi(\veck)$是实数场的傅立叶变换,必须满足$\phi^\dagger(\veck) = \phi(-\veck)$, 上述积分可以只对半个$\veck$空间(例如限定$k_1>0$)进行。在半空间内把$\phi(\veck)$分解为实部和虚部$\phi(\veck) = \frac{u(\veck) + i \vel(\veck)}{\sqrt{2}}$,$u$和$v$就是独立变量了:

\skipline
$$L = \int_{k_1>0} d^3\veck\,\left[ \frac{1}{2}\left(\dot u^2 + \dot \vel^2 \right) - \frac{\omega^2}{2}\left(u^2+v^2\right)\right]$$

\ech

\end{frame}

\begin{frame}
\chtitle{实标量场的量子化}
\bch
于是每个$u(\veck)\sqrt{d^3\veck}$和$\vel(\veck)\sqrt{d^3\veck}$均为单位质量谐振子, 我们可以写出
$$u  = \frac{\hata_u + \adag_u}{\sqrt{2\omega d^3\veck}}$$
$$\vel  = \frac{\hata_{\vel} + \adag_{\vel}}{\sqrt{2\omega d^3\veck}}$$
\ech
\end{frame}


\end{document}
