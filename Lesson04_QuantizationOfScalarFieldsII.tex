\documentclass[CJK]{beamer}
\usepackage{CJKutf8}
\usepackage{beamerthemesplit}
\usetheme{Malmoe}
\useoutertheme[footline=authortitle]{miniframes}
\usepackage{amsmath}
\usepackage{amssymb}
\usepackage{graphicx}
\usepackage{color}
\graphicspath{{figures/}}
\def \bch {\begin{CJK}{UTF8}{gbsn}}
\def \ech {\end{CJK}}
\def \bex {\begin{minipage}{0.3\textwidth}\includegraphics[width=1in]{jugelizi.png}\end{minipage}\begin{minipage}{0.6\textwidth}}
\def \eex {\end{minipage}}
\def \chtitle#1 {\frametitle{\bch #1 \ech}}
\def \skipline { {\vskip 0.1in}}
\def \langr {\mathcal{L}}
\def \hamil {\mathcal{H}}
\def \vecx {\mathbf{x}}
\def \veck {\mathbf{k}}
\def \vecp {\mathbf{p}}
\def \hatphi {\hat{\phi}}
\def \hatq {\hat{q}}
\def \hatpi  {\hat{\pi}}
\def \vel {\upsilon}
\def \Dint {\mathcal{D}}
\def \adag {{\hat{a}^\dagger}}
\def \hata {\hat{a}}
\def \hatN {\hat{N}}
\def \hatH {\hat{H}}
\def \nket { {| n \rangle}}
\def \bran { {\langle n |}}

\title{Quantum Field Theory I \\ Lesson 04 - Quantization of Scalar Fields II}
\author{}
\date{}


\begin{document}

\begin{frame}
 
\begin{center}
\begin{Large}
\bch
量子场论 I 

{\vskip 0.3in}

第四课 标量场的量子化(II)

\ech
\end{Large}
\end{center}

\vskip 0.2in

\bch
课件下载
\ech
https://github.com/zqhuang/SYSU\_QFTI

\end{frame}



\begin{frame}
\chtitle{实标量场的量子化}
\bch
傅立叶变换后的$\phi(\veck)$的拉氏量:
$$L = \int d^3\veck\, \left[\frac{1}{2}|\dot\phi(\veck)|^2 - \frac{\omega^2}{2}|\phi(\veck)|^2\right]$$
\skipline

考虑到$\phi(\veck)$是实数场的傅立叶变换,必须满足$\phi^\dagger(\veck) = \phi(-\veck)$, 上述积分可以只对半个$\veck$空间(例如限定$k_1\ge 0$)进行。 在半空间内把$\phi(\veck)$分解为实部和虚部$\phi(\veck) = \frac{u(\veck) + i \vel(\veck)}{\sqrt{2}}$,$u$和$v$就是独立变量了:

\skipline
$$L = \int_{k_1\ge 0} d^3\veck\,\left[ \frac{1}{2}\left(\dot u^2 + \dot \vel^2 \right) - \frac{\omega^2}{2}\left(u^2+v^2\right)\right]$$

\ech

\end{frame}

\begin{frame}
\chtitle{实标量场的量子化}
\bch
于是每个$u(\veck)\sqrt{d^3\veck}$和$\vel(\veck)\sqrt{d^3\veck}$均为单位质量谐振子, 我们可以写出
$$\hat{u}(\veck)  = \frac{\hata_u(\veck) + \adag_u(\veck)}{\sqrt{2\omega\, d^3\veck}}$$
$$\hat{\vel}(\veck)  = \frac{\hata_{\vel}(\veck) + \adag_{\vel}(\veck)}{\sqrt{2\omega\, d^3\veck}}$$
于是
\begin{equation}
\hatphi(\veck) = \left\{
\begin{array}{ll}
\frac{1}{\sqrt{2\omega\, d^3\veck}}\left(\frac{\hata_u(\veck) + i\hata_\vel(\veck)}{\sqrt{2}} + \frac{\adag_u(\veck)+i\adag_\vel(\veck)}{\sqrt{2}}\right), & \mathrm{if\ }k_1\ge 0 \\
\frac{1}{\sqrt{2\omega\, d^3\veck}}\left(\frac{\hata_u(-\veck) - i\hata_\vel(-\veck)}{\sqrt{2}} + \frac{\adag_u(-\veck)-i\adag_\vel(-\veck)}{\sqrt{2}}\right), & \mathrm{if\ }k_1< 0 
\end{array}
\right.
\end{equation}

\ech
\end{frame}

\begin{frame}
\chtitle{实标量场的量子化}
\bch
为了简化符号,我们定义
\begin{equation}
\hata_{\veck} \equiv \left\{ \begin{array}{ll}
\frac{1}{\sqrt{2}}\left(\hata_u(\veck) + i \hata_{\vel}(\veck)\right), & \ \mathrm{if\ }k_1\ge 0 \\
\frac{1}{\sqrt{2}}\left(\hata_u(-\veck) - i \hata_{\vel}(-\veck)\right), & \ \mathrm{if\ }k_1<0 
\end{array}\right. \nonumber
\end{equation}
并求共轭转置得到
\begin{equation}
\adag_{\veck} = \left\{ \begin{array}{ll}
\frac{1}{\sqrt{2}}\left(\adag_u(\veck) - i \adag_{\vel}(\veck)\right), & \ \mathrm{if\ }k_1\ge 0 \\
\frac{1}{\sqrt{2}}\left(\adag_u(-\veck) + i \adag_{\vel}(-\veck)\right), & \ \mathrm{if\ }k_1<0 
\end{array}\right. \nonumber
\end{equation}
容易验证
$$\hatphi(\veck) = \frac{\hata_{\veck}+\adag_{-\veck}}{\sqrt{2\omega\, d^3\veck}}$$
对任意$\veck$成立。
\ech
\end{frame}

\begin{frame}
\chtitle{实标量场的量子化}
\bch
注意$u$和$\vel$,以及半空间内的$\veck$均标记不相关的自由度,不同自由度的产生或湮灭算符都对易。利用这个来证明:
\begin{itemize}
\item{$[\hata_{\veck}, \hata_{\veck'}]=[\adag_{\veck},\adag_{\veck'}] = 0$}
\item{$[\hata_{\veck},\adag_{\veck'}] = \delta_{\veck,\veck'}$ (即当且仅当$\veck=\veck'$时为$1$,否则为零)}
\end{itemize}


事实上,$\adag_{\veck}$和$\hata_{\veck}$分别是三维动量为$\veck$的粒子的产生算符和湮灭算符。
\ech
\end{frame}

\begin{frame}
\chtitle{课堂互动}
\bch
在海森堡绘景下,态$|0\rangle$, $|1\rangle$, $\ldots$均不随时间变化,而算符均按海森堡方程随时间变化:
$$ i \frac{d \hat{O}}{dt} =  [\hat{O}, \hat{H}] $$
取$\hat{O}$为单位质量谐振子的产生算符和湮灭算符,$\hat{H} = (\hat{N}+1/2)\omega$, 试证明
$$\frac{d \hata}{dt} = -i\omega \hata, \ \  \frac{d \adag}{dt} =  i\omega \adag$$

\skipline
根据上面的方程我们可以解出
$$\hata(t) = \hata_{t=0} e^{-i\omega t}, \ \ \adag(t) = \adag_{t=0} e^{i\omega t}$$
\ech
\end{frame}

\begin{frame}
\chtitle{实标量场的量子化}
\bch
上述产生算符和湮灭算符的随时间变化规律对$\hata_u(\veck)$, $\hata_\vel(\veck)$等均成立,所以对它们的线性组合$\hata_{\veck}$和$\adag_{\veck}$也成立。
于是我们可以写出海森堡绘景下$\hatphi(\veck)$在任意时刻的表达式:
$$\hatphi(\veck; t) = \frac{\hata_{\veck}e^{-i\omega t} +\adag_{-\veck}e^{i\omega t}}{\sqrt{2\omega\, d^3\veck}}$$
其中右边$\hata$和$\adag$为$t=0$时刻的湮灭和产生算符,为了书写方便我们省略了$t=0$的标注。
\ech
\end{frame}

\begin{frame}
\chtitle{实标量场的量子化}
\bch
最后,我们把$\hatphi(\veck, t)$进行三维空间的反傅立叶变换得到:
$$\hatphi(\vecx, t) = \frac{1}{(2\pi)^{3/2}}\int \frac{d^3\veck}{\sqrt{2\omega\, d^3\veck}} \, e^{i\veck\cdot\vecx}\left(\hata_{\veck}e^{-i\omega t} +\adag_{-\veck}e^{i\omega t}\right)$$
利用四维内积$k_\mu x^\mu = \omega t - \veck\cdot \vecx$我们可以进一步把上面的式子简化写成
$$\hatphi(x) = \frac{1}{(2\pi)^{3/2}} \int \sqrt{\frac{d^3\veck}{2\omega}} \left(\hata_{\veck} e^{-ik_\mu x^\mu} + \adag_{\veck}e^{ik_\mu x^\mu}\right) $$

\skipline
在作业中我们会证明:四维时空中两点的$\hatphi$的对易$\langle \hatphi(x)\hatphi(x') \rangle$是洛仑兹变换下的不变量。
\ech
\end{frame}


\begin{frame}
\chtitle{复标量场}
\bch
复标量场的拉氏密度
$$\langr = \partial_\mu\phi^\dagger \partial^\mu \phi - m^2\phi^\dagger\phi\, .$$
跟实标量场相比少了个$1/2$因子,我们之后会看到这样定义的原因。
\skipline
同样对固定时刻的$\phi$作傅立叶变换之后可得到拉氏量
$$L = \int d^3\veck \, \left[|\dot\phi(\veck)|^2 - \omega^2|\phi(\veck)|^2 \right]\, .$$
\ech
\end{frame}


\begin{frame}
\chtitle{复标量场的量子化}
\bch
对复标量场,我们可以在全$\veck$空间分解$\phi(\veck) = \frac{u(\veck) + i \vel(\veck)}{\sqrt{2}}$而无须担心自由度重复的问题。
$$L = \int d^3\veck\,\left[ \frac{1}{2}\left(\dot u^2 + \dot \vel^2 \right) - \frac{\omega^2}{2}\left(u^2+v^2\right)\right]$$
复标量场没有两个半$\veck$空间的重复求和带来的因子$2$,而开始的$\langr$的定义少了$1/2$因子,所以两者抵消后$u\sqrt{d^3\veck}$, $\vel\sqrt{d^3\veck}$仍为单位质量谐振子。

于是
$$\hat{u}(\veck)  = \frac{\hata_u(\veck) + \adag_u(\veck)}{\sqrt{2\omega\, d^3\veck}},\ \hat{\vel}(\veck)  = \frac{\hata_{\vel}(\veck) + \adag_{\vel}(\veck)}{\sqrt{2\omega\, d^3\veck}}$$

\ech
\end{frame}

\begin{frame}
\chtitle{复标量场的量子化}
\bch

通过重新定义
$$\hata_\veck \equiv \frac{1}{\sqrt{2}}\left[\hata_u(\veck) + i\hata_\vel(\veck)\right],\ \hatb_\veck \equiv\frac{1}{\sqrt{2}}\left[\hata_u(-\veck) -i \hata_\vel(-\veck)\right]$$
就得到
$$\hatphi(\veck) = \frac{\hata_{\veck}+\bdag_{-\veck}}{\sqrt{2\omega\, d^3\veck}}$$

\ech
\end{frame}

\begin{frame}
\chtitle{课堂互动}
\bch
\begin{itemize}
\item{证明:除了$[\hata_\veck,\adag_{\veck'}]=[\hatb_\veck,\bdag_{\veck'}]=\delta_{\veck,\veck'}$,其余$\hata$, $\adag$, $\hatb$, $\bdag$之间均两两对易。也就是说:$\hata$和$\hatb$代表三维动量为$\veck$的两种不同粒子的湮灭算符。}
\item{对上述复标量场$\hatphi$证明$$\hatphi(x) = \frac{1}{(2\pi)^{3/2}} \int \sqrt{\frac{d^3\veck}{2\omega}} \left(\hata_{\veck} e^{-ik_\mu x^\mu} + \bdag_{\veck}e^{ik_\mu x^\mu}\right) $$}
\end{itemize}
\ech
\end{frame}

\begin{frame}
\chtitle{复标量场的规范变换和守恒荷}
\bch
复标量场的作用量显然在变换$\phi \rightarrow \phi e^{iq\epsilon}$下是严格的不变量($q$为任意实常数,$\epsilon\rightarrow 0^+$),所以必然有一个守恒量与之对应。在这个变换下:
$\frac{\delta \phi}{\delta \epsilon} = iq\phi$, $\frac{\delta\phi^\dagger}{\delta\epsilon} = - i q\phi^\dagger$, $\delta \langr = 0$ (即Noether定理中的$F^\mu = 0$)。按照Noether定理:
$$j^\mu = iq\left(\phi^\dagger\partial^\mu\phi-\phi\partial^\mu\phi^\dagger\right)$$
是个守恒流。对三维空间积分我们得到复标量场的守恒电荷:
$$Q = iq\int d^3\vecx\, \left(\phi^\dagger \dot\phi  - \phi \dot\phi^\dagger\right)\,.$$

\skipline
变换$\phi \rightarrow \phi e^{i\epsilon}$又称为规范变换,下面我们从量子场的观点来讨论复标量场的守恒荷。

\ech
\end{frame}

\begin{frame}
\chtitle{复标量场的守恒荷}
\bch
根据实空间的内积等价于傅立叶空间的内积的数学定理(请回顾上节课内容),守恒荷的表达式可以写成
$$Q = iq \int d^3\veck \left(\phi^\dagger(\veck) \dot\phi (\veck) - \phi (\veck)\dot\phi^\dagger(\veck)\right)\, .$$
把$\hatphi = \frac{\hata_\veck + \bdag_{-\veck}}{\sqrt{2\omega\, d^3\veck}}$代入上式,并利用前面得到的$d \hata_\veck/dt = -i\omega\hata_\veck$, $d \hatb_\veck/dt = -i\omega\hatb_\veck$, $d\adag_\veck/dt = i\omega\adag_\veck$, $d\bdag_\veck/dt = i\omega \bdag_\veck$  就得到
$$Q =q\sum_{\veck} \left(\adag_\veck\hata_\veck - \bdag_{\veck}\hatb_{\veck}\right)$$

我们看到$a$和$b$实际上是荷相反的两种粒子(互为反粒子),故有荷的守恒律。而实数标量场的粒子是中性的,所以没有守恒荷。
\ech
\end{frame}

\begin{frame}
\chtitle{定域规范变换}
\bch
思考:如果把上述规范变换的$q$替换为一个依赖时空坐标的函数$\gamma(x)$(这样的规范变换称为定域规范变换),拉氏密度和作用量就都不是不变的了。怎样修改拉氏密度可以使作用量不变?
\ech
\end{frame}

\end{document}
