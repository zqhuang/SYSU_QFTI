\documentclass[CJK]{beamer}
\usepackage{CJKutf8}
\usepackage{beamerthemesplit}
\usetheme{Malmoe}
\useoutertheme[footline=authortitle]{miniframes}
\usepackage{amsmath}
\usepackage{amssymb}
\usepackage{graphicx}
\usepackage{color}
\graphicspath{{figures/}}
\def \bch {\begin{CJK}{UTF8}{gbsn}}
\def \ech {\end{CJK}}
\def \bex {\begin{minipage}{0.3\textwidth}\includegraphics[width=1in]{jugelizi.png}\end{minipage}\begin{minipage}{0.6\textwidth}}
\def \eex {\end{minipage}}
\def \chtitle#1 {\frametitle{\bch #1 \ech}}
\def \skipline { {\vskip 0.1in}}
\def \langr {\mathcal{L}}
\def \hamil {\mathcal{H}}
\def \vecx {\mathbf{x}}
\def \veck {\mathbf{k}}
\def \vecp {\mathbf{p}}
\def \hatphi {\hat{\phi}}
\def \hatq {\hat{q}}
\def \hatpi  {\hat{\pi}}
\def \vel {\upsilon}
\def \Dint {\mathcal{D}}
\def \adag {{\hat{a}^\dagger}}
\def \hata {\hat{a}}
\def \hatN {\hat{N}}
\def \hatH {\hat{H}}
\def \nket { {| n \rangle}}
\def \bran { {\langle n |}}

\title{Quantum Field Theory I \\ Lesson 02 - Action Princple}
  \author{}
  \date{}


\begin{document}

\begin{frame}
 
\begin{center}
\begin{Large}
\bch
量子场论 I 

{\vskip 0.3in}

第三次课后作业 (共八次,每次2.5分)

交作业时间: 10月24日,星期一,13:30pm

\ech
\end{Large}
\end{center}

\vskip 0.2in

\bch
课件下载
\ech
https://github.com/zqhuang/SYSU\_QFTI

\end{frame}

\begin{frame}
\chtitle{第1题(0.5分)}
\bch
证明自由$U(1)$规范场$A^\mu$在库仑规范$A^0=0, \nabla\cdot \vecA=0$下的Hamilton密度为
$$\hamil = \frac{1}{2}|\dot\vecA|^2 + \frac{1}{2}|\nabla\times \vecA|^2$$
并证明在傅立叶空间Hamilton量可以写成
$$H = \int d^3\veck\left[\frac{1}{2}|\dot\vecA|^2 + \frac{1}{2}|\veck \times \vecA|^2\right]$$
\ech
\end{frame}

\begin{frame}
\chtitle{第2题(0.5分)}
\bch
利用上题的Hamilton量的表达式以及我们在课上得到的$\hat{\vecA}$在傅立叶空间的解:
$$\hat{\vecA} = \frac{1}{\sqrt{2|\veck|d^3\veck}}\sum_{s=\pm 1} \left(\hata_{\veck, s} + \adag_{\veck, s}\right)$$
证明Hamilton量的算符表达式为
$$\hat{H} = \sum_{\veck} \sum_{s=\pm 1}(\hat{N}_{\veck,s}+1/2)|\veck|$$
其中 $\hat{N}_{\veck, s} \equiv \adag_{\veck,s}\hata_{\veck, s}$是动量为$\veck$,自旋为$s$的粒子数算符。
\ech
\end{frame}

\begin{frame}
\chtitle{第3题(0.5分)}
\bch
证明自由$U(1)$规范场$A^\mu$在库仑规范$A^0=0, \nabla\cdot \vecA=0$下的动量
$$P^i = \int d^3\vecx\, (\partial^0\vecA \cdot \partial^i\vecA)$$
是守恒量。并把它写成傅立叶空间的积分。
\skipline
\skipline

提示:用Nother定理和三个方向的空间平移不变性。 
\ech
\end{frame}


\begin{frame}
\chtitle{第4题(0.5分)}
\bch
利用上题的动量$P^i$的表达式以及我们在课上得到的$\hat{\vecA}$在傅立叶空间的解:
$$\hat{\vecA} = \frac{1}{\sqrt{2|\veck|d^3\veck}}\sum_{s=\pm 1} \left(\hata_{\veck, s} + \adag_{\veck, s}\right)$$
证明动量$\vecP \equiv (P^1, P^2, P^3)$的算符表达式为
$$\hat{\vecP} = \sum_{\veck}\sum_{s = \pm 1} \veck \hat{N}_{\veck, s}$$
其中 $\hat{N}_{\veck, s} \equiv \adag_{\veck,s}\hata_{\veck, s}$是动量为$\veck$,自旋为$s$的粒子数算符。
\ech
\end{frame}


\begin{frame}
\chtitle{第5题(0.5分)}
\bch
对旋量$\psi$,证明$\bar\psi\slashed{\partial}\psi$是标量。
\ech
\end{frame}


\end{document}
