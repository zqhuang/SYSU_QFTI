\documentclass[CJK]{beamer}
\usepackage{CJKutf8}
\usepackage{beamerthemesplit}
\usetheme{Malmoe}
\useoutertheme[footline=authortitle]{miniframes}
\usepackage{amsmath}
\usepackage{amssymb}
\usepackage{graphicx}
\usepackage{color}
\graphicspath{{figures/}}
\def \bch {\begin{CJK}{UTF8}{gbsn}}
\def \ech {\end{CJK}}
\def \bex {\begin{minipage}{0.3\textwidth}\includegraphics[width=1in]{jugelizi.png}\end{minipage}\begin{minipage}{0.6\textwidth}}
\def \eex {\end{minipage}}
\def \chtitle#1 {\frametitle{\bch #1 \ech}}
\def \skipline { {\vskip 0.1in}}
\def \langr {\mathcal{L}}
\def \hamil {\mathcal{H}}
\def \vecx {\mathbf{x}}
\def \veck {\mathbf{k}}
\def \vecp {\mathbf{p}}
\def \hatphi {\hat{\phi}}
\def \hatq {\hat{q}}
\def \hatpi  {\hat{\pi}}
\def \vel {\upsilon}
\def \Dint {\mathcal{D}}
\def \adag {{\hat{a}^\dagger}}
\def \hata {\hat{a}}
\def \hatN {\hat{N}}
\def \hatH {\hat{H}}
\def \nket { {| n \rangle}}
\def \bran { {\langle n |}}

\title{Quantum Field Theory I \\ Lesson 12 - The First Feynman diagram}
\author{}
\date{}


\begin{document}

\begin{frame}
 
\begin{center}
\begin{Large}
\bch
量子场论 I 

{\vskip 0.3in}

第十二课 第一个Feynman图

\ech
\end{Large}
\end{center}

\vskip 0.2in

\bch
课件下载
\ech
https://github.com/zqhuang/SYSU\_QFTI

\end{frame}


\begin{frame} 
\chtitle{相互作用(Interaction)绘景} 
\bch
假设Hamilton算符可以拆成两部分: $\hat{H} = \hat{H}_0 + \hat{H}_I$, 则在各种绘景下态和可观测量算符的演化方程为:

\begin{tabular}{c|c|c|c}
\hline
\hline
            & Schr\"odinger & Heisenberg  & Interaction \\
\hline
State         & $\frac{d |\psi\rangle}{dt} = - \ii \hat{H}|\psi\rangle$ & constant & $\frac{d |\psi\rangle}{dt} = - \ii \hat{H}_I|\psi\rangle$  \\
\hline
Observable & const. & $\frac{d\hat{O}}{dt} = -\ii [ \hat{O}, \hat{H} ]$ & $\frac{d\hat{O}}{dt} = -\ii [ \hat{O}, \hat{H}_0 ]$ \\
\hline
\end{tabular}

注意
\begin{itemize}
\item{不可观测量算符不一定遵循上述变化规律,例如密度矩阵$|\psi\rangle\langle\psi|$需要按态的演化方程去计算,而不遵循可观测量的算符演化方程。}
\item{Schr\"odinger绘景和Heisenberg绘景都是Interaction绘景的特例,前者是取了$\hat{H}_0 = 0$,后者是取了$\hat{H}_I = 0$}
\end{itemize}

\ech
\end{frame}

\begin{frame} 
\chtitle{证明一切绘景等价} 
\bch
{\small 
我们只要证明一切Interaction绘景和Sch\"odinger绘景等价,Heisenberg绘景作为Interaction绘景的一种特殊情况则无须再额外证明。所谓两个绘景等价是指任何可观测量的矩阵元$\langle \psi_1 | \hat{O}|\psi_2\rangle$在两个绘景里都相同。
\skipline

把Sch\"odinger绘景下的可观测算符记为$(\hat{O})_S$, 态记为$|\psi\rangle_S$。Interaction绘景下的可观测算符记为$(\hat{O})_I$, 态记为$|\psi\rangle_I$。假设在$t=0$时刻两个绘景下的可观测算符和态均相同,之后分别按各自绘景下的演化方程进行演化。
\begin{itemize}
\item{证明$(\hatH_0)_I$不随时间变化,从而有$(\hatH_0)_I=(\hatH_0)_S$。 对$H_0$我们无须再注明是哪个绘景。}
\item{证明$(\hat{O})_I = e^{\ii\hatH_0 t}(\hat{O})_S e^{-\ii\hatH_0 t}$, 特别地$(\hatH_I)_I= e^{\ii\hatH_0 t}(\hatH_I)_S e^{-\ii\hatH_0 t}$.}
\item{证明$|\psi\rangle_I = e^{\ii\hatH_0 t} |\psi\rangle_S $}
\item{证明对任何态$|\psi_1\rangle$和$|\psi_2\rangle$,$\langle \psi_1 | \hat{O}|\psi_2\rangle$在两个绘景下相同。}
\end{itemize}
}
\ech
\end{frame}

\begin{frame} 
\chtitle{相互作用绘景的形式解} 
\bch
证明相互作用绘景下
$$|\psi\rangle_{t_2} = e^{-\ii\int_{t_1}^{t_2} \hatH_I(t') dt'} |\psi\rangle_{t_1}$$
\ech
\end{frame}


\begin{frame} 
\chtitle{有自相互作用的标量场} 
\bch
考虑拉氏密度为
$$\lagr = \frac{1}{2}\partial^\mu\phi\partial_\mu\phi - \frac{1}{2}m^2\phi^2 - \frac{\lambda}{4!}\phi^4$$
的实标量场。其中$\lambda>0$是耦合常数。我们仅考虑它很小($\lambda \ll 1$)的情况。

显然,跟自由场相比,Hamilton量多了一项正比于$\lambda$的相互作用项。我们取如下的相互作用表象:
$$ H_0 = H_{\rm free},\ H_I = \frac{\lambda}{4!} \int d^3\vecx\, \phi^4$$
其中$H_{\rm free} = \int d^3\vecx\, \frac{1}{2}(\dot\phi^2 + |\nabla \phi|^2 + m^2\phi^2)$是我们以前学过的自由场的Hamilton量。

\ech
\end{frame}

\begin{frame} 
\chtitle{热身问题} 
\bch
我们的热身问题是:

求两个动量为$\vecp_1$, $\vecp_2$的粒子发生散射,变为两个动量为$\vecp_3$, $\vecp_4$的粒子的概率幅。
$$\calM T/V= \langle \vecp_3, \vecp_4|  e^{-\ii\int_{-\infty}^\infty \hat{H}_I dt}|\vecp_1,\vecp_2\rangle$$
空间总体积$V\rightarrow \infty$和总时间长度$T\rightarrow \infty$。我们在定义概率振幅时引入了$T/V$的因子来描述空间越大两个自由粒子越难碰到发生散射,以及时间越长越越容易发生散射。在最后的表达式中两边会约去$T/V$。

为了讨论简单起见,我们假设这四个动量$\vecp_1$, $\vecp_2$, $\vecp_3$, $\vecp_4$互不相同。
\ech
\end{frame}

\begin{frame} 
\chtitle{一阶微扰展开} 
\bch

我们做一阶微扰展开:
$$e^{-\ii\int_{-\infty}^\infty \hat{H}_I dt}\approx 1 -\ii\int_{-\infty}^\infty \hat{H}_I dt = 1-\ii\int d^4x\, \frac{\lambda}{4!}\hat\phi^4 $$

若末态与初态不同则第一项$1$没有贡献。我们来计算最低阶的非零贡献:
$$\calM T/V \approx -\ii \langle \vecp_3, \vecp_4| \int d^4x\, \frac{\lambda}{4!}\hat\phi^4 |\vecp_1,\vecp_2\rangle$$

\ech
\end{frame}

\begin{frame} 
\chtitle{写成产生湮灭算符} 
\bch

{\small
由于该相互作用表象下$\hatphi$仍按$d\hat\phi/dt = - \ii [\hatphi,  \hatH_{\rm free}]$演化,$\hatphi$的解就和自由场相同
$$\hatphi = \frac{1}{(2\pi)^{3/2}}\int \sqrt{\frac{d^3\veck}{2\omega}} \left(\hata_{\veck}e^{-ikx}+\adag_{\veck}e^{ikx}\right)$$
代入要求解的概率幅后得到
\be
\calM T/V \approx \frac{-\ii\lambda}{4!(2\pi)^6} \int d^4x \langle \vecp_3, \vecp_4|\left[\int \sqrt{\frac{d^3\veck}{2\omega}} \left(\hata_{\veck}e^{-ikx}+\adag_{\veck}e^{ikx}\right)\right]^4 |\vecp_1,\vecp_2\rangle
\ee
显然,要使得矩阵元非零,在四个括号里必须分别取出动量为$\vecp_1$, $\vecp_2$的湮灭算符以及动量为$\vecp_3$, $\vecp_4$的产生算符。这样一共有$4!$种取法,刚好和外面分母里的$4!$抵消了,于是得到:
\be
\calM T/V \approx \frac{-\ii\lambda}{(2\pi)^6} \int d^4x\, \frac{(d^3\veck)^2}{4\sqrt{\omega_1\omega_2\omega_3\omega_4}} e^{-i(p_1+p_2-p_3-p_4)x} 
\ee
}
\ech
\end{frame}

\begin{frame} 
\chtitle{化简} 
\bch
{\small
利用四维空间的积分公式:
$$\frac{1}{(2\pi)^4}\int d^4 x\, e^{-i(p_1+p_2-p_3-p_4)x} = \delta(p_1+p_2-p_3-p_4)$$
我们最终得到:
\be
\calM T/V \approx \frac{-\ii\lambda}{(2\pi)^2} \frac{(d^3\veck)^2}{4\sqrt{\omega_1\omega_2\omega_3\omega_4}} \delta(p_1+p_2-p_3-p_4)
\ee
注意到$d^3\veck = (2\pi)^3/V$(例如考虑一个边长为$L$的立方盒子),以及$d^4k = (2\pi/T)d^3\veck$,我们可以把上式写成
\be
\calM \approx -\ii\lambda \frac{1}{4\sqrt{\omega_1\omega_2\omega_3\omega_4}} \delta(p_1+p_2-p_3-p_4)d^4k
\ee
}
\ech
\end{frame}

\begin{frame} 
\chtitle{对结果的物理讨论} 
\bch
{\small
结果中的$\delta(p_1+p_2-p_3-p_4)d^4k$在$p_1+p_2-p_3-p_4=0$时为1,否则为零。这是散射问题能量动量守恒的自然结果。一般定义散射振幅时会把这个能量动量守恒因子排除在外,按这样的定义方法,最后结果就是:
\be
\calM \approx -\ii\lambda \frac{1}{4\sqrt{\omega_1\omega_2\omega_3\omega_4}} 
\ee
}
\ech
\end{frame}

\begin{frame} 
\chtitle{这个问题对应的Feynman图和Feynman规则} 
\bch
\begin{minipage}{0.45\textwidth}
\includegraphics[width=2in]{Feynman_lf4_tree.png}
\end{minipage}
\begin{minipage}{0.45\textwidth}
实标量场$\frac{\lambda}{4!}\phi^4$相互作用项的Feynman规则:
\begin{itemize}
\item{四线交叉顶角给出因子$-\ii\lambda$}
\item{每条外线给出因子$1/\sqrt{2\omega}$}
\end{itemize}
\end{minipage}

\ech
\end{frame}

\begin{frame}
\chtitle{选择题时间} 
\bch
我们算完了人生中第一个Feynman图,你的感想是:
\begin{itemize}
\item[A]{无敌是多么寂寞}
\item[B]{老师你故意找个最好算的图逗我们的吧!}
\item[C]{刚才发生了什么……}
\end{itemize}
\ech
\end{frame}

\begin{frame}
\chtitle{答案是B} 
\bch

下节课我们要动真格的了。
\ech
\end{frame}

\end{document}


