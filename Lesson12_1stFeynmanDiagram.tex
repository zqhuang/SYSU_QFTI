\documentclass[CJK]{beamer}
\usepackage{CJKutf8}
\usepackage{beamerthemesplit}
\usetheme{Malmoe}
\useoutertheme[footline=authortitle]{miniframes}
\usepackage{amsmath}
\usepackage{amssymb}
\usepackage{graphicx}
\usepackage{color}
\graphicspath{{figures/}}
\def \bch {\begin{CJK}{UTF8}{gbsn}}
\def \ech {\end{CJK}}
\def \bex {\begin{minipage}{0.3\textwidth}\includegraphics[width=1in]{jugelizi.png}\end{minipage}\begin{minipage}{0.6\textwidth}}
\def \eex {\end{minipage}}
\def \chtitle#1 {\frametitle{\bch #1 \ech}}
\def \skipline { {\vskip 0.1in}}
\def \langr {\mathcal{L}}
\def \hamil {\mathcal{H}}
\def \vecx {\mathbf{x}}
\def \veck {\mathbf{k}}
\def \vecp {\mathbf{p}}
\def \hatphi {\hat{\phi}}
\def \hatq {\hat{q}}
\def \hatpi  {\hat{\pi}}
\def \vel {\upsilon}
\def \Dint {\mathcal{D}}
\def \adag {{\hat{a}^\dagger}}
\def \hata {\hat{a}}
\def \hatN {\hat{N}}
\def \hatH {\hat{H}}
\def \nket { {| n \rangle}}
\def \bran { {\langle n |}}

\title{Quantum Field Theory I \\ Lesson 12 - The First Feynman diagram}
\author{}
\date{}


\begin{document}

\begin{frame}
 
\begin{center}
\begin{Large}
\bch
量子场论 I 

{\vskip 0.3in}

第十二课 第一个Feynman图

\ech
\end{Large}
\end{center}

\vskip 0.2in

\bch
课件下载
\ech
https://github.com/zqhuang/SYSU\_QFTI

\end{frame}


\begin{frame} 
\chtitle{相互作用(Interaction)绘景} 
\bch
假设Hamilton算符可以拆成两部分: $\hat{H} = \hat{H}_0 + \hat{H}_I$, 则在各种绘景下态和可观测量算符的演化方程为:

\begin{tabular}{c|c|c|c}
\hline
\hline
            & Schr\"odinger & Heisenberg  & Interaction \\
\hline
State         & $\frac{d |\psi\rangle}{dt} = - \ii \hat{H}|\psi\rangle$ & constant & $\frac{d |\psi\rangle}{dt} = - \ii \hat{H}_I|\psi\rangle$  \\
\hline
Observable & const. & $\frac{d\hat{O}}{dt} = -\ii [ \hat{O}, \hat{H} ]$ & $\frac{d\hat{O}}{dt} = -\ii [ \hat{O}, \hat{H}_0 ]$ \\
\hline
\end{tabular}

注意
\begin{itemize}
\item{非可观测量算符不一定遵循上述变化规律,例如密度矩阵$|\psi\rangle\langle\psi|$需要按态的演化方程去计算,而不遵循可观测量的算符演化方程。}
\item{Schr\"odinger绘景和Heisenberg绘景都是Interaction绘景的特例,前者是取了$\hat{H}_0 = 0$,后者是取了$\hat{H}_I = 0$}
\end{itemize}

\ech
\end{frame}

\begin{frame} 
\chtitle{证明一切绘景等价} 
\bch
{\small 
我们只要证明一切Interaction绘景和Sch\"odinger绘景等价,Heisenberg绘景作为Interaction绘景的一种特殊情况则无须再额外证明。所谓两个绘景等价是指任何可观测量的矩阵元$\langle \psi_1 | \hat{O}|\psi_2\rangle$在两个绘景里都相同。
\skipline

把Sch\"odinger绘景下的可观测算符记为$(\hat{O})_S$, 态记为$|\psi\rangle_S$。Interaction绘景下的可观测算符记为$(\hat{O})_I$, 态记为$|\psi\rangle_I$。假设在$t=0$时刻两个绘景下的可观测算符和态均相同,之后分别按各自绘景下的演化方程进行演化。
\begin{itemize}
\item{证明$(\hatH_0)_I$不随时间变化,从而有$(\hatH_0)_I=(\hatH_0)_S$。 对$H_0$我们无须再注明是哪个绘景。}
\item{证明$(\hat{O})_I = e^{\ii\hatH_0 t}(\hat{O})_S e^{-\ii\hatH_0 t}$, 特别地$(\hatH_I)_I= e^{\ii\hatH_0 t}(\hatH_I)_S e^{-\ii\hatH_0 t}$.}
\item{证明$|\psi\rangle_I = e^{\ii\hatH_0 t} |\psi\rangle_S $}
\item{证明对任何态$|\psi_1\rangle$和$|\psi_2\rangle$,$\langle \psi_1 | \hat{O}|\psi_2\rangle$在两个绘景下相同。}
\end{itemize}
}
\ech
\end{frame}

\begin{frame} 
\chtitle{有自相互作用的标量场} 
\bch
考虑拉氏密度为
$$\lagr = \frac{1}{2}\partial^\mu\phi\partial_\mu\phi - \frac{1}{2}m^2\phi^2 - \frac{\lambda}{4!}\phi^4$$
的实标量场。
 
\ech
\end{frame}

\begin{frame} 
\chtitle{} 
\bch

\ech
\end{frame}

\begin{frame} 
\chtitle{} 
\bch

\ech
\end{frame}

\end{document}


