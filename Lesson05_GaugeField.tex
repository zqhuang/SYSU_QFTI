\documentclass[CJK]{beamer}
\usepackage{CJKutf8}
\usepackage{beamerthemesplit}
\usetheme{Malmoe}
\useoutertheme[footline=authortitle]{miniframes}
\usepackage{amsmath}
\usepackage{amssymb}
\usepackage{graphicx}
\usepackage{color}
\graphicspath{{figures/}}
\def \bch {\begin{CJK}{UTF8}{gbsn}}
\def \ech {\end{CJK}}
\def \bex {\begin{minipage}{0.3\textwidth}\includegraphics[width=1in]{jugelizi.png}\end{minipage}\begin{minipage}{0.6\textwidth}}
\def \eex {\end{minipage}}
\def \chtitle#1 {\frametitle{\bch #1 \ech}}
\def \skipline { {\vskip 0.1in}}
\def \langr {\mathcal{L}}
\def \hamil {\mathcal{H}}
\def \vecx {\mathbf{x}}
\def \veck {\mathbf{k}}
\def \vecp {\mathbf{p}}
\def \hatphi {\hat{\phi}}
\def \hatq {\hat{q}}
\def \hatpi  {\hat{\pi}}
\def \vel {\upsilon}
\def \Dint {\mathcal{D}}
\def \adag {{\hat{a}^\dagger}}
\def \hata {\hat{a}}
\def \hatN {\hat{N}}
\def \hatH {\hat{H}}
\def \nket { {| n \rangle}}
\def \bran { {\langle n |}}

\title{Quantum Field Theory I \\ Lesson 05 - Gauge Field}
\author{}
\date{}


\begin{document}

\begin{frame}
 
\begin{center}
\begin{Large}
\bch
量子场论 I 

{\vskip 0.3in}

第五课 定域规范变换和规范场

\ech
\end{Large}
\end{center}

\vskip 0.2in

\bch
课件下载
\ech
https://github.com/zqhuang/SYSU\_QFTI

\end{frame}



\begin{frame}
\chtitle{定域规范变换(Local Gauge Transformation)}
\bch
上节课讲到,局域规范变换$\phi \rightarrow \phi e^{i\gamma(x)}$会改变不同时空点的$\phi$的相位差,从而改变作用量。联系不同时空点的$\phi$的算符为动量算符$-i\partial_\mu$,一个很自然的想法就是,能否把拉氏密度中的$-i\partial_\mu$算符替换成另一个矢量算符而使得作用量不变?

\skipline
答案是肯定的,通过定义“正则动量算符”:
$$-iD_\mu \equiv -i\partial_\mu + q A_\mu$$
并规定矢量场$A_\mu$在定域规范变换下按如下的规则进行变换:
$$ A_\mu \rightarrow A_\mu - \frac{1}{q} \partial_\mu \gamma\, ,$$
就可以满足作用量不变的要求。

\ech
\end{frame}

\begin{frame}
\chtitle{课堂互动}
\bch
按上述定义
$$D_\mu \equiv \partial_\mu + iq A_\mu$$
证明在定域规范变换$\phi \rightarrow e^{i\gamma(x)}\phi$, $A_\mu \rightarrow A_\mu - \frac{1}{q}\partial\gamma$下,拉氏密度
$$\langr = D_\mu\phi^\dagger D^\mu\phi - m^2\phi^\dagger \phi$$
是不变量。因此作用量也是不变量。

\skipline
$\phi \rightarrow e^{i\gamma(x)}\phi$在数学上称为一维幺正变换,简写为$U(1)$。与之对应的规范场$A_\mu$就叫做$U(1)$规范场。群论中U(1)属于阿贝尔群,所以$A_\mu$又称阿贝尔规范场。
\ech
\end{frame}

\begin{frame}
\chtitle{是否自寻烦恼?}
\bch
思考:为了使作用量形式上不变而引进的$A_\mu$场是“自寻烦恼”吗?
\skipline
关键在于:能否选取适当的局域规范变换使得$A_\mu$处处为零?
\ech
\end{frame}

\begin{frame}
\chtitle{课堂互动}
\bch
试证明:
\begin{itemize}
\item{如果存在$\gamma(x)$使得$A_\mu - \frac{1}{q}\partial_\mu\gamma = 0$处处成立,则
$F^{\mu\nu} \equiv \partial^\mu A^\nu - \partial^\nu A^\mu$ 处处为零。}
\item{若$F^{\mu\nu} \equiv \partial^\mu A^\nu - \partial^\nu A^\mu$处处为零,则存在$\gamma(x)$使得$A_\mu - \frac{1}{q}\partial_\mu\gamma = 0$处处成立}
\end{itemize}
\ech
\end{frame}

\begin{frame}
\chtitle{}
\bch
\ech
\end{frame}

\begin{frame}
\chtitle{}
\bch
\ech
\end{frame}

\begin{frame}
\chtitle{}
\bch
\ech
\end{frame}

\begin{frame}
\chtitle{}
\bch
\ech
\end{frame}

\begin{frame}
\chtitle{}
\bch
\ech
\end{frame}

\begin{frame}
\chtitle{}
\bch
\ech
\end{frame}

\begin{frame}
\chtitle{}
\bch
\ech
\end{frame}

\end{document}
