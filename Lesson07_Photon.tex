\documentclass[CJK]{beamer}
\usepackage{CJKutf8}
\usepackage{beamerthemesplit}
\usetheme{Malmoe}
\useoutertheme[footline=authortitle]{miniframes}
\usepackage{amsmath}
\usepackage{amssymb}
\usepackage{graphicx}
\usepackage{color}
\graphicspath{{figures/}}
\def \bch {\begin{CJK}{UTF8}{gbsn}}
\def \ech {\end{CJK}}
\def \bex {\begin{minipage}{0.3\textwidth}\includegraphics[width=1in]{jugelizi.png}\end{minipage}\begin{minipage}{0.6\textwidth}}
\def \eex {\end{minipage}}
\def \chtitle#1 {\frametitle{\bch #1 \ech}}
\def \skipline { {\vskip 0.1in}}
\def \langr {\mathcal{L}}
\def \hamil {\mathcal{H}}
\def \vecx {\mathbf{x}}
\def \veck {\mathbf{k}}
\def \vecp {\mathbf{p}}
\def \hatphi {\hat{\phi}}
\def \hatq {\hat{q}}
\def \hatpi  {\hat{\pi}}
\def \vel {\upsilon}
\def \Dint {\mathcal{D}}
\def \adag {{\hat{a}^\dagger}}
\def \hata {\hat{a}}
\def \hatN {\hat{N}}
\def \hatH {\hat{H}}
\def \nket { {| n \rangle}}
\def \bran { {\langle n |}}

\title{Quantum Field Theory I \\ Lesson 06 - Photon}
\author{}
\date{}


\begin{document}

\begin{frame}
 
\begin{center}
\begin{Large}
\bch
量子场论 I 

{\vskip 0.3in}

第七课 光子

\ech
\end{Large}
\end{center}

\vskip 0.2in

\bch
课件下载
\ech
https://github.com/zqhuang/SYSU\_QFTI

\end{frame}



\begin{frame}
\chtitle{$U(1)$规范场的量子化}
\bch
上节课我们留了一个问题还没有解决:自由$U(1)$规范场的粒子是怎样完成量子化的,这种粒子和谐振子又有何不同?

\skipline
这个问题的主要难点是每个时空点的$A^\mu$场仅有两个物理自由度,一旦我们要对这两个自由度进行量子化,就要取特定的规范来固定这两个自由度。不难想象,这样的推导并不会十分“优美”。

\ech
\end{frame}

\begin{frame}
\chtitle{第一种方案:库仑规范}
\bch
先试试库仑规范$A^0 = 0$, $\nabla \cdot \vecA = 0$.
这样$F_{00} = 0$, $F_{0i} = \partial_0 A_i$

$$ \lagr = -\frac{1}{4} \Fup\Fdown = \frac{1}{2} |\dot\vecA|^2  - \frac{1}{2}|\nabla \times \vecA|^2$$

对固定时间的拉氏量进行傅立叶变换后得到:
$$L = \int d^3\veck \left[\frac{1}{2} |\dot\vecA_\veck|^2  - \frac{1}{2}|\veck \times \vecA_\veck|^2\right]$$
我们对一个固定的$\veck$进行研究。

\ech
\end{frame}

\begin{frame}
\chtitle{第一种方案:库仑规范}
\bch
对固定的$\veck$,我们可以取三个方向$\vecn_1$, $\vecn_2$, $\vecn_3 = \veck / |\veck|$构成右手正交系。库仑规范要求$\vecA\cdot\veck = 0$, 所以$\vecA$可分解为$\vecA = u_{\veck,+1} \vece_{\veck,+1} + u_{\veck,-1} \vece_{\veck,-1}$。其中$\vece_{\veck,\pm 1} = \frac{1}{\sqrt{2}}(\vecn_1\pm \ii \vecn_2)$是$\hat{J}_{\vecn_3}$的本征值为$\pm 1$的归一化本征矢(参考上节课角动量的内容,添加了$1/\sqrt{2}$因子是为了归一化)。

利用正交归一条件$\vece_{\veck,+1}^\dagger \cdot \vece_{\veck,+1} = \vece_{\veck,-1}^\dagger \cdot \vece_{\veck,-1} = 1$, $\vece_{\veck,+1}^\dagger \cdot \vece_{\veck,-1} = \vece_{\veck,-1}^\dagger \cdot \vece_{\veck,+1} = 0$,可以得到:

$$|\dot\vecA_\veck|^2 = \dot\vecA_\veck^\dagger \cdot \dot\vecA_\veck = |\dot u_{\veck,+1}|^2 + |\dot u_{\veck,-1}|^2$$

\ech
\end{frame}


\begin{frame}
\chtitle{第一种方案:库仑规范}
\bch
注意到$\veck \times \frac{1}{\sqrt{2}}(\vecn_1\pm \ii\vecn_2) =|\veck|  \frac{1}{\sqrt{2}}(\vecn_2 \mp \ii \vecn_1)$, 而$ \frac{1}{\sqrt{2}}(\vecn_2 \mp \ii \vecn_1)$仍然是正交归一的两个矢量,于是得出
$$ |\veck \times \vecA_\veck|^2 = |\veck|^2(|u_{\veck,+1}|^2 + |u_{\veck,-1}|^2)$$

最终得到$\vecA_\veck$对拉氏量的贡献为:
$$ L_\veck = \frac{d^3\veck}{2}\sum_{s=\pm 1}  |\dot u_{\veck,s}|^2  - |\veck|^2 |u_{\veck,s}|^2$$

\ech
\end{frame}


\begin{frame}
\chtitle{第一种方案:库仑规范}
\bch

我们约定如果对$\veck$取了右手正交系$\vecn_1$, $\vecn_2$, $\veck/|\veck|$,则对$-\veck$就取$\vecn_1$, $-\vecn_2$, $-\veck/|\veck|$为正交右手系。那么对$s=\pm 1$都有
$$\vece_{-\veck, s}  = \vece_{\veck, s}^*$$

这样$\vecA$是实数场,也就是$\vecA_{-\veck} = \vecA_\veck^*$的条件就转化为$u_{-\veck, s} = u_{\veck, s}^*$。也就是说$u_{\veck, s}$ 可以看成一个实数场$u_s(\vecx)$的傅立叶变换。
\ech
\end{frame}

\begin{frame}
\chtitle{第一种方案:库仑规范}
\bch

这样,$u_s$的拉氏量就和质量为零的实标量场的拉氏量完全一样了。我们可以直接写出
$$ u_{\veck,s} = \frac{\hata_{\veck,s} + \adag_{-\veck,s}}{\sqrt{2 |\veck| d^3\veck}}$$。

最终进行反傅立叶变换,得到$\hat{\vecA}$的解为:
$$\hat{\vecA}(x) = \frac{1}{(2\pi)^{3/2}} \int \sqrt{\frac{d^3\veck}{2|\veck|}} \sum_{s=\pm 1}\mathbf{e}_{\veck s}\left(\hata_{\veck s} e^{-ik_\mu x^\mu} + \adag_{\veck s}e^{ik_\mu x^\mu}\right) $$
其中$\vece_{\veck,\pm 1} = \vecn_1 \pm \ii \vecn_2$, $\vecn_1$, $\vecn_2$, $\veck/|\veck|$是构成正交右手系的单位矢量,并如前所述有约定$\vece_{-\veck, s} = \vece_{\veck,s}^*$。 

\ech
\end{frame}

\begin{frame}
\chtitle{第一种方案:库仑规范}
\bch
上述推导中,我们顺便证明了U(1)规范场的粒子(例如光子)的质量为零,自旋为$1$。
\ech
\end{frame}

\begin{frame}
\chtitle{第二种方案:重矢量场}
拉氏密度里加质量项:
$$\lagr
\end{document}
