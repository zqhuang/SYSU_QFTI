\documentclass[CJK]{beamer}
\usepackage{CJKutf8}
\usepackage{beamerthemesplit}
\usetheme{Malmoe}
\useoutertheme[footline=authortitle]{miniframes}
\usepackage{amsmath}
\usepackage{amssymb}
\usepackage{graphicx}
\usepackage{color}
\graphicspath{{figures/}}
\def \bch {\begin{CJK}{UTF8}{gbsn}}
\def \ech {\end{CJK}}
\def \bex {\begin{minipage}{0.3\textwidth}\includegraphics[width=1in]{jugelizi.png}\end{minipage}\begin{minipage}{0.6\textwidth}}
\def \eex {\end{minipage}}
\def \chtitle#1 {\frametitle{\bch #1 \ech}}
\def \skipline { {\vskip 0.1in}}
\def \langr {\mathcal{L}}
\def \hamil {\mathcal{H}}
\def \vecx {\mathbf{x}}
\def \veck {\mathbf{k}}
\def \vecp {\mathbf{p}}
\def \hatphi {\hat{\phi}}
\def \hatq {\hat{q}}
\def \hatpi  {\hat{\pi}}
\def \vel {\upsilon}
\def \Dint {\mathcal{D}}
\def \adag {{\hat{a}^\dagger}}
\def \hata {\hat{a}}
\def \hatN {\hat{N}}
\def \hatH {\hat{H}}
\def \nket { {| n \rangle}}
\def \bran { {\langle n |}}

\title{Quantum Field Theory I \\ Lesson 02 - Action Princple}
  \author{}
  \date{}


\begin{document}

\begin{frame}
 
\begin{center}
\begin{Large}
\bch
量子场论 I 

{\vskip 0.3in}

第一次课后作业参考答案
\skipline
\skipline

如发现参考答案有错误请不吝告知(微信zhiqihuang或邮箱huangzhq25@sysu.edu.cn)
\ech
\end{Large}
\end{center}

\vskip 0.2in

\bch
课件下载
\ech
https://github.com/zqhuang/SYSU\_QFTI

\end{frame}

\begin{frame}
\chtitle{第1题:题目和思路}
\bch
题目:若算得的截面为$\sigma = 10^{-3}/m_W^2$, $m_W \approx 80 GeV$是$W^{\pm}$的粒子质量,试换算出以$\mathrm{cm}^2$为单位的截面值。若算得的寿命的$\tau = 100/m_w$,试问等于多少秒?

\skipline
思路:这道题要求把自然单位制表示的物理量用普通单位制表示出来。首先要理解$\sigma = 10^{-3}/m_W^2$这种自然单位制写法的含义:它表示在普通单位制里选取适当的$p$,$q$可以使$\sigma = 10^{-3}/m_W^2 c^p\hbar^q$,在自然单位制下规定$c=\hbar=1$所以可以不写$c^p\hbar^q$。我们只要求出$p$和$q$就能知道$\sigma$在普通单位制下的表示。

\ech
\end{frame}

\begin{frame}
\chtitle{第1题:第一种解法}
\bch
设在普通单位制下$\sigma = 10^{-3}/m_W^2 c^p \hbar^q$。我们用$M$, $L$, $T$分别表示普通单位制下的质量,长度,时间量纲。则无论是按”截面”的字面意思,还是按题目要求(普通单位制下用$\mathrm{cm}^2$表示),都能确定左边是一个面积,量纲为$L^2$;右边的量纲为$M^{-2} (LT^{-1})^p(ML^2T^{-1})^q = M^{q-2}L^{p+2q}T^{-p-q}$。很显然要两边在普通单位制下量纲一致必须有$p=-2$, $q=2$。 所以
$$\sigma = 10^{-3}\hbar^2 /m_W^2/c^2$$
注意题目给的$m_W$以能量单位$\mathrm{GeV}=1.6022\times 10^{-10}J$表示,实际上是普通单位制里的$m_Wc^2$。所以我们把上式写成:
\begin{eqnarray}
\sigma &=& 10^{-3}\hbar^2 c^2/(m_Wc^2)^2 \newl
       &=& 10^{-3}\times \frac{(1.0546 \times 10^{-34} \mathrm{J\cdot s})^2(2.9979\times10^{10}\SIcm\,\SIs^{-1})^2}{(80\times 1.6022 \times 10^{-10} \mathrm{J})^2} \newl
       &=& 6.08\times 10^{-35} \mathrm{cm}^2 \nonumber
\end{eqnarray}
\ech
\end{frame}

\begin{frame}
\chtitle{第1题:第一种解法(续)}
\bch
同样,可以假设$\tau = 100/m_W c^p \hbar^q$。左边的量纲为$T$,右边的量纲为$M^{q-1}L^{p+2q}T^{-p-q}$。于是$p=-2$, $q=1$。
\begin{eqnarray}
\tau &=& 100 \hbar/(m_Wc^2) \newl
     &=& 100 \times (1.0546\times 10^{-34} \mathrm{J\cdot s}) /(80\times 1.6022 \times 10^{-10} \mathrm{J}) \newl
     &=& 8.23 \times 10^{-25}\SIs \nonumber
\end{eqnarray}



\ech
\end{frame}

\begin{frame}
\chtitle{第1题:第二种解法}
\bch
在自然单位制下可以任意地乘或除$c$和$\hbar$。所以我们总结出:
\begin{itemize}
\item{显然时间和长度可以用乘或除$c$来互相转化。}
\item{根据$E=mc^2$,能量和质量可以通过乘或除$c^2$互相转化。}
\item{根据量子力学的能量测不准原理$\Delta E\Delta t \gtrsim \hbar$, 时间和能量可以用$\hbar$去除相互转化。}
\item{根据量子力学的动量测不准原理$\Delta (mv) \Delta x \gtrsim \hbar$, $v$和$c$量纲相同,所以质量和长度可以用$\hbar/c$去除互相转化。}
\end{itemize}

\ech
\end{frame}

\begin{frame}
\chtitle{第1题:第二种解法(续)}
\bch
我们先把$\SIGeV$(能量)换算成时间:
$$ 1 \SIGeV = \frac{1.6022\times 10^{-10}J}{1.0546\times 10^{-34}\SIJ\cdot\SIs} = \frac{1}{6.582\times 10^{-25}\SIs}$$
然后把时间转化为长度。
$$ 6.582\times 10^{-25}\SIs =  (6.582\times 10^{-25}\SIs)  (2.9979\times 10^{10} \SIcm/\SIs) = 1.973\times 10^{-14} \SIcm $$
所以第一问答案为:
$$\sigma = \frac{10^{-3}}{80^2} \SIGeV^{-2} = \frac{10^{-3}}{80^2}\times  (1.973\times 10^{-14} \SIcm)^2 = 6.08\times 10^{-35}\SIcm^2$$
第二问答案为:
$$\tau = 100/80 \SIGeV^{-1} = 1.25 \times (6.582\times 10^{-25}\SIs) = 8.23\times 10^{-25}\SIs$$
\ech
\end{frame}

\begin{frame}
\chtitle{第2题: 题目和思路}
\bch
题目:证明任意四维时空坐标系$(x^0, x^1, x^2, x^3)$的积分元$\sqrt{-g}d^4x$是一个标量。其中$g$是度规矩阵$g_{\mu\nu}$的行列式的简写,$d^4x$是积分元$dx^1dx^2dx^3dx^4$的简写。
\skipline

思路:$g_{\mu\nu}$矩阵的变换规则按张量定义可以写出来。$d^4x$的变化在微积分课程里学过:多元函数积分进行变量替换时,积分元要乘Jacobian行列式的绝对值的倒数。

\ech
\end{frame}

\begin{frame}
\chtitle{第2题解答}
\bch
假设坐标变换为$x \rightarrow \tilde{x}$,则Jacobian矩阵为$J^\mu_{\,\nu} = \frac{\partial \tilde{x}^\mu}{\partial x^\nu}$, 其行列式简写为$\det{J}$。按多元函数积分的变量替换规则,
\begin{equation}
 d^4x = |\det{J}|^{-1} d^4\tilde{x} \label{eq:dd}
\end{equation}
张量$g_{\mu\nu}$的变换规则为$ g_{\mu\nu} =  \frac{\partial \tilde{x}^\alpha}{\partial x^\mu} \frac{\partial \tilde{x}^\beta}{\partial x^\nu} \tilde{g}_{\alpha\beta} $

把$g_{\mu\nu}$矩阵简写为$G$,上式可写成矩阵乘法$ G = J^T \tilde{G} J$

两边取行列式,并根据“积的行列式等于行列式的积”,以及“矩阵转置不改变行列式”,得到:
\begin{equation}
\det{G} = |\det{J}|^2 \det{\tilde{G}} \label{eq:gg}
\end{equation}
结合\eqref{eq:dd}和\eqref{eq:gg}即得所求证的结论$\sqrt{-g}d^4x = \sqrt{-\tilde{g}}d^4\tilde{x}$。

\ech
\end{frame}



\begin{frame}
\chtitle{第3题题目和思路}
\bch
题目: 考虑一维空间x和一维时间t构成的时空里的标量场$\phi(x,t)$,若其作用量为
$$ S = \int dx dt \   \frac{1}{2} \left[\left(\frac{\partial \phi}{\partial t}\right)^2 - \left(\frac{\partial \phi}{\partial x}\right)^2 - V(\phi)\right]\, ,$$
其中$V(\phi)$为给定的势能函数。试用求作用量稳定点的方法推导$\phi(x, t)$的运动方程并将结果与Euler-Lagrange方程做比较。

\skipline
思路:仿照课上的离散化方法取作用量的稳定点。
\ech
\end{frame}

\begin{frame}
\chtitle{第3题解答}
\bch
把时间和空间均进行离散化。固定格点的间隔$dx$和$dt$。该物理系统的所有自由度为$\phi(x+idx, t+j dt)$ ($i, j = 0, \pm 1, \pm 2\ldots$)。
作用量可以写成离散求和:
\begin{eqnarray}
\frac{S}{dx\,dt} &=& \frac{1}{2}\left(\frac{\phi(x, t+dt) - \phi(x,t)}{dt}\right)^2 + \frac{1}{2}\left(\frac{\phi(x, t) - \phi(x,t-dt)}{dt}\right)^2 \newl
 &-&\frac{1}{2}\left(\frac{\phi(x+dx, t) - \phi(x,t)}{dx}\right)^2 - \frac{1}{2}\left(\frac{\phi(x, t) - \phi(x-dx,t)}{dx}\right)^2 \newl
 &-&\frac{1}{2}V(\phi(x, t)) \newl
 &+&\ldots \nonumber
\end{eqnarray}
其中$\ldots$代表所有跟$\phi(x, t)$这个自由度无关的项。
\ech
\end{frame}

\begin{frame}
\chtitle{第3题解答(续)}
\bch
作用量对$\phi(x, t)$求偏导并令其为零,得到
\begin{eqnarray}
0 &=& \frac{1}{dt} \left [ -\frac{\phi(x, t+dt) - \phi(x,t)}{dt} + \frac{\phi(x, t) - \phi(x,t-dt)}{dt}\right]  \newl
  && -\frac{1}{dx}\left [- \frac{\phi(x+dx, t) - \phi(x,t)}{dx} + \frac{\phi(x, t) - \phi(x-dx,t)}{dx}\right] \newl
  && -\frac{1}{2}V'\left(\phi(x, t)\right) \newl 
  &=& \frac{1}{dt}\left[-\partial_t\phi(x, t+dt/2) + \partial_t\phi(x, t-dt/2)\right] \newl
   &&- \frac{1}{dx}\left[-\partial_x\phi(x+dx/2, t) + \partial_x\phi(x-dx/2, t)\right] \newl
   &&-  \frac{1}{2} V'\left(\phi(x, t)\right)  \newl
  &=& (- \partial_t^2 + \partial_x^2 )\phi -\frac{1}{2} V'(\phi) \nonumber 
\end{eqnarray}
\ech
\end{frame}

\begin{frame}
\chtitle{第3题解答(续)}
\bch
最终得到运动方程:$(\partial_t^2 - \partial_x^2 )\phi + \frac{1}{2} V'(\phi) = 0 \,.$,和Euler-Lagrange方程一致。
\skipline

注意:一般习惯上写拉氏密度时把$V$单独拿出来不乘$1/2$因子。
$$\langr = \frac{1}{2} \left[\left(\frac{\partial \phi}{\partial t}\right)^2 - \left(\frac{\partial \phi}{\partial x}\right)^2\right] - V(\phi) $$
按这种定义推导出来的运动方程就是$(\partial_t^2 - \partial_x^2 )\phi + V'(\phi) = 0 \,.$

题目采取了比较奇怪的定义(逻辑上完全没有问题,但会对以后阅读文献有不好影响),是因为我出题不够仔细。对此非常抱歉。
\ech
\end{frame}


\begin{frame}
\chtitle{第4题题目和思路}
\bch
题目:谐振子
$$ S = \int_{-\infty}^\infty \frac{m}{2}\left[(\frac{d\phi}{dt})^2 - \omega^2\phi^2\right] dt$$
把时间维特殊化以后就只有一个自由度$\phi$。写出$\phi$对应的正则动量$\pi$,系统的Hamilton量,以及Hamilton方程。试把$\pi$从两个Hamilton方程中消去,得到的结果和Euler-Lagrange方程一致吗?

\skipline
思路:世上再也没有比套公式更令人愉快的事了。
\ech
\end{frame}

\begin{frame}
\chtitle{第4题解答}
\bch
$$\pi = \frac{\partial L}{\partial (\dot\phi)} = m\dot\phi \, $$
$$ H = \dot\phi \pi - L = \frac{\pi^2}{2m} +  \frac{1}{2}m\omega^2\phi^2 \, $$
Hamilton方程:
$$\dot\pi = -\frac{\partial H}{\partial\phi} = - m\omega^2\phi$$  
$$\dot\phi = \frac{\partial H}{\partial \pi} = \frac{\pi}{m} $$
消去$\pi$后得到,
$$\ddot\phi + \omega^2\phi = 0$$
和Euler-Lagrange方程相同。
\ech
\end{frame}

\begin{frame}
\chtitle{第5题题目和思路}
\bch
作用量
$$S = \int d^4x\ \langr(\phi, \partial_\mu\phi)\,.$$
在坐标平移下显然具有不变性。由此用Noether定理推导场$\phi$的能量动量守恒方程。

\skipline

思路:能量动量守恒是由时间平移和空间平移对称性导致的。所以我们考虑四种无穷小变换$x^\mu \rightarrow x^\mu + \epsilon^r \delta_r^\mu$ ($r = 0, 1, 2, 3$)。
\ech
\end{frame}


\begin{frame}
\chtitle{第5题解答}
\bch
考虑无穷小变换(即坐标系不动,时空以及时空里的场一起相对于坐标系平移)$x^\mu \rightarrow x^\mu + \epsilon^r \delta_r^\mu$(这里不对$r$求和,分别考虑固定$r=0,1,2,3$时的守恒流)。

移动后在坐标$x^\mu$处的$\phi$实际上是移动前在$x^\mu - \epsilon^r \delta_r^\mu$处的$\phi$。所以$\phi$的变化为:
$$\delta\phi = - \epsilon^r \delta_r^\mu \partial_\mu\phi$$
即
$$\frac{\delta\phi}{\delta\epsilon^r} = \delta_r^\mu \partial_\mu \phi = \partial_r\phi$$

\skipline

同理
$$\delta\langr = - \epsilon^r \delta_r^\mu \partial_\mu\langr$$
对固定的$r$而言$ \delta_r^\mu$是不依赖于坐标$x^\mu$的常数,所以上式可写成
$$\delta\langr =\epsilon^r \partial_\mu (-  \delta_r^\mu \langr)$$

\ech
\end{frame}

\begin{frame}
\chtitle{第5题解答(续)}
\bch
根据Noether定理,有
$$j^\mu_r = - \delta_r^\mu \langr +  \partial_r \phi\frac{\partial \langr}{\partial (\partial_\mu \phi)}$$
为守恒流。

写成更对称的“能量动量张量”形式:
$$T^{\mu\nu} = \frac{\partial \langr}{\partial (\partial_\mu \phi)}\partial^\nu\phi- \langr g^{\mu\nu}$$
满足
$$\partial_\mu T^{\mu\nu} = 0$$
积分$\int T^{00} d^3\vecx$, $\int T^{0i}d^3\vecx$ ($i=1,2,3$)分别对应总能量和三个方向的动量。它们都是守恒的。
\ech
\end{frame}

\end{document}
