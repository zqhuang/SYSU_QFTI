\documentclass[CJK]{beamer}
\usepackage{CJKutf8}
\usepackage{beamerthemesplit}
\usetheme{Malmoe}
\useoutertheme[footline=authortitle]{miniframes}
\usepackage{amsmath}
\usepackage{amssymb}
\usepackage{graphicx}
\usepackage{color}
\graphicspath{{figures/}}
\def \bch {\begin{CJK}{UTF8}{gbsn}}
\def \ech {\end{CJK}}
\def \bex {\begin{minipage}{0.3\textwidth}\includegraphics[width=1in]{jugelizi.png}\end{minipage}\begin{minipage}{0.6\textwidth}}
\def \eex {\end{minipage}}
\def \chtitle#1 {\frametitle{\bch #1 \ech}}
\def \skipline { {\vskip 0.1in}}
\def \langr {\mathcal{L}}
\def \hamil {\mathcal{H}}
\def \vecx {\mathbf{x}}
\def \veck {\mathbf{k}}
\def \vecp {\mathbf{p}}
\def \hatphi {\hat{\phi}}
\def \hatq {\hat{q}}
\def \hatpi  {\hat{\pi}}
\def \vel {\upsilon}
\def \Dint {\mathcal{D}}
\def \adag {{\hat{a}^\dagger}}
\def \hata {\hat{a}}
\def \hatN {\hat{N}}
\def \hatH {\hat{H}}
\def \nket { {| n \rangle}}
\def \bran { {\langle n |}}

\title{Quantum Field Theory I \\ Homework 2 solution}
  \author{}
  \date{}


\begin{document}

\begin{frame}
 
\begin{center}
\begin{Large}
\bch
量子场论 I 

{\vskip 0.3in}

第二次课后作业参考答案
\skipline
\skipline

如发现参考答案有错误请不吝告知(微信zhiqihuang或邮箱huangzhq25@sysu.edu.cn)
\ech
\end{Large}
\end{center}

\vskip 0.2in

\bch
课件下载
\ech
https://github.com/zqhuang/SYSU\_QFTI

\end{frame}

\begin{frame}
\chtitle{第1题:题目和思路}
\bch
题目:对质量为$m$的自由实标量场$\phi$,四维动量$k$满足$k^0 = \omega \equiv \sqrt{\veck^2+m^2}$,其中$\veck\equiv (k^1, k^2, k^3)$为三维动量。对任意洛仑兹变换下不变的函数$f(k)$,证明积分$\int \frac{d^3\veck}{2\omega} f(k)$ 也是洛仑兹变换下的不变量。

\skipline
思路:对任意标量函数$h(k)$,积分$\int d^4k\, h(k)$也是洛仑兹变换下的标量,所以问题就简化为:能不能找到洛仑兹变换下的标量函数$h(k)$,使得 $\int h(k)d k^0 = \frac{f(k)}{2\omega}$。
\ech
\end{frame}

\begin{frame}
\chtitle{第1题解答}
\bch
取$h(k) = \frac{1}{2} \delta(k^2 - m^2) f(k)$,

首先,因为$k^2 \equiv k^\mu k_\mu$是洛仑兹不变量,所以它的函数$\delta(k^2 - m^2)$也是洛仑兹不变量,乘上另一个洛仑兹不变量$f(k)$之后得到$h(k)$也是洛仑兹不变量。

其次, 利用$\delta$函数性质(每个零点附近作变量替换$y=f(x)$可证)
 $$\int \delta\left(f(x)\right) dx= \sum_{x*:\,f(x*) = 0}\frac{1}{|f'(x*)|}$$
可以得到
$$\int h(k) dk^0 = \frac{1}{4\omega}(f(\omega, \veck) + f(-\omega, \veck))$$
因为$f(k)$是任意洛仑兹变换下的不变量,故在时间反演下也是不变量,两项贡献相等。
从而有$\int h(k) dk^0 = \frac{1}{2\omega}f(\omega, \veck)$

综上我们证明了$\int \frac{d^3\veck}{2\omega} f(k) = \int h(k) d^4k$ 是洛仑兹不变量
\ech
\end{frame}



\begin{frame}
\chtitle{第2题: 题目和思路}
\bch
题目:对质量为$m$的自由实标量场$\phi$,证明四维时空下的任意两点场算符的对易$[\hatphi(x),\hatphi(x')]$是洛仑兹变换下的不变量。(提示:利用$\hatphi$的算符表达式和第1题结论。)

\skipline
思路:利用课上得到的$\hatphi$表达式硬算就行。
\ech
\end{frame}

\begin{frame}
\chtitle{第2题解答}
\bch
利用课上得到的
$$\hatphi(x) = \frac{1}{(2\pi)^{3/2}} \int \sqrt{\frac{d^3\veck}{2\omega}} \left(\hata_{\veck} e^{-ik_\mu x^\mu} + \adag_{\veck}e^{ik_\mu x^\mu}\right) $$
得到
{\scriptsize
\begin{eqnarray}
&& \left[\hatphi(x),\hatphi(x')\right] \newl
&=& \frac{1}{(2\pi)^3}\sum_{\veck,\veck'} \frac{\sqrt{d^3\veck\, d^3\veck'}}{2\sqrt{\omega \omega'}}\left([\hata_\veck, \adag_{\veck'}]e^{-\ii (k_\mu x^\mu-k'_\mu x'^\mu)} + [\adag_\veck, \hata_{\veck'}]e^{\ii (k_\mu x^\mu-k'_\mu x'^\mu) }\right) \newl
&=& \frac{1}{(2\pi)^3} \sum_{\veck} \frac{d^3\veck}{2\omega}\left(e^{-\ii k_\mu (x^\mu - x'^\mu)} - e^{\ii k_\mu (x^\mu - x'^\mu)}\right) \nonumber
\end{eqnarray}
}
把求和号换成积分号,再根据上题结论即得证。
\ech
\end{frame}

\begin{frame}
\chtitle{第3题题目和思路}
\bch
题目:
{\scriptsize 
质量为$m$的自由复标量场,
$$\lagr = \partial_\mu\phi^\dagger \partial^\mu \phi - m^2\phi^\dagger\phi\, $$
把$\phi$和$\phi^\dagger$分别看作独立自由度,它们对应的正则动量分别为
$$\pi = \frac{\partial \lagr}{\partial \dot\phi} = \dot\phi^\dagger, \,\pi^\dagger = \frac{\partial \lagr}{\partial \dot\phi^\dagger} = \dot\phi $$
于是Hamilton密度为
$$\hamil = \dot\phi^\dagger \pi^\dagger +\dot\phi \pi - \lagr = \pi^\dagger \pi + \nabla\phi^\dagger \cdot\nabla\phi + m^2\phi^\dagger \phi $$
我们在课堂上推到了量子化后的$\phi$为
$$\hatphi(x) = \frac{1}{(2\pi)^{3/2}} \int \sqrt{\frac{d^3\veck}{2\omega}} \left(\hata_{\veck} e^{-ik_\mu x^\mu} + \bdag_{\veck}e^{ik_\mu x^\mu}\right) $$
试求量子化后的总Hamilton量$\hat{H}$.
}

\skipline
思路:题目提示的思路是直接代入并利用产生湮灭算符的对易性质,这样计算实际上比较繁复。也可以考虑把$H$写成傅立叶空间的积分来简化计算。
\ech
\end{frame}

\begin{frame}
\chtitle{第3题第一种解答}
\bch
我们先考虑固定时间$t=0$的$\hat{H}$,把$\hat{H}$写成傅立叶空间的积分
$$\hat{H} = \int d^3\veck\, \frac{1}{2}\left(\hatpi^\dagger\hatpi + \omega^2\hatphi^\dagger\hatphi\right)$$ 
在傅立叶空间
$$\hatphi = \frac{\hata_{\veck}+\bdag_{-\veck}}{\sqrt{2\omega\,d^3\veck}}$$
$$\hatphi^\dagger =  \frac{\adag_{\veck}-\hatb_{-\veck}}{\sqrt{2\omega\,d^3\veck}}$$
$$\hatpi^\dagger = \dot\hatphi = -\ii\omega \frac{\hata_{\veck}-\bdag_{-\veck}}{\sqrt{2\omega\,d^3\veck}}$$
$$\hatpi = \dot\hatphi^\dagger = \ii \omega\frac{\adag_{\veck}+\hatb_{-\veck}}{\sqrt{2\omega\,d^3\veck}} $$

\ech
\end{frame}


\begin{frame}
\chtitle{第3题第一种解答(续)}
\bch
直接代入得到
$$\hat{H} = \sum_{\veck} \frac{\omega}{2}\left(\adag_{\veck}\hata_{\veck} + \hata_{\veck}\adag_{\veck} + \bdag_{\veck}\hatb_{\veck} + \hatb_{\veck}\bdag_{\veck}\right)$$ 
利用产生湮灭算符对易关系上式也可写成
$$\hat{H} = \sum_{\veck} \omega(\adag_{\veck}\hata_{\veck} + \bdag_{\veck}\hatb_{\veck} + 1)$$
\ech
\end{frame}



\begin{frame}
\chtitle{第3题第二种解答}
\bch
先求出
$$\hatpi^\dagger = \dot\hatphi = \frac{1}{(2\pi)^{3/2}} \int \sqrt{\frac{d^3\veck}{2\omega}} \ii \omega \left(-\hata_{\veck} e^{-ik_\mu x^\mu} + \bdag_{\veck}e^{ik_\mu x^\mu}\right) $$
$$\hatpi = \dot\hatphi^\dagger = \frac{1}{(2\pi)^{3/2}} \int \sqrt{\frac{d^3\veck}{2\omega}} \ii \omega \left(\adag_{\veck} e^{ik_\mu x^\mu} - \hatb_{\veck}e^{-ik_\mu x^\mu}\right) $$
$$\nabla\hatphi = \frac{1}{(2\pi)^{3/2}} \int \sqrt{\frac{d^3\veck}{2\omega}} \ii\veck \left(\hata_{\veck} e^{-ik_\mu x^\mu} - \bdag_{\veck}e^{ik_\mu x^\mu}\right) $$
$$\nabla\hatphi^\dagger = \frac{1}{(2\pi)^{3/2}} \int \sqrt{\frac{d^3\veck}{2\omega}} \ii \veck \left(-\adag_{\veck} e^{ik_\mu x^\mu} + \hatb_{\veck}e^{-ik_\mu x^\mu}\right) $$
\ech
\end{frame}

\begin{frame}
\chtitle{第3题第二种解答(续)}
\bch
把上述表达式代入
$$\hat{H}(t) = \int d^3\vecx\, \hat{\hamil}(t, \vecx)$$
并利用
$$\frac{1}{(2\pi)^3}\int d^3\vecx\, e^{-i(\veck-\veck')\cdot\vecx} = \delta(\veck - \veck')$$ 
化简得到
$$\hat{H} = \sum_{\veck} \omega(\adag_{\veck}\hata_{\veck} + \bdag_{\veck}\hatb_{\veck} + 1)$$
\ech
\end{frame}



\begin{frame}
\chtitle{第4题题目和思路}
\bch
题目:考虑和规范场耦合的复标量场的拉氏密度:
$$\lagr = (D^\mu\phi)^\dagger D_\mu \phi - m^2\phi^\dagger \phi  - \frac{1}{4}\Fup\Fdown$$
利用Euler-Lagrange方程
$$\partial_\mu \Fup = j^\nu$$
证明$j_\mu$是守恒流.

\skipline
思路:太简单不需要思路。
\ech
\end{frame}

\begin{frame}
\chtitle{第4题解答}
\bch
$$\partial_\nu j^\nu = \partial_\nu \partial_\mu \Fup$$
$$\partial_\mu j^\mu = \partial_\mu \partial_\nu F^{\nu\mu}$$
两式相加并利用$F$反对称即得证$\partial_\mu j^\mu = 0$。
\ech
\end{frame}

\begin{frame}
\chtitle{第5题题目和思路}
\bch
题目:
{\small 关于三维空间转动算符
\begin{itemize}
\item{证明相反方向的角动量算符相差一个负号$\hat{J}_{-\vecn} = - \hatJn$}
\item{因为绕固定轴转动$2\pi$的没有任何效果,所以转动算符$e^{-2\pi\ii\hatJn}=1$,由此证明$\hatJn$的本征值一定是整数。}
\item{三个形成右手正交系的方向$\vecn_1$, $\vecn_2$, $\vecn_3$,对应的转动算符分别为$\hatj{1} $, $\hatj{2} $, $\hatj{3}$,证明$\hatj{1}^2+\hatj{2}^2+\hatj{3}^2 = 2$。.}
\end{itemize}
}

\skipline

思路:直接利用课上介绍的$\hatJn$性质。
\ech
\end{frame}


\begin{frame}
\chtitle{第5题解答}
\bch
\begin{itemize}
\item{显然沿$\vecn$转$\theta$和沿$-\vecn$转$-\theta$等效,所以 $ e^{\ii \theta \hat{J}_{-\vecn}}= e^{-\ii\theta \hatJn}  $。两边对$\theta$求导并令$\theta = 0$即得$\hat{J}_{-\vecn} = - \hatJn$}
\item{根据算符的解析函数的本征值等于算符的本征值取该解析函数,若$\hatJn$有本征值$\lambda$,则$e^{-2\pi\ii\hatJn}$有本征值$e^{-2\pi\ii\lambda}$。又单位矩阵只有本征值1,所以$e^{-2\pi\ii\lambda}=1$。即$\lambda$必须是整数。}
\item{利用$\hatj{i}\vecn_j = \ii\epsilon_{ijk}\vecn_k$可以得到
$(\sum_i\hat{J}_i^2)\vecn_{j} = \ii \sum_{ik} \epsilon_{ijk} \hat{J}_i \vecn_k = -\sum_{ikl} \epsilon_{ijk}\epsilon_{ikl}\vecn_l =\sum_{ikl} \epsilon_{ikj}\epsilon_{ikl}\vecn_l = \sum_{ik} \epsilon_{ikj}\epsilon_{ikj}\vecn_j =2\vecn_j$。因为任何三维矢量均可写成$\vecn_1,\vecn_2,\vecn_3$的线性组合,所以对任何三维矢量$\vecv$均有$(\sum_i\hat{J}_i^2)\vecv = 2\vecv$。从而证明了算符等式:$\sum_i\hat{J}_i^2 =2$。
}
\end{itemize}
\ech
\end{frame}

\end{document}
