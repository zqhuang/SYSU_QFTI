\documentclass[CJK]{beamer}
\usepackage{CJKutf8}
\usepackage{beamerthemesplit}
\usetheme{Malmoe}
\useoutertheme[footline=authortitle]{miniframes}
\usepackage{amsmath}
\usepackage{amssymb}
\usepackage{graphicx}
\usepackage{color}
\graphicspath{{figures/}}
\def \bch {\begin{CJK}{UTF8}{gbsn}}
\def \ech {\end{CJK}}
\def \bex {\begin{minipage}{0.3\textwidth}\includegraphics[width=1in]{jugelizi.png}\end{minipage}\begin{minipage}{0.6\textwidth}}
\def \eex {\end{minipage}}
\def \chtitle#1 {\frametitle{\bch #1 \ech}}
\def \skipline { {\vskip 0.1in}}
\def \langr {\mathcal{L}}
\def \hamil {\mathcal{H}}
\def \vecx {\mathbf{x}}
\def \veck {\mathbf{k}}
\def \vecp {\mathbf{p}}
\def \hatphi {\hat{\phi}}
\def \hatq {\hat{q}}
\def \hatpi  {\hat{\pi}}
\def \vel {\upsilon}
\def \Dint {\mathcal{D}}
\def \adag {{\hat{a}^\dagger}}
\def \hata {\hat{a}}
\def \hatN {\hat{N}}
\def \hatH {\hat{H}}
\def \nket { {| n \rangle}}
\def \bran { {\langle n |}}

\title{Quantum Field Theory I \\ Homework 6}
  \author{}
  \date{}


\begin{document}

\begin{frame}
 
\begin{center}
\begin{Large}
\bch
量子场论 I 

{\vskip 0.3in}

第六次课后作业 (共八次,每次2.5分)

交作业时间: 12月5日,星期一,13:30pm

\ech
\end{Large}
\end{center}

\vskip 0.2in

\bch
课件下载
\ech
https://github.com/zqhuang/SYSU\_QFTI

\end{frame}

\begin{frame}
\chtitle{第1题(0.5分)}
\bch
证明下述矩阵的迹为零:
\begin{itemize}
\item{$\slashed{a}\slashed{b}\gamma^0$}
\item{$\slashed{a}\slashed{b}\gamma^5$}
\item{$\slashed{a}\slashed{b}\slashed{c}$}
\item{$\slashed{a}\slashed{b}\slashed{c}\gamma^5$}
\item{$\slashed{a}\slashed{b}\slashed{c}\slashed{d}-\slashed{d}\slashed{c}\slashed{b}\slashed{a}$}
\end{itemize}
\ech
\end{frame}

\begin{frame}
\chtitle{第2题(0.5分)}
\bch
证明下述恒等式:
\begin{itemize}
\item{$\slashed{a}\slashed{b} + \slashed{b}\slashed{a} = 2 ab$}
\item{$\trof{\slashed{a}\gamma^\mu} = 4 a^\mu$}
\end{itemize}
\ech
\end{frame}

\begin{frame}
\chtitle{第3题(0.5分)}
\bch 
把
$$\trof{\slashed{a}\gamma^\mu\slashed{b}\slashed{c}\slashed{d}\gamma_\mu}$$
化简为只含矢量内积的最简形式。

\ech
\end{frame}



\begin{frame}
\chtitle{第4题(0.5分)}
\bch
设$p$为电子的四维动量,$k$为光子的四维动量,化简
$$\trof{ (\slashed{p}+m)\gamma^\mu\frac{1}{\slashed{p}+\slashed{k}-m}\gamma_\mu}$$
\ech
\end{frame}


\begin{frame}
\chtitle{第5题(0.5分)}
\bch
在Compton散射过程中,取电子初始状态为静止,入射光子沿$z$轴方向的参照系。限定入射光子四维动量为$k^\mu=(\omega, 0, 0, \omega)$;出射光子能量为$\omega'$,方向限定在某固定方向$\vecn$附近的立体角$d\Omega$内;$\vecn$与$z$轴的夹角为$\theta$。当$d\Omega$很小时,散射截面与$d\Omega$成正比,它们之间的比称为微分散射截面:$\frac{d\sigma}{d\Omega}$。试根据课上所求的散射概率计算该微分散射截面(写成$\omega$, $\omega'$,和$\theta$的函数)。
\ech
\end{frame}

\end{document}
