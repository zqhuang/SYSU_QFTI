\documentclass[CJK]{beamer}
\usepackage{CJKutf8}
\usepackage{beamerthemesplit}
\usetheme{Malmoe}
\useoutertheme[footline=authortitle]{miniframes}
\usepackage{amsmath}
\usepackage{amssymb}
\usepackage{graphicx}
\usepackage{color}
\graphicspath{{figures/}}
\def \bch {\begin{CJK}{UTF8}{gbsn}}
\def \ech {\end{CJK}}
\def \bex {\begin{minipage}{0.3\textwidth}\includegraphics[width=1in]{jugelizi.png}\end{minipage}\begin{minipage}{0.6\textwidth}}
\def \eex {\end{minipage}}
\def \chtitle#1 {\frametitle{\bch #1 \ech}}
\def \skipline { {\vskip 0.1in}}
\def \langr {\mathcal{L}}
\def \hamil {\mathcal{H}}
\def \vecx {\mathbf{x}}
\def \veck {\mathbf{k}}
\def \vecp {\mathbf{p}}
\def \hatphi {\hat{\phi}}
\def \hatq {\hat{q}}
\def \hatpi  {\hat{\pi}}
\def \vel {\upsilon}
\def \Dint {\mathcal{D}}
\def \adag {{\hat{a}^\dagger}}
\def \hata {\hat{a}}
\def \hatN {\hat{N}}
\def \hatH {\hat{H}}
\def \nket { {| n \rangle}}
\def \bran { {\langle n |}}

\title{Quantum Field Theory I \\ Homework 5}
  \author{}
  \date{}


\begin{document}

\begin{frame}
 
\begin{center}
\begin{Large}
\bch
量子场论 I 

{\vskip 0.3in}

第五次课后作业 (共八次,每次2.5分)

交作业时间: 11月21日,星期一,13:30pm

\ech
\end{Large}
\end{center}

\vskip 0.2in

\bch
课件下载
\ech
https://github.com/zqhuang/SYSU\_QFTI

\end{frame}

\begin{frame}
\chtitle{第1题(0.5分)}
\bch
考虑一个有自相互作用的复标量场
$$ \lagr = \partial^\mu\phi^\dagger \partial_\mu \phi - m^2\phi^\dagger\phi - \frac{\lambda}{4}\left(\phi^\dagger\phi\right)^2$$
这里$0<\lambda\ll 1$。我们已经知道复标量场有$a$, $b$两种粒子互为反粒子。两个动量为$p_1$, $p_2$的$a$粒子发生散射变为动量为$p_3$, $p_4$的粒子。仅考虑最低阶近似。证明末态粒子只能都是$a$粒子,并求出散射振幅。

\skipline
{\small
注:$p_1$, $p_2$, $p_3$, $p_4$均为四维动量。}
\ech
\end{frame}

\begin{frame}
\chtitle{第2题(0.5分)}
\bch
仍然考虑上题中的有自相互作用的复标量场
$$ \lagr = \partial^\mu\phi^\dagger \partial_\mu \phi - m^2\phi^\dagger\phi - \frac{\lambda}{4}\left(\phi^\dagger\phi\right)^2$$
试在质心参考系求两个动量为$p_1$, $p_2$的$a$粒子发生散射的散射截面,仍只要求算到最低阶近似。

\skipline
{\small
注:$p_1$, $p_2$均为四维动量。}

\ech
\end{frame}

\begin{frame}
\chtitle{第3题(0.5分)}
\bch{\small 
对自由实标量场$\phi$,定义$n$点Green函数为:
$$G(x_1, x_2,\ldots,x_n) \equiv \langle 0| \torder{\phi(x_1)\phi(x_2)\dots\phi(x_n)}|0\rangle$$
其中$|0\rangle$为真空态。
\skipline

证明:当$n$为奇数时$G(x_1, x_2,\ldots,x_n)=0$,当$n$为偶数时$G(x_1, x_2,\ldots,x_n)$等于把$x_1, x_2,\ldots,x_n$进行两两配对所得两点Green函数的乘积的遍历配对方式之和,例如$n=4$时,
$$G(x_1, x_2, x_3, x_4) = G(x_1, x_2)G(x_3, x_4) + G(x_1, x_3)G(x_2, x_4) + G(x_1, x_4) G(x_2, x_3)$$
}

\skiplines
{\scriptsize 提示:用Wick定理或者路径积分基本定理}
\ech
\end{frame}



\begin{frame}
\chtitle{第4题(0.5分)}
\bch
仍然考虑自由实标量场$\phi$,考虑固定时刻的傅立叶空间$\phi(\veck)$,定义$n$点功率谱为
$$P(\veck_1,\veck_2,\ldots,\veck_n) \equiv \langle 0 | \hatphi(\veck_1)\hatphi(\veck_2)\ldots\hatphi(\veck_n)|0\rangle$$
证明:若总波矢$\sum_{i=1}^n\veck_i$不为零,则$P(\veck_1,\veck_2,\ldots,\veck_n) = 0$。
\ech
\end{frame}


\begin{frame}
\chtitle{第5题(0.5分)}
\bch
{\small
现代宇宙学标准模型认为宇宙中一切密度扰动均源于宇宙早期暴涨时的实标量场$\phi$的真空量子波动。我们在傅立叶空间考虑任意时刻的密度扰动$\rho(\veck, t)$和初始时刻的场$\phi(\veck, t_0)$。它们之间可以用一个线性增长因子$T(\veck,t)$联系起来
$$\rho (\veck, t) = T(\veck, t) \phi(\veck, t_0)$$
我们关心的是密度场的功率谱:
$$P(\veck,t)\equiv \langle |\rho (\veck, t)|^2\rangle_{\rm stat} = |T(\veck, t)|^2\langle 0\left\vert |\hatphi(\veck, t_0)|^2 \right\vert 0 \rangle$$
和功率谱的统计方差:{\scriptsize
$$ \left(\delta P(\veck,t)\right)^2 \equiv \langle |\rho (\veck, t)|^4 \rangle_{\rm stat} - \left(\langle |\rho (\veck, t)|^2 \rangle_{\rm stat} \right)^ 2 = |T(\veck, t)|^4\langle 0\left\vert |\hatphi(\veck, t_0)|^4 \right\vert 0 \rangle - P(\veck,t)^2 $$ }
其中$\langle \cdot \rangle_{\rm stat}$表示求统计平均。$|0\rangle$表示初始时刻真空态。
试证明$$\delta P(\veck,t) = P(\veck,t)$$
{\scriptsize 注:由于$\rho$是实数场,满足$P(-\veck,t) = P(\veck,t)$(一半信息缺失),所以单个$P(\veck)$的统计方差为$\delta P(\veck,t) = \sqrt{2}P(\veck,t)$。这个方差在宇宙学里叫Cosmic Variance.}
}

\ech
\end{frame}

\end{document}
