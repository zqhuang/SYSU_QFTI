\documentclass[CJK]{beamer}
\usepackage{CJKutf8}
\usepackage{beamerthemesplit}
\usetheme{Malmoe}
\useoutertheme[footline=authortitle]{miniframes}
\usepackage{amsmath}
\usepackage{amssymb}
\usepackage{graphicx}
\usepackage{color}
\graphicspath{{figures/}}
\def \bch {\begin{CJK}{UTF8}{gbsn}}
\def \ech {\end{CJK}}
\def \bex {\begin{minipage}{0.3\textwidth}\includegraphics[width=1in]{jugelizi.png}\end{minipage}\begin{minipage}{0.6\textwidth}}
\def \eex {\end{minipage}}
\def \chtitle#1 {\frametitle{\bch #1 \ech}}
\def \skipline { {\vskip 0.1in}}
\def \langr {\mathcal{L}}
\def \hamil {\mathcal{H}}
\def \vecx {\mathbf{x}}
\def \veck {\mathbf{k}}
\def \vecp {\mathbf{p}}
\def \hatphi {\hat{\phi}}
\def \hatq {\hat{q}}
\def \hatpi  {\hat{\pi}}
\def \vel {\upsilon}
\def \Dint {\mathcal{D}}
\def \adag {{\hat{a}^\dagger}}
\def \hata {\hat{a}}
\def \hatN {\hat{N}}
\def \hatH {\hat{H}}
\def \nket { {| n \rangle}}
\def \bran { {\langle n |}}

\title{Quantum Field Theory I \\ Lesson 20 - openmind II}
\author{}
\date{}


\begin{document}

\begin{frame}
 
\begin{center}
\begin{Large}
\bch
量子场论 I 

{\vskip 0.3in}

第二十课 脑洞大开系列(二)

\ech
\end{Large}
\end{center}

\vskip 0.2in

\bch
课件下载
\ech
https://github.com/zqhuang/SYSU\_QFTI

\end{frame}



\begin{frame}
\chtitle{先考虑守恒律}
\bch
利用守恒律往往可以跳过繁复的计算直接得到结论
\ech
\end{frame}

\begin{frame}
\chtitle{例子1}
\bch
证明一对正负电子湮灭不可能只产生一个光子。
\ech
\end{frame}

\begin{frame}
\chtitle{例子2}
\bch
在“孪生子悖论”中,当两个孪生子见面时,到底是谁更年轻?
\ech
\end{frame}

\begin{frame}
\chtitle{理科PhD学生第二守则}
\bch
当不知道统计对象满足什么统计时,先假设它是高斯分布。
\ech
\end{frame}

\begin{frame}
\chtitle{统计学里超出小学数学的两个定理之一}
\bch
中心极限定理(物理学家版本): 满足同一统计的很多个变量的和满足高斯统计。

\skiplines
{\tiny
注:另一个超出小学数学的统计定理和Monte Carlo Markov Chain有关,这里不作介绍}
\ech
\end{frame}

\begin{frame}
\chtitle{例子3}
\bch
假设卫星每年的维护成本为发射成本的1/20, 试从经济学角度阐述为什么很多卫星观测实验(例如观测宇宙背景微波辐射的Planck卫星)都大大少于20年。
\ech
\end{frame}

\begin{frame}
\chtitle{例子4}
\bch
小明从淘宝买了个铿跌牌体重秤,每次测量体重都有10kg的随机误差。忽略所有系统误差。小明为了把体重测量到0.1kg的精度,需要进行多少次测量?
\ech
\end{frame}


\begin{frame}
\chtitle{例子5}
\bch
物理与天文学院从京东买了个铿跌牌体重秤和布铿跌牌卷尺,体重秤每次测量有10kg的随机误差,卷尺误差可以忽略。忽略所有系统误差。现在要测量1000名学生的身高$h$和体重$w$的最小二乘法拟合线性关系$w = k h$的系数$k$。为了节省时间每个人只测一次体重和一次身高。显然$k$的测量结果会有一定误差来自于体重秤的随机误差。试问:能否通过把身高分成若干个小段,每个小段内的体重先求平均的方法来减少由于体重秤带来的误差?
\ech
\end{frame}


\end{document}


