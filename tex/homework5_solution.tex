\documentclass[CJK]{beamer}
\usepackage{CJKutf8}
\usepackage{beamerthemesplit}
\usetheme{Malmoe}
\useoutertheme[footline=authortitle]{miniframes}
\usepackage{amsmath}
\usepackage{amssymb}
\usepackage{graphicx}
\usepackage{color}
\graphicspath{{figures/}}
\def \bch {\begin{CJK}{UTF8}{gbsn}}
\def \ech {\end{CJK}}
\def \bex {\begin{minipage}{0.3\textwidth}\includegraphics[width=1in]{jugelizi.png}\end{minipage}\begin{minipage}{0.6\textwidth}}
\def \eex {\end{minipage}}
\def \chtitle#1 {\frametitle{\bch #1 \ech}}
\def \skipline { {\vskip 0.1in}}
\def \langr {\mathcal{L}}
\def \hamil {\mathcal{H}}
\def \vecx {\mathbf{x}}
\def \veck {\mathbf{k}}
\def \vecp {\mathbf{p}}
\def \hatphi {\hat{\phi}}
\def \hatq {\hat{q}}
\def \hatpi  {\hat{\pi}}
\def \vel {\upsilon}
\def \Dint {\mathcal{D}}
\def \adag {{\hat{a}^\dagger}}
\def \hata {\hat{a}}
\def \hatN {\hat{N}}
\def \hatH {\hat{H}}
\def \nket { {| n \rangle}}
\def \bran { {\langle n |}}

\title{Quantum Field Theory I \\ Homework 5 solution}
  \author{}
  \date{}


\begin{document}

\begin{frame}
 
\begin{center}
\begin{Large}
\bch
量子场论 I 

{\vskip 0.3in}

第五次课后作业参考答案
\skipline
\skipline

如发现参考答案有错误请不吝告知(微信zhiqihuang或邮箱huangzhq25@sysu.edu.cn)
\ech
\end{Large}
\end{center}

\vskip 0.2in

\bch
课件下载
\ech
https://github.com/zqhuang/SYSU\_QFTI

\end{frame}


\begin{frame}
\chtitle{第1题 题目和思路}
\bch
{\small
题目:考虑一个有自相互作用的复标量场
$$ \lagr = \partial^\mu\phi^\dagger \partial_\mu \phi - m^2\phi^\dagger\phi - \frac{\lambda}{4}\left(\phi^\dagger\phi\right)^2$$
这里$0<\lambda\ll 1$。我们已经知道复标量场有$a$, $b$两种粒子互为反粒子。两个动量为$p_1$, $p_2$的$a$粒子发生散射变为动量为$p_3$, $p_4$的粒子。仅考虑最低阶近似。证明末态粒子只能都是$a$粒子,并求出散射振幅。}

\skipline

思路:参考课上对实标量场$\lambda\phi^4$相互作用的处理,在相互作用表象算到一阶微扰。
\ech
\end{frame}

\begin{frame}
\chtitle{第1题解答}
\bch
{\small
量子化的复标量场为:
$$\hatphi(x) = \frac{1}{(2\pi)^{3/2}} \int \sqrt{\frac{d^3\veck}{2\omega}} \left(\hata_{\veck} e^{-ik_\mu x^\mu} + \bdag_{\veck}e^{ik_\mu x^\mu}\right) $$

散射概率幅的一阶微扰近似为
$$ \frac{-\ii \lambda}{4}\langle p_3 p_4|\int d^4x\, (\hatphi^\dagger\hatphi)^2|p_1 p_2\rangle $$
因为已经知道$p_1$, $p_2$为$a$粒子,要使矩阵元非零,只能从两个$\hatphi$中分别提取$\hata_{\vecp_1}$和$\hata_{\vecp_2}$(这样有两种不同取法)。剩下的$\hatphi^\dagger$包含的产生算符都是$\adag$,所以末态粒子只能都是$a$粒子,要使矩阵元非零只能从两个$\hatphi^\dagger$中分别提取出$\adag_{\vecp_3}$和$\adag_{\vecp_4}$(有两种不同的取法)。总计的四种提取方法和外面的$1/4$因子抵消。提取这些产生湮灭算符带来的因子为$\frac{1}{4\sqrt{\omega_1\omega_2\omega_3\omega_4}}$。所以最后散射振幅为
$$\calM = \frac{- \ii\lambda}{ 4\sqrt{\omega_1\omega_2\omega_3\omega_4}}$$
}

\ech
\end{frame}

\begin{frame}
\chtitle{第2题 题目和思路}
\bch
{\small
题目:仍然考虑上题中的有自相互作用的复标量场
$$ \lagr = \partial^\mu\phi^\dagger \partial_\mu \phi - m^2\phi^\dagger\phi - \frac{\lambda}{4}\left(\phi^\dagger\phi\right)^2$$
试在质心参考系求两个动量为$p_1$, $p_2$的$a$粒子发生散射的散射截面,仍只要求算到最低阶近似。
}

\skipline
{\small
思路:利用课上推导的散射截面对散射振幅的依赖关系}

\ech
\end{frame}

\begin{frame}
\chtitle{第2题解答}
\bch
{\small
课上对实标量场的散射截面的计算并没有限制散射振幅是如何算出来的。所以结论可以直接套用:
$$\sigma = \frac{\lambda^2}{32\pi E^2_{\rm tot}}$$
}
\ech
\end{frame}



\begin{frame}
\chtitle{第3题 题目和思路}
\bch{\small 

题目:对自由实标量场$\phi$,定义$n$点Green函数为:
$$G(x_1, x_2,\ldots,x_n) \equiv \langle 0| \torder{\phi(x_1)\phi(x_2)\dots\phi(x_n)}|0\rangle$$
其中$|0\rangle$为真空态。
\skipline

证明:当$n$为奇数时$G(x_1, x_2,\ldots,x_n)=0$,当$n$为偶数时$G(x_1, x_2,\ldots,x_n)$等于把$x_1, x_2,\ldots,x_n$进行两两配对所得两点Green函数的乘积的遍历配对方式之和,例如$n=4$时,
$$G(x_1, x_2, x_3, x_4) = G(x_1, x_2)G(x_3, x_4) + G(x_1, x_3)G(x_2, x_4) + G(x_1, x_4) G(x_2, x_3)$$
}

\skiplines
{\small 思路:用Wick定理或者路径积分基本定理}
\ech
\end{frame}


\begin{frame}
\chtitle{第3题第一种解答}
\bch{\small 
利用Wick定理把$\torder{\hatphi(x_1)\hatphi(x_2)\ldots \hatphi(x_n)}$展开成带所有可能收缩的正则排序之和。因为含产生或湮灭算符的正则排序在真空态的平均值为零,所以留下的非零项必然是仅含两两配对收缩的项。又因
$$\contr{\hatphi(x_i)}{\hatphi(x_j)} = \langle 0 | \torder{\hatphi(x_i)\hatphi(x_j)}| 0 \rangle = G(x_i, x_j)$$
故命题得证。
}
\ech
\end{frame}

\begin{frame}
\chtitle{第3题第二种解答}
\bch
{\small 
利用路径积分基本定理和多元高斯积分公式:
$$G(x_1, x_2,\ldots,x_n) =  \frac{1}{(d^3\vecx)^{n/2}}\left.\left(\frac{\partial^n}{\partial J_1 \partial J_2\ldots\partial J_n} e^{\frac{1}{2}J^TCJ}\right)\right\vert_{J=0}$$
其中$J = (J_1, J_2, \ldots, J_n, J_{n+1}, \ldots)^T$是辅助源矢量,$C$是某固定的矩阵。
因最后要取$J=0$的极限,等式右边的$n$阶偏微分展开式中包含$J$的分量的项均无贡献,故非零项仅剩下如
$$\frac{1}{(d^3\vecx)^{n/2}} C_{i_1 i_2}C_{i_3 i_4} \ldots C_{i_{n-1}i_n}$$
的项,其中$(i_1, i_2), \ldots, (i_{n=1},i_n)$是$1,2,\ldots, n$ 的所有可能两两配对。
特别地,$n=2$时
$$G(x_{i_1},x_{i_2}) = \frac{1}{d^3\vecx}C_{i_1i_2}$$
故命题得证。
}
\ech
\end{frame}


\begin{frame}
\chtitle{第4题 题目和思路}
\bch
{\small 
题目:仍然考虑自由实标量场$\phi$,考虑固定时刻的傅立叶空间$\phi(\veck)$,定义$n$点功率谱为
$$P(\veck_1,\veck_2,\ldots,\veck_n) \equiv \langle 0 | \hatphi(\veck_1)\hatphi(\veck_2)\ldots\hatphi(\veck_n)|0\rangle$$
证明:若总波矢$\sum_{i=1}^n\veck_i$不为零,则$P(\veck_1,\veck_2,\ldots,\veck_n) = 0$。

\skipline
思路:利用$\hatphi_{\veck}$的表达式

}

\ech
\end{frame}

\begin{frame}
\chtitle{第4题 解答}
\bch
{\small 
因为
$$\hatphi_\veck = \frac{1}{\sqrt{2\omega d^3\veck}}\left(\hata_\veck + \adag_{-\veck}\right)$$
所以$\hatphi_{\veck_i}$作用于$\hatphi(\veck_{i+1})\hatphi(\veck_{i+2})\ldots\hatphi(\veck_n)|0\rangle$的效果或者是湮灭一个动量为$\veck_i$的粒子,或者产生一个动量为$-\veck_i$的粒子,或者把整个式子变为零。要使整个式子不为零,则必须$\hatphi(\veck_1)\hatphi(\veck_2)\ldots\hatphi(\veck_n)|0\rangle$是总动量为$-\sum_{i=1}^{n}\veck_i$的粒子态。又左边是真空态,所以必须满足$\sum_{i=1}^{n}\veck_i=0$才有可能使功率谱非零。

}

\ech
\end{frame}


\begin{frame}
\chtitle{第5题 题目和思路}
\bch
{\scriptsize
题目:现代宇宙学标准模型认为宇宙中一切密度扰动均源于宇宙早期暴涨时的实标量场$\phi$的真空量子波动。我们在傅立叶空间考虑任意时刻的密度扰动$\rho(\veck, t)$和初始时刻的场$\phi(\veck, t_0)$。它们之间可以用一个线性增长因子$T(\veck,t)$联系起来
$$\rho (\veck, t) = T(\veck, t) \phi(\veck, t_0)$$
我们关心的是密度场的功率谱:
$$P(\veck),t\equiv \langle |\rho (\veck, t)|^2\rangle_{\rm stat} = |T(\veck, t)|^2\langle 0\left\vert |\hatphi(\veck, t_0)|^2 \right\vert 0 \rangle$$
和功率谱的统计方差:{\scriptsize
$$ \delta P(\veck,t)^2 \equiv \langle |\rho (\veck, t)|^4 \rangle_{\rm stat} - \left(\langle |\rho (\veck, t)|^2 \rangle_{\rm stat} \right)^ 2 = |T(\veck, t)|^4\langle 0\left\vert |\hatphi(\veck, t_0)|^4 \right\vert 0 \rangle - P(\veck,t)^2 $$ }
其中$\langle \cdot \rangle_{\rm stat}$表示求统计平均。$|0\rangle$表示初始时刻真空态。
试证明$$\delta P(\veck,t) = P(\veck,t)$$
}
{\small 
思路:仍然是直接利用$\hatphi(\veck)$的表达式}
\ech
\end{frame}


\begin{frame}
\chtitle{第5题 解答}
\bch
{\scriptsize
在固定时刻$t_0$,利用
$$\hatphi(\veck)= \frac{1}{\sqrt{2\omega d^3\veck}}\left(\hata_\veck + \adag_{-\veck}\right);\ \ \ \hatphi^\dagger(\veck) = \frac{1}{\sqrt{2\omega d^3\veck}}\left(\adag_\veck + \hata_{-\veck}\right)$$
得到
$$\contr{\hatphi^\dagger(\veck)}{\hatphi(\veck)} =\contr{\hatphi(\veck)}{\hatphi^\dagger(\veck)} = \frac{1}{2\omega d^3\veck};\ \ \ \contr{\hatphi(\veck)}{\hatphi(\veck)} = \contr{\hatphi^\dagger(\veck)}{\hatphi^\dagger(\veck)} = 0$$
根据Wick定理以及非零长度正则乘积的真空平均态为零的性质
\be
 \langle 0 | \hatphi^\dagger(\veck)\hatphi(\veck) | 0\rangle = \contr{\hatphi^\dagger(\veck)}{\hatphi(\veck)}
\ee
和
\bea
&& \langle 0 | (\hatphi^\dagger(\veck)\hatphi(\veck))^2 | 0 \rangle  \newl
&=& \contr{\hatphi^\dagger(\veck)}{\hatphi(\veck)}\contr{\hatphi^\dagger(\veck)}{\hatphi(\veck)} + \contraction{}{\hatphi^\dagger(\veck)}{\hatphi(\veck)\hatphi^\dagger(\veck)}{\hatphi(\veck)}
\bcontraction{\hatphi^\dagger(\veck)}{\hatphi(\veck)}{}{\hatphi^\dagger(\veck)}
{\hatphi^\dagger(\veck)\hatphi(\veck)\hatphi^\dagger(\veck)\hatphi(\veck)} \newl
&=& 2\left( \langle 0 | (\hatphi^\dagger(\veck)\hatphi(\veck))^2 | 0 \rangle\right)^2
\eea
所以
\be
\delta P(\veck,t)^2 = 2 P(\veck, t)^2 - P(\veck,t)^2 = P(\veck,t)^2
\ee
即均方根为
$$\delta P(\veck,t) = P(\veck,t)$$
}
\ech
\end{frame}

\end{document}
