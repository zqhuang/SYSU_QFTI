\documentclass[CJK]{beamer}
\usepackage{CJKutf8}
\usepackage{beamerthemesplit}
\usetheme{Malmoe}
\useoutertheme[footline=authortitle]{miniframes}
\usepackage{amsmath}
\usepackage{amssymb}
\usepackage{graphicx}
\usepackage{color}
\graphicspath{{figures/}}
\def \bch {\begin{CJK}{UTF8}{gbsn}}
\def \ech {\end{CJK}}
\def \bex {\begin{minipage}{0.3\textwidth}\includegraphics[width=1in]{jugelizi.png}\end{minipage}\begin{minipage}{0.6\textwidth}}
\def \eex {\end{minipage}}
\def \chtitle#1 {\frametitle{\bch #1 \ech}}
\def \skipline { {\vskip 0.1in}}
\def \langr {\mathcal{L}}
\def \hamil {\mathcal{H}}
\def \vecx {\mathbf{x}}
\def \veck {\mathbf{k}}
\def \vecp {\mathbf{p}}
\def \hatphi {\hat{\phi}}
\def \hatq {\hat{q}}
\def \hatpi  {\hat{\pi}}
\def \vel {\upsilon}
\def \Dint {\mathcal{D}}
\def \adag {{\hat{a}^\dagger}}
\def \hata {\hat{a}}
\def \hatN {\hat{N}}
\def \hatH {\hat{H}}
\def \nket { {| n \rangle}}
\def \bran { {\langle n |}}

\title{Quantum Field Theory I \\ Lesson 13 - The Second Feynman diagram}
\author{}
\date{}


\begin{document}

\begin{frame}
 
\begin{center}
\begin{Large}
\bch
量子场论 I 

{\vskip 0.3in}

第十三课 第二个Feynman图

\ech
\end{Large}
\end{center}

\vskip 0.2in

\bch
课件下载
\ech
https://github.com/zqhuang/SYSU\_QFTI

\end{frame}

\begin{frame}
\chtitle{关于形式解的小bug}
\bch

$$|\psi\rangle_{t_2} \qeq e^{-\ii\int_{t_1}^{t_2} \hatH_I(t) dt} |\psi\rangle_{t_1}$$

\bmini{0.35}
\includegraphics[width=1.2in]{douwone.jpg}
\emini
\bmini{0.55}
上节课我们讲到的Interaction绘景的态的形式解其实有bug。

当你把$e^{-\ii\int H_Idt}$展开到二阶以上时这个bug就可能被触发。

\emini


\skipline
{\small
出bug的原因:在不同时刻的$\hatH_I$如果不对易,则$\frac{d e^{-\ii\int^t \hatH_I dt'}}{dt} \ne -\ii \hatH_I(t) e^{-\ii\int^t \hatH_I dt'}$
}
\skipline
\ech
\end{frame}

\begin{frame}
\chtitle{神奇的编时算符}
\bch
我们引入编时算符$\calT$,它作用于一串算符的乘积时把算符乘积顺序按时间从晚到早排序,而等时的算符保持乘积次序不变。例如,若$t_1>t_2 = t_3>t_4$,则
$$\torder{\hat{A}(t_3)\hat{B}(t_2)\hat{C}(t_4)\hat{D}(t_1)} = \hat{D}(t_1)\hat{A}(t_3)\hat{B}(t_2)\hat{C}(t_4)$$

\skipline
试证明:对任意算符$\hat{O}$均有
$$\frac{d \left(\torder{e^{\int^t_{t_0} \hat{O}(t')dt'}}\right)}{dt} = \hat{O}(t) \left(\torder{e^{\int^t_{t_0} \hat{O}(t')dt'}}\right)$$

\ech
\end{frame}

\begin{frame}
\chtitle{神奇的编时超算符}
\bch
在相互作用表象,
$$|\psi\rangle_{t_2} = \torder{ e^{-\ii\int_{t_1}^{t_2} \hatH_I(t) dt} }|\psi\rangle_{t_1}$$

(这次真的没bug了)
\ech
\end{frame}

\begin{frame} 
\chtitle{三次项的相互作用} 
\bch
考虑拉氏密度为
$$\lagr = \frac{1}{2}\partial^\mu\phi\partial_\mu\phi - \frac{1}{2}m^2\phi^2 - \frac{g}{3!}\phi^3- \frac{\lambda}{4!}\phi^4$$
的实标量场。其中$\lambda>0$和$g$都是耦合常数。我们仅考虑它们很小($\lambda \ll 1$, $g\ll m$)的情况。

我们取如下的相互作用表象:
$$ H_0 = H_{\rm free},\ H_I = \int d^3\vecx\, \left(\frac{g}{3!}\phi^3 + \frac{\lambda}{4!} \phi^4\right)$$
其中$H_{\rm free} = \int d^3\vecx\, \frac{1}{2}(\dot\phi^2 + |\nabla \phi|^2 + m^2\phi^2)$是自由场的Hamilton量。

\ech
\end{frame}




\begin{frame} 
\chtitle{继续上节课的热身问题} 
\bch
上节课的热身问题是:
\skipline

求两个动量为$\vecp_1$, $\vecp_2$的粒子发生散射,变为两个动量为$\vecp_3$, $\vecp_4$的粒子的概率幅。
$$\calM T/V \delta(p_1+p_2-p_3-p_4)d^4k= \langle \vecp_3, \vecp_4|  e^{-\ii\int_{-\infty}^\infty \hat{H}_I dt}|\vecp_1,\vecp_2\rangle$$
如上节课末尾所讲,我们已经把$\delta(p_1+p_2-p_3-p_4)d^4k$引入了左边的定义式。
\skipline

我们仍假设这四个动量$\vecp_1$, $\vecp_2$, $\vecp_3$, $\vecp_4$互不相同,即初态和末态粒子都是可以区分的。
\ech
\end{frame}

\begin{frame} 
\chtitle{一阶微扰展开} 
\bch
我们做微扰展开,对带$\lambda$的项我们仍展开到一次,对带$g$的项我们要展开到二次。这是因为单个$g\phi^3$最多只能提供三个产生湮灭算符的乘积,无法湮灭两个粒子并产生两个新粒子。

\bea
\torder{e^{-\ii\int_{-\infty}^\infty \hat{H}_I dt}}&\approx& 1-\ii\int d^4x\,\frac{g}{3!}\hat\phi(x)^3- \ii\int d^4x\, \frac{\lambda}{4!}\hat\phi(x)^4  \newl
&& -\frac{1}{2}\torder{\left(\int d^4x\,\frac{g}{3!}\hat\phi(x)^3\right)\left(\int d^4y\,\frac{g}{3!}\hat\phi(y)^3\right)} 
\eea

第一项和第二项均没有贡献,第三项的贡献我们上节课算了。我们这节课来计算第四项的贡献:
$$ -\frac{g^2}{2(3!)^2} \int d^4x\, \int d^4y\,\langle \vecp_3, \vecp_4|  \hat\phi(x)^3 \hat\phi(y)^3 |\vecp_1,\vecp_2\rangle$$

\ech
\end{frame}

\end{document}


