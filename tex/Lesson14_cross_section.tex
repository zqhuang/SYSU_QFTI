\documentclass[CJK]{beamer}
\usepackage{CJKutf8}
\usepackage{beamerthemesplit}
\usetheme{Malmoe}
\useoutertheme[footline=authortitle]{miniframes}
\usepackage{amsmath}
\usepackage{amssymb}
\usepackage{graphicx}
\usepackage{color}
\graphicspath{{figures/}}
\def \bch {\begin{CJK}{UTF8}{gbsn}}
\def \ech {\end{CJK}}
\def \bex {\begin{minipage}{0.3\textwidth}\includegraphics[width=1in]{jugelizi.png}\end{minipage}\begin{minipage}{0.6\textwidth}}
\def \eex {\end{minipage}}
\def \chtitle#1 {\frametitle{\bch #1 \ech}}
\def \skipline { {\vskip 0.1in}}
\def \langr {\mathcal{L}}
\def \hamil {\mathcal{H}}
\def \vecx {\mathbf{x}}
\def \veck {\mathbf{k}}
\def \vecp {\mathbf{p}}
\def \hatphi {\hat{\phi}}
\def \hatq {\hat{q}}
\def \hatpi  {\hat{\pi}}
\def \vel {\upsilon}
\def \Dint {\mathcal{D}}
\def \adag {{\hat{a}^\dagger}}
\def \hata {\hat{a}}
\def \hatN {\hat{N}}
\def \hatH {\hat{H}}
\def \nket { {| n \rangle}}
\def \bran { {\langle n |}}

\title{Quantum Field Theory I \\ Lesson 14 - Cross section}
\author{}
\date{}


\begin{document}

\begin{frame}
 
\begin{center}
\begin{Large}
\bch
量子场论 I 

{\vskip 0.3in}

第十四课 散射截面

\ech
\end{Large}
\end{center}

\vskip 0.2in

\bch
课件下载
\ech
https://github.com/zqhuang/SYSU\_QFTI

\end{frame}

\begin{frame}
\chtitle{散射截面是粒子的有效面积}
\bch
对散射问题,如果一开始就是取一个粒子为参照系,即$\vecp_1 = 0$。不妨取粒子2的运动方向为$z$轴,其动量为
$$\vecp_2 = \left(0, 0, \frac{m\upsilon}{\sqrt{1-\upsilon^2}}\right)$$
其中$\upsilon$是粒子2在这个参考系里的运动速度。
\skipline

散射截面$\sigma$是粒子2的“有效面积”,粒子2在总时间$T$内扫过的体积为$\sigma \upsilon T$。在总体积为$V$的情况下,粒子2“碰撞”到粒子1的概率为
$$\frac{\sigma \upsilon T}{V}$$


\ech
\end{frame}


\begin{frame}
\chtitle{和散射振幅联系起来}
\bch
另一方面,我们计算了发生散射的概率为
{\small
$$ \sum_{\vecp_3, \vecp_4} \left\vert\calM \frac{T}{V}  \delta(p_1+p_2-p_3-p_4) d^4k\right\vert^2 = \frac{1}{2!}\sum_{\vecp_3, \vecp_4} |\calM|^2 \frac{T^2}{V^2}  \delta(p_1+p_2-p_3-p_4) d^4k $$
}
因为是总概率,所以要对所有$\vecp_3$, $\vecp_4$求和。但是,“产生一个动量为$\vecp_3$的粒子和另一个动量为$\vecp_4$的粒子”和“产生一个动量为$\vecp_4$的粒子和另一个动量为$\vecp_3$的粒子”描述的是同一件事情。所以有了外面的$\frac{1}{2!}$因子。

要求能量动量守恒的因子$\delta(p_1+p_2-p_3-p_4) d^4k$,因为它满足能量动量守恒时为1,否则为零。所以它的平方总是等于自己。

\skipline 
令两种方法算出来的概率相等,即有
$$\sigma  = \frac{1}{2!}\sum_{\vecp_3, \vecp_4} |\calM|^2 \frac{T}{V\upsilon}  \delta(p_1+p_2-p_3-p_4) d^4k $$
\ech
\end{frame}


\begin{frame}
\chtitle{化为标准形式}
\bch
再利用$d^4k = \frac{(2\pi)^4}{VT}$,以及$d^3\vecp_3 = d^3\vecp_4 = \frac{(2\pi)^3}{V}$,就得到散射截面公式的标准形式
$$\sigma  = \frac{1}{2!(2\pi)^2} \int d^3\vecp_3 \int d^3\vecp_4\, \frac{|\calM|^2}{\upsilon}  \delta(p_1+p_2-p_3-p_4)  $$
注意式中的$\delta$函数是四维的。并且,如果产生的两个粒子是两种不同的粒子,则没有$\frac{1}{2!}$因子。

\ech
\end{frame}


\begin{frame}
\bch
问题是,只会小学数学的我们,真的能算出这个积分吗?
\ech
\end{frame}


\begin{frame}
\chtitle{先算最简单的}
\bch
先来看最简单的情况,假设只有四次耦合项$\frac{\lambda}{4!}\phi^4$。那么
$$|\calM|^2 = \frac{\lambda^2}{16\omega_1\omega_2\omega_3\omega_4}$$

代入散射截面公式,
{\small
\bea
\sigma &=& \frac{\lambda^2}{128\pi^2\upsilon \omega_1\omega_2} \newl
&& \times \int d^3\vecp_3 \int d^3\vecp_4\, \frac{1}{\omega_3\omega_4}\delta\left(\omega_1+\omega_2-\omega_3-\omega_4\right)\delta\left(\vecp_2 - \vecp_3 -\vecp_4\right) 
\eea
这里
$$\omega_1 = m,\  \omega_2 = \frac{m}{\sqrt{1-\upsilon^2}},\ \omega_3 = \sqrt{m^2+\vecp_3^2}, \ \omega_4 = \sqrt{m^2+\vecp_4^2}$$
}
 
\ech
\end{frame}


\begin{frame}
\chtitle{把$\vecp_4$积掉}
\bch
{\small 
可以先对$\vecp_4$积分,得到
\be
\sigma = \frac{\lambda^2}{128\pi^2 \upsilon\omega_1\omega_2} \int d^3\vecp_3 \, \frac{1}{\omega_3\omega_4}\delta\left(\omega_1+\omega_2-\omega_3-\omega_4\right)
\ee
当然,上式中$\omega_4$的含义发生了变化:
$$\omega_4 = \sqrt{m^2+(\vecp_2-\vecp_3)^2}$$
事实上,在取定的参考系里,$\omega_4\ge m$,又要能量守恒$\omega_1+\omega_2  =\omega_3+\omega_4$,所以$\omega_3\le \omega_2$。那么上面的积分仅在
$$|\vecp_3| \le |\vecp_2| $$
时有非零贡献。
}
\ech
\end{frame}


\begin{frame}
\chtitle{取球坐标系}
\bch
{\small
设$\vecp_3$的球座标系坐标为$(|\vecp_3|,\theta,\varphi)$, 则
$$\int d^3\vecp_3 = \int_0^{|\vecp_2|} |\vecp_3|^2 d|\vecp_3| \int_{-1}^1 d(\cos\theta) \int_0^{2\pi}d\varphi $$
因为$\sigma$的积分函数与$\varphi$无关,可以先对$\varphi$积分得到$2\pi$。
于是
$$\sigma = \frac{\lambda^2}{64\pi \upsilon\omega_1\omega_2}  \int_0^{|\vecp_2|} |\vecp_3|^2 d|\vecp_3| \int_{-1}^1 d(\cos\theta) \frac{1}{\omega_3\omega_4}\delta(\omega_1+\omega_2-\omega_3-\omega_4)  $$
这时$\omega_3$, $\omega_4$的含义为
$$\omega_3 = \sqrt{m^2+|\vecp_3|^2},\ \omega_4 = \sqrt{m^2+ |\vecp_2|^2 + |\vecp_3|^2 - 2|\vecp_2||\vecp_3|\cos\theta} $$ 
} 
\ech
\end{frame}

\begin{frame}
\chtitle{积掉$\cos\theta$}
\bch
{\small
先对$\cos\theta$进行积分,为此我们要计算出
$$\left\vert\frac{d(\omega_1+\omega_2-\omega_3-\omega_4)}{d (\cos\theta)}\right\vert = \frac{|\vecp_2||\vecp_3|}{\omega_4}$$ 
再利用积分公式
$$\int \delta\left(f(x)\right) dx = \sum_{x*:\ f(x*)=0} \frac{1}{|f'(x*)|}$$
就得到
$$\sigma = \frac{\lambda^2}{64\pi \upsilon\omega_1\omega_2|\vecp_2|}  \int_0^{|\vecp_2|}  d|\vecp_3| \frac{|\vecp_3|}{\omega_3} $$

} 
\ech
\end{frame}

\begin{frame}
\chtitle{最后积掉$|\vecp_3|$}
\bch
最后利用$|\vecp_3|d|\vecp_3| =  \omega_3 d\omega_3$,就得到
\bea
\sigma &=& \frac{\lambda^2}{64\pi \upsilon\omega_1\omega_2|\vecp_2|}  \int_m^{\omega_2}  d\omega_3  \newl
&=& \frac{\lambda^2(\omega_2 - m)}{64\pi \upsilon\omega_1\omega_2|\vecp_2|} \newl
&=& \frac{\lambda^2}{64\pi  m^2\left(1+\frac{1}{\sqrt{1-\upsilon^2}}\right)} 
\eea
非相对论极限$\upsilon\ll 1$下,
$$\sigma \approx \frac{\lambda^2}{128\pi m^2}$$
\ech
\end{frame}

\begin{frame}
\chtitle{取质心参照系计算更方便}
\bch
{\small 
实际上,散射截面不一定要在一个粒子的静止参照系里计算。我们可以取质心参照系,设粒子1的速度为沿$z$轴向上$u$,粒子2的速度相反。在这个参照系里,若要能量动量都守恒则必须有
$$\vecp_3+\vecp_4 = 0$$
且
$$|\vecp_1| = |\vecp_2| = |\vecp_3| = |\vecp_4|$$
因为粒子之间相对速度为$2u$,上面开始几步步骤中需要把$\upsilon$替换为$2u$。直到我们在球坐标系里写下:
$$\sigma = \frac{\lambda^2}{64\pi (2u)\omega_1\omega_2}  \int_0^{\infty} |\vecp_3|^2 d|\vecp_3| \int_{-1}^1 d(\cos\theta) \frac{1}{\omega_3\omega_4}\delta(\omega_1+\omega_2-\omega_3-\omega_4)  $$
注意现在$\omega_3 = \omega_4 = \sqrt{|\vecp_3|^2+m^2}$,所以积分函数就不依赖于$\theta$,直接可以对$\theta$积分得到
$$\sigma = \frac{\lambda^2}{32\pi (2u)\omega_1\omega_2}  \int_0^{\infty} |\vecp_3|^2 d|\vecp_3|  \frac{1}{\omega_3\omega_4}\delta(\omega_1+\omega_2-\omega_3-\omega_4)  $$
}
\ech
\end{frame}

\begin{frame}
\chtitle{取质心参照系计算更方便}
\bch
{\small 
然后我们计算出
$$\left\vert\frac{d(\omega_1+\omega_2-\omega_3-\omega_4)}{d|\vecp_3|} \right\vert = \frac{2|\vecp_3|}{\omega_4}=\frac{2|\vecp_1|}{\omega_1}$$
利用$\delta$函数积分公式即得
$$\sigma = \frac{\lambda^2|\vecp_1|  }{128 \pi u \omega_1^3} $$
再利用$u\omega_1 =|\vecp_1|$,上式就化简为
$$\sigma = \frac{\lambda^2}{128 \pi \omega_1^2} = \frac{\lambda^2}{32 \pi E_{\rm tot}^2}  $$
其中$E_{\rm tot}=2\omega_1 $是质心系总能量。
}

\ech
\end{frame}


\begin{frame}
\chtitle{课堂讨论}
\bch
在两个参照系里算出来的结果一样吗?
\ech
\end{frame}

\end{document}


