\documentclass[CJK]{beamer}
\usepackage{CJKutf8}
\usepackage{beamerthemesplit}
\usetheme{Malmoe}
\useoutertheme[footline=authortitle]{miniframes}
\usepackage{amsmath}
\usepackage{amssymb}
\usepackage{graphicx}
\usepackage{color}
\graphicspath{{figures/}}
\def \bch {\begin{CJK}{UTF8}{gbsn}}
\def \ech {\end{CJK}}
\def \bex {\begin{minipage}{0.3\textwidth}\includegraphics[width=1in]{jugelizi.png}\end{minipage}\begin{minipage}{0.6\textwidth}}
\def \eex {\end{minipage}}
\def \chtitle#1 {\frametitle{\bch #1 \ech}}
\def \skipline { {\vskip 0.1in}}
\def \langr {\mathcal{L}}
\def \hamil {\mathcal{H}}
\def \vecx {\mathbf{x}}
\def \veck {\mathbf{k}}
\def \vecp {\mathbf{p}}
\def \hatphi {\hat{\phi}}
\def \hatq {\hat{q}}
\def \hatpi  {\hat{\pi}}
\def \vel {\upsilon}
\def \Dint {\mathcal{D}}
\def \adag {{\hat{a}^\dagger}}
\def \hata {\hat{a}}
\def \hatN {\hat{N}}
\def \hatH {\hat{H}}
\def \nket { {| n \rangle}}
\def \bran { {\langle n |}}

\title{Quantum Field Theory I \\ Homework 1}
  \author{}
  \date{}


\begin{document}

\begin{frame}
 
\begin{center}
\begin{Large}
\bch
量子场论 I 

{\vskip 0.3in}

第一次课后作业 (共八次,每次2.5分)

交作业时间: 9月19日,星期一,13:30pm

\ech
\end{Large}
\end{center}

\vskip 0.2in

\bch
课件下载
\ech
https://github.com/zqhuang/SYSU\_QFTI

\end{frame}

\begin{frame}
\chtitle{第1题(0.5分)}
\bch
若算得的截面为$\sigma = 10^{-3}/m_W^2$, $m_W \approx 80 GeV$是$W^{\pm}$的粒子质量,试换算出以$\mathrm{cm}^2$为单位的截面值。若算得的寿命的$\tau = 100/m_w$,试问等于多少秒?
\ech
\end{frame}



\begin{frame}
\chtitle{第2题(0.5分)}
\bch
证明任意四维时空坐标系$(x^0, x^1, x^2, x^3)$的积分元$\sqrt{-g}d^4x$是一个标量。其中$g$是度规矩阵$g_{\mu\nu}$的行列式的简写,$d^4x$是积分元$dx^1dx^2dx^3dx^4$的简写。
\ech
\end{frame}


\begin{frame}
\chtitle{第3题(0.5分)}
\bch
考虑一维空间x和一维时间t构成的时空里的标量场$\phi(x,t)$,若其作用量为
$$ S = \int dx dt \   \frac{1}{2} \left[\left(\frac{\partial \phi}{\partial t}\right)^2 - \left(\frac{\partial \phi}{\partial x}\right)^2 - V(\phi)\right]\, ,$$
其中$V(\phi)$为给定的势能函数。试用求作用量稳定点的方法推导$\phi(x, t)$的运动方程并将结果与Euler-Lagrange方程做比较。
\ech
\end{frame}

\begin{frame}
\chtitle{第4题(0.5分)}
\bch
谐振子
$$ S = \int_{-\infty}^\infty \frac{m}{2}\left[(\frac{d\phi}{dt})^2 - \omega^2\phi^2\right] dt$$
把时间维特殊化以后就只有一个自由度$\phi$。写出$\phi$对应的正则动量$\pi$,系统的Hamilton量,以及Hamilton方程。试把$\pi$从两个Hamilton方程中消去,得到的结果和Euler-Lagrange方程一致吗?
\ech
\end{frame}


\begin{frame}
\chtitle{第5题(0.5分)}
\bch
作用量
$$S = \int d^4x\ \lagr(\phi, \partial_\mu\phi)\,.$$
在坐标平移下显然具有不变性。由此用Noether定理推导场$\phi$的能量动量守恒方程。

\skipline

提示:考虑四种无穷小变化$x^\mu \rightarrow x^\mu + \epsilon_r \delta_r^\mu$ ($r = 0, 1, 2, 3$)带来的$\phi$的变化和$\mathcal{L}$的变化。
\ech
\end{frame}


\end{document}
