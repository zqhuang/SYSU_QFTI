\documentclass[CJK]{beamer}
\usepackage{CJKutf8}
\usepackage{beamerthemesplit}
\usetheme{Malmoe}
\useoutertheme[footline=authortitle]{miniframes}
\usepackage{amsmath}
\usepackage{amssymb}
\usepackage{graphicx}
\usepackage{color}
\graphicspath{{figures/}}
\def \bch {\begin{CJK}{UTF8}{gbsn}}
\def \ech {\end{CJK}}
\def \bex {\begin{minipage}{0.3\textwidth}\includegraphics[width=1in]{jugelizi.png}\end{minipage}\begin{minipage}{0.6\textwidth}}
\def \eex {\end{minipage}}
\def \chtitle#1 {\frametitle{\bch #1 \ech}}
\def \skipline { {\vskip 0.1in}}
\def \langr {\mathcal{L}}
\def \hamil {\mathcal{H}}
\def \vecx {\mathbf{x}}
\def \veck {\mathbf{k}}
\def \vecp {\mathbf{p}}
\def \hatphi {\hat{\phi}}
\def \hatq {\hat{q}}
\def \hatpi  {\hat{\pi}}
\def \vel {\upsilon}
\def \Dint {\mathcal{D}}
\def \adag {{\hat{a}^\dagger}}
\def \hata {\hat{a}}
\def \hatN {\hat{N}}
\def \hatH {\hat{H}}
\def \nket { {| n \rangle}}
\def \bran { {\langle n |}}

\title{Quantum Field Theory I \\ Homework 4 solution}
  \author{}
  \date{}


\begin{document}

\begin{frame}
 
\begin{center}
\begin{Large}
\bch
量子场论 I 

{\vskip 0.3in}

第四次课后作业参考答案
\skipline
\skipline

如发现参考答案有错误请不吝告知(微信zhiqihuang或邮箱huangzhq25@sysu.edu.cn)
\ech
\end{Large}
\end{center}

\vskip 0.2in

\bch
课件下载
\ech
https://github.com/zqhuang/SYSU\_QFTI

\end{frame}

\begin{frame}
\chtitle{第1题:题目和思路}
\bch
题目:对两个矢量$A^\mu$, $B^\mu$,证明
$$\mathrm{Tr}(\slashed{A}\slashed{B}) = 4A^\mu B_\mu$$
其中$\mathrm{Tr}$表示矩阵求迹。


\skipline
思路:利用矩阵乘积的迹在轮换乘法次序时不变的性质
\ech
\end{frame}

\begin{frame}
\chtitle{第1题解答}
\bch
$$\mathrm{Tr}(\slashed{A}\slashed{B}) = A_\mu B_\nu \mathrm{Tr}\left(\gamma^\mu\gamma^\nu\right)$$
因$\mathrm{Tr}(XY) = \mathrm{Tr}(YX)$,所以 
$$\mathrm{Tr}(\slashed{A}\slashed{B}) = A_\mu B_\nu \mathrm{Tr}\left(\gamma^\nu\gamma^\mu\right)$$
两式相加除以2,并利用$\gamma^\mu\gamma^\nu + \gamma^\nu\gamma^\mu = 2 g^{\mu\nu}I_{4\times 4}$,
$$\mathrm{Tr}(\slashed{A}\slashed{B}) = A_\mu B_\nu g^{\mu\nu} \mathrm{Tr}(I_{4\times 4}) = 4 A^\mu B_\mu$$
 
\ech
\end{frame}

\begin{frame}
\chtitle{第2题:题目和思路}
\bch
题目:对旋量$\psi$证明$\bar{\psi}\gamma^\mu\gamma^\nu\gamma^\rho\psi$为洛仑兹变换下的三阶张量。


\skipline
思路:这类题目总是要用$\Lambda^\dagger = \gamma^0\Lambda^{-1}\gamma^0$和$\Lambda$矩阵的定义
\ech
\end{frame}

\begin{frame}
\chtitle{第2题解答}
\bch
设洛仑兹变换$x^\mu \rightarrow a^\mu_{\ \nu} x^\nu$对应的旋量变换矩阵为$\Lambda$。利用课上证明了的$\Lambda^\dagger = \gamma^0\Lambda^{-1}\gamma^0$以及$\Lambda$矩阵的定义,
\bea
\bar{\psi}\gamma^\mu\gamma^\nu\gamma^\rho\psi &\rightarrow &\psi^\dagger\Lambda^\dagger \gamma^0 \gamma^\mu\gamma^\nu\gamma^\rho\Lambda\psi  \newl
&=&\psi^\dagger\gamma^0\Lambda^{-1}(\gamma^0)^2\gamma^\mu\gamma^\nu\gamma^\rho\Lambda\psi \newl
&=&\barpsi\Lambda^{-1}\gamma^\mu\gamma^\nu\gamma^\rho\Lambda\psi \newl
&=&\barpsi\Lambda^{-1}\gamma^\mu\Lambda\Lambda^{-1}\gamma^\nu\Lambda\Lambda^{-1}\gamma^\rho\Lambda\psi \newl
&=&\barpsi a^\mu_{\ \alpha}\gamma^\alpha a^\nu_{\ \beta}\gamma^\beta a^\rho_{\ \sigma}\gamma^\sigma \psi \newl
&=&(a^\mu_{\ \alpha} a^\nu_{\ \beta} a^\rho_{\ \sigma})\barpsi \gamma^\alpha \gamma^\beta \gamma^\sigma \psi
\eea
上式表明$\bar{\psi}\gamma^\mu\gamma^\nu\gamma^\rho\psi$满足三阶张量的定义。
\ech
\end{frame}

\begin{frame}
\chtitle{第3题:题目和思路}
\bch
题目:如果一个实标量场$\phi$和一个旋量场$\psi$有相互作用,拉氏密度为
$$\lagr = \frac{1}{2}\partial_\mu\phi\partial^\mu\phi - \frac{1}{2}m^2\phi^2 - \frac{g}{2}\phi^2\bar\psi\psi + \bar\psi(i\slashed{\partial}-m)\psi$$
其中$g$为耦合常数。

试推导$\phi$和$\psi$的运动方程。

\skipline
思路:对$\phi$和$\bar\psi$写出Euler-Lagrange方程即可
\ech
\end{frame}

\begin{frame}
\chtitle{第3题解答}
\bch
直接利用Euler-Lagrange方程得到$\phi$的运动方程
$$ (\partial^2 + m^2 + g\barpsi\psi) \phi = 0$$
$\psi$的运动方程(通过取$\barpsi$的Euler-Lagrange方程得到)
$$ (-\frac{g}{2}\phi^2 + i\slashed{\partial} - m) \psi = 0 $$
\ech
\end{frame}

\begin{frame}
\chtitle{第4题:题目和思路}
\bch
题目:设有三维动量$\veck = (k_x, k_y, k_z)$。请在以$z$方向自旋向上的态$|\uparrow\rangle$和自旋向下的态$|\downarrow\rangle$为基的表象里,写出沿$\veck$方向的电子自旋算符的矩阵表达式,并求它的所有本征值$s$和本征矢$\zeta_{\veck,s}$。

\skipline
思路:同一表象下的算符的和的矩阵表示等于算符的矩阵表示的和。

\ech
\end{frame}


\begin{frame}
\chtitle{第4题解答}
\bch
沿$\veck$方向的自旋算符为
\be
\frac{1}{2} \sigma\cdot\frac{\veck}{|\veck|} = \frac{k_x}{2|\veck|}\sigma^1 + \frac{k_y}{2|\veck|}\sigma^2 + \frac{k_z}{2|\veck|}\sigma^3 = 
\frac{1}{2|\veck|}
\bmat{cc}
k_z & k_x - \ii  k_y \\
k_x + \ii k_y & -k_z
\emat
\ee
本征值$s$满足:
$$(k_z-2|\veck|s)(-k_z - 2|\veck|s) - (k_x+\ii k_y)(k_x-\ii k_y) = 0$$
化简即得$s = \pm 1/2$
对应$s = \pm 1/2$的本征矢量分别为:
{\scriptsize
\be
\zeta_{\veck, 1/2} = \frac{1}{\sqrt{2|\veck|(|\veck|+k_z)}}
\bmat{c}
k_z+|\veck|\\
k_x+\ii k_y
\emat,\ 
\zeta_{\veck, -1/2} = \frac{1}{\sqrt{2|\veck|(|\veck|+k_z)}}
\bmat{c}
k_x-\ii k_y \\
-k_z-|\veck|
\emat
\ee
}

\ech
\end{frame}

\begin{frame}
\chtitle{第5题:题目和思路}
\bch
题目:对$m=0$的旋量和非零的三维动量$\veck$,记相应的四维动量为$k$,证明$\slashed{k}$只有两个线性独立的本征态。 

\skipline
思路:由于矩阵的本征态个数不依赖于表象,我们可以简单取$z$轴沿$\veck$方向写出$\slashed{k}$。
\ech
\end{frame}

\begin{frame}
\chtitle{第5题解答}
\bch
{\small
取$\veck$方向为$z$轴,则$k_\mu = (|\veck|, 0 , 0, -|\veck|)$}
{\scriptsize
\be
\slashed{k} = \gamma^0k_0 + \gamma^3k_3 =
|\veck|
\bmat{rrrr}
1 & 0 & -1 & 0 \\
0 & 1 & 0 & 1 \\
1 & 0 & -1 & 0 \\
0 & -1 & 0 & -1 
\emat
\ee
}
{\small
设$\slashed{k}$有本征值$\lambda$和非零本征矢$\psi$,则利用课上的结论$\slashed{k}^2=m^2=0$有
$$\lambda^2\psi = \lambda\slashed{k}\psi = \slashed{k}^2\psi  = 0$$ 
故$\lambda = 0$。
所以$\slashed{k}$的本征矢$\psi$满足}
{\scriptsize
\be
\bmat{rrrr}
1 & 0 & -1 & 0 \\
0 & 1 & 0 & 1 \\
1 & 0 & -1 & 0 \\
0 & -1 & 0 & -1 
\emat
\bmat{c}
\psi_1\\
\psi_2\\
\psi_3\\
\psi_4
\emat = 0
\ee
}{\small
显然它只有两个线性独立的解分别对应于$\psi_1 = \psi_3$和$\psi_2 = -\psi_4$。
}
\ech
\end{frame}

\begin{frame}
\chtitle{第5题解答}
\bch
{\small
或者更清晰的说法是, 上面的方程等价于$\psi_1 = \psi_3$, $\psi_2 = -\psi_4$,即解可以写成
\be
\psi = \psi_1 \bmat{c}
1 \\
0 \\
1 \\
0
\emat  + 
\psi_2 \bmat{c}
0 \\
1 \\
0 \\
-1
\emat
\ee
也就是说只有两个线性独立的本征矢。
}
\ech
\end{frame}

\end{document}
