\documentclass[CJK]{beamer}
\usepackage{CJKutf8}
\usepackage{beamerthemesplit}
\usetheme{Malmoe}
\useoutertheme[footline=authortitle]{miniframes}
\usepackage{amsmath}
\usepackage{amssymb}
\usepackage{graphicx}
\usepackage{color}
\graphicspath{{figures/}}
\def \bch {\begin{CJK}{UTF8}{gbsn}}
\def \ech {\end{CJK}}
\def \bex {\begin{minipage}{0.3\textwidth}\includegraphics[width=1in]{jugelizi.png}\end{minipage}\begin{minipage}{0.6\textwidth}}
\def \eex {\end{minipage}}
\def \chtitle#1 {\frametitle{\bch #1 \ech}}
\def \skipline { {\vskip 0.1in}}
\def \langr {\mathcal{L}}
\def \hamil {\mathcal{H}}
\def \vecx {\mathbf{x}}
\def \veck {\mathbf{k}}
\def \vecp {\mathbf{p}}
\def \hatphi {\hat{\phi}}
\def \hatq {\hat{q}}
\def \hatpi  {\hat{\pi}}
\def \vel {\upsilon}
\def \Dint {\mathcal{D}}
\def \adag {{\hat{a}^\dagger}}
\def \hata {\hat{a}}
\def \hatN {\hat{N}}
\def \hatH {\hat{H}}
\def \nket { {| n \rangle}}
\def \bran { {\langle n |}}

\title{Quantum Field Theory I \\ Lesson 19 - QED techniques II}
\author{}
\date{}


\begin{document}

\begin{frame}
 
\begin{center}
\begin{Large}
\bch
量子场论 I 

{\vskip 0.3in}

第十九课 QED计算技巧(二)

\ech
\end{Large}
\end{center}

\vskip 0.2in

\bch
课件下载
\ech
https://github.com/zqhuang/SYSU\_QFTI

\end{frame}



\begin{frame}
\chtitle{$\gamma$矩阵的迹}
\bch
{\small
这节课我们尝试来完成Compton散射的散射振幅的计算,为此我们先回顾下之前学过的有用的结论。

\begin{itemize}
\item{奇数个不包含$\gamma^5$的$\gamma$矩阵的乘积的迹为零}
\item{$$\trof{\slashed{u}_1\slashed{u_2}} = 4u_1u_2$$ }
\item{
$$\trof{\slashed{u}_1\slashed{u}_2\slashed{u}_3\slashed{u}_4} = 4(u_1u_2)(u_3u_4) - 4(u_1u_3)(u_2u_4)+4(u_1u_4)(u_2u_3) $$
}
\end{itemize}
}
\ech
\end{frame}


\begin{frame}
\chtitle{带形式下标的$\gamma$矩阵的性质}
\bch
{\small
默认重复指标求和:
$$ \gamma^\mu\gamma_\mu = 4 $$
$$ \gamma^\mu \gamma^\nu \gamma_\mu = -2\gamma^\nu $$
$$ \gamma^\mu \gamma^\alpha\gamma^\beta\gamma_\mu = 4g^{\alpha\beta} $$
$$ \gamma^\mu \gamma^\alpha\gamma^\beta\gamma^\rho\gamma_\mu = -2\gamma^\rho\gamma^\beta\gamma^\alpha $$

对只含$\gamma$矩阵乘积的等式,只要保持乘积的顺序不变,张量的指标升降规则仍然适用。例如
$$\{\gamma_\mu, \gamma_\nu\} = 2 g_{\mu\nu}$$
$$ \gamma_\mu \gamma_\nu \gamma^\mu = -2\gamma_\nu $$
$$ \gamma_\mu \gamma^\alpha\gamma_\beta \gamma^\mu = 4\delta^\alpha_\beta $$

}
\ech
\end{frame}

\begin{frame}
\chtitle{Dirac费米子的自旋求和}
\bch
$$\sum_{\rm spins} u_p \bar{u}_p = \frac{\slp + m}{2\omega}$$
$$\sum_{\rm spins} \upsilon_{-p} \bar{\upsilon}_{-p} = \frac{\slp - m}{2\omega}$$
\ech
\end{frame}

\begin{frame}
\chtitle{Dirac费米子的四维动量的性质}
\bch
{\small
设Dirac费米子的质量为$m$,四维动量为$p$,则有
$$p^2 = m^2$$
由此可推出
$$\slp(\slp + m) = m(\slp+m)$$
$$\slp(\slp - m) = -m(\slp-m)$$
$$(\slp+m)\gamma^\mu(\slp+m) = 2p^\mu (\slp+m)$$
$$(\slp-m)\gamma^\mu(\slp-m) = 2p^\mu (\slp-m)$$
}
\ech
\end{frame}

\begin{frame}
\chtitle{光子的四维动量的性质}
\bch
{\small
设光子四维动量为$k$,则有
$$k^2 = 0$$
由此可推出
$$\slk^2 = 0$$
$$\slk\gamma^\mu\slk = 2p^\mu \slk$$
}
\ech
\end{frame}



\begin{frame}
\chtitle{Compton散射}
\bch
{\small 我们把入射电子和光子的四维动量记为$p$, $k$,出射电子和光子的四维动量记为$p'$, $k'$}
{\tiny
\bea
 |\calM|^2&=& \frac{q^4}{16 \omega_\gamma\omega_\gamma'}\sum_{\rm spins}(\vece_{k'})_\mu (\vece^*_{k})_\nu (\vece^*_{k'})_\alpha (\vece_{k})_\beta  \newl
&&\times  \left(\bar{u}_{p'} \left(\gamma^\mu\frac{\ii}{\slp+\slk-m}\gamma^\nu+\gamma^\nu\frac{\ii}{\slp-\slk'-m}\gamma^\mu\right)  u_{p}  \right)^\dagger \newl
&& \times\left(\bar{u}_{p'} \left(\gamma^\alpha \frac{\ii}{\slp+\slk-m}\gamma^\beta +\gamma^\beta \frac{\ii}{\slp-\slk'-m}\gamma^\alpha \right)u_{p}  \right) \newl
&=& \frac{q^4}{16 \omega_\gamma\omega_\gamma'} \sum_{\rm spins} (-g_{\alpha\mu})(- g_{\nu\beta} )\left(\bar{u}_{p'} \left(\gamma^\mu\frac{\ii}{\slp+\slk-m}\gamma^\nu+\gamma^\nu\frac{\ii}{\slp-\slk'-m}\gamma^\mu\right)  u_{p}  \right)^\dagger \newl
&& \times\left(\bar{u}_{p'} \left(\gamma^\alpha \frac{\ii}{\slp+\slk-m}\gamma^\beta +\gamma^\beta \frac{\ii}{\slp-\slk'-m}\gamma^\alpha \right)u_{p}  \right) 
\eea
}
\ech
\end{frame}

\begin{frame}
\chtitle{Compton散射}
\bch
利用$(\gamma^\mu)^\dagger = \gamma^0\gamma^\mu\gamma^0$,以及$1\times 1$矩阵的迹等于自身,上式可写成
{\tiny
\bea
|\calM|^2 &=& \frac{q^4}{16 \omega_\gamma\omega_\gamma'} \sum_{\rm spins} \bar{u}_{p}\left(\gamma^\nu\frac{1}{\slp+\slk-m}\gamma^\mu+\gamma^\mu\frac{1}{\slp-\slk'-m}\gamma^\nu \right)u_{p'}    \newl
&& \times \bar{u}_{p'} \left(\gamma_\mu \frac{1}{\slp+\slk-m}\gamma_\nu+\gamma_\nu \frac{1}{\slp-\slk'-m}\gamma_\mu \right)u_{p}  \newl
&=& \frac{q^4}{16 \omega_\gamma\omega_\gamma'} \sum_{\rm spins} \mathrm{Tr}\left[ \bar{u}_{p}\left(\gamma^\nu\frac{1}{\slp+\slk-m}\gamma^\mu+\gamma^\mu\frac{1}{\slp-\slk'-m}\gamma^\nu\right) u_{p'}     \right.\newl
&& \times \left. \bar{u}_{p'} \left(\gamma_\mu \frac{1}{\slp+\slk-m}\gamma_\nu+\gamma_\nu \frac{1}{\slp-\slk'-m}\gamma_\mu \right)u_{p} \right] \newl
&=& \frac{q^4}{16 \omega_\gamma\omega_\gamma'} \sum_{\rm spins} \mathrm{Tr}\left[ u_{p}\bar{u}_{p}\left(\gamma^\nu\frac{1}{\slp+\slk-m}\gamma^\mu+\gamma^\mu\frac{1}{\slp-\slk'-m}\gamma^\nu\right) u_{p'} \bar{u}_{p'}    \right.\newl
&& \times \left.  \left(\gamma_\mu \frac{1}{\slp+\slk-m}\gamma_\nu+\gamma_\nu \frac{1}{\slp-\slk'-m}\gamma_\mu \right)\right]
\eea
}

\ech
\end{frame}


\begin{frame}
\chtitle{Compton散射}
\bch
利用$u\bar{u}$的自旋求和性质,得到
{\scriptsize 
$$|\calM|^2 = \frac{q^4}{64 \omega_\gamma\omega_\gamma'\omega_e\omega_e'} K$$}
其中
{\tiny
\bea
K &=&  \mathrm{Tr}\left[ (\slp+m)\left(\gamma^\nu\frac{1}{\slp+\slk-m}\gamma^\mu+\gamma^\mu\frac{1}{\slp-\slk'-m}\gamma^\nu\right) (\slp'+m)    \right.\newl
&& \times \left.  \left(\gamma_\mu \frac{1}{\slp+\slk-m}\gamma_\nu+\gamma_\nu \frac{1}{\slp-\slk'-m}\gamma_\mu \right)\right]
\eea
}

\ech
\end{frame}

\begin{frame}
\chtitle{Compton散射}
\bch
{\small
利用$\slashed{A}^2 = A^2$, 以及$p^2 = p'^2 = m^2$, $k^2 = k'^2 = 0$, Feynman符号在分母的项可以化简为
$$\frac{1}{\slp+\slk-m} = \frac{\slp+\slk+m}{(p+k)^2-m^2} =  \frac{\slp+\slk+m}{2pk} $$
$$\frac{1}{\slp-\slk'-m} = \frac{\slp-\slk'+m}{(p-k')^2-m^2} = - \frac{\slp-\slk'+m}{2pk'} $$
得到
{\scriptsize
\bea
K &=&  \mathrm{Tr}\left[(\slp+m) \left(\gamma^\nu\frac{\slp+\slk+m}{2pk}\gamma^\mu-\gamma^\mu\frac{\slp-\slk'+m}{2p k'}\gamma^\nu\right) (\slp'+m)    \right.\newl
&& \times \left.  \left(\gamma_\mu \frac{\slp+\slk+m}{2p k}\gamma_\nu-\gamma_\nu \frac{\slp-\slk'+m}{2pk'}\gamma_\mu \right)\right]
\eea
}
}
\ech
\end{frame}


\begin{frame}
\chtitle{Compton散射}
\bch
{\small
然后利用Dirac费米子的四维动量的性质,得到
{\scriptsize
\bea
K &=&  \mathrm{Tr}\left[(\slp+m) \left(\frac{\gamma^\nu\slk\gamma^\mu+2\gamma^\mu p^\nu}{2pk}+\frac{\gamma^\mu\slk'\gamma^\nu - 2\gamma^\nu p^\mu}{2p k'}\right) (\slp'+m)    \right.\newl
&& \times \left.  \left(\gamma_\mu \frac{\slp+\slk+m}{2p k}\gamma_\nu-\gamma_\nu \frac{\slp-\slk'+m}{2pk'}\gamma_\mu \right)\right]
\eea
把$(\slp+m)$轮换到最后,再次利用Dirac费米子的四维动量的性质
\bea
K &=&  \mathrm{Tr}\left[\left(\frac{\gamma^\nu\slk\gamma^\mu+2\gamma^\mu p^\nu}{2pk}+\frac{\gamma^\mu\slk'\gamma^\nu - 2\gamma^\nu p^\mu}{2p k'}\right) (\slp'+m)    \right.\newl
&& \times \left.  \left( \frac{\gamma_\mu\slk\gamma_\nu+2\gamma_\mu p_\nu}{2p k}+ \frac{\gamma_\nu\slk'\gamma_\mu - 2 \gamma_\nu p_\mu}{2pk'} \right)(\slp+m) \right]
\eea
}
}
\ech
\end{frame}


\begin{frame}
\chtitle{Compton散射}
\bch
把$K$展成四项:$K = \frac{K_1}{4(pk)^2} + \frac{K_2}{4(pk)(pk')}+ \frac{K_3}{4(pk)(pk')}+ \frac{K_4}{4(pk')^2}$
其中
{\scriptsize
\bea
K_1 &=& \mathrm{Tr}\left[(\gamma^\nu\slk\gamma^\mu+2\gamma^\mu p^\nu)(\slp'+m)(\gamma_\mu\slk\gamma_\nu+2\gamma_\mu p_\nu)(\slp+m)\right] \newl
K_2 &=& \mathrm{Tr}\left[(\gamma^\nu\slk\gamma^\mu+2\gamma^\mu p^\nu)(\slp'+m)(\gamma_\nu\slk'\gamma_\mu - 2 \gamma_\nu p_\mu)(\slp+m)\right] \newl
K_3 &=& \mathrm{Tr}\left[(\gamma^\mu\slk'\gamma^\nu - 2\gamma^\nu p^\mu)(\slp'+m)(\gamma_\mu\slk\gamma_\nu+2\gamma_\mu p_\nu)(\slp+m)\right] \newl
K_4 &=& \mathrm{Tr}\left[(\gamma^\mu\slk'\gamma^\nu - 2\gamma^\nu p^\mu)(\slp'+m)(\gamma_\nu\slk'\gamma_\mu - 2 \gamma_\nu p_\mu)(\slp+m)\right] 
\eea
}
\ech
\end{frame}


\begin{frame}
\chtitle{计算$K_1$}
\bch
把$K_1$展开,因为是求迹运算,只须保留偶次项。
{\tiny
\bea
K_1 &=&\mathrm{Tr}\left[(\gamma^\nu\slk\gamma^\mu+2\gamma^\mu p^\nu)(\slp'+m)(\gamma_\mu\slk\gamma_\nu+2\gamma_\mu p_\nu)(\slp+m)\right] \newl
&=& \trof{\gamma^\nu\slk\gamma^\mu\slp'\gamma_\mu\slk\gamma_\nu\slp} + m^2\trof{\gamma^\nu\slk\gamma^\mu\gamma_\mu\slk\gamma_\nu}  + 4m^2\trof{\gamma^\mu \slp'\gamma_\mu \slp} + 4m^4\trof{\gamma^\mu \gamma_\mu} \newl
&& + 2\trof{\gamma^\nu\slk\gamma^\mu\slp'\gamma_\mu p_\nu\slp} + 2m^2\trof{\gamma^\nu\slk\gamma^\mu\gamma_\mu p_\nu}  + 2\trof{\gamma^\mu p^\nu\slp'\gamma_\mu\slk\gamma_\nu\slp} + 2m^2 \trof{\gamma^\mu p^\nu\gamma_\mu\slk\gamma_\nu} \newl
&=& -2\trof{\gamma^\nu\slk\slp'\slk\gamma_\nu\slp} -8m^2\trof{\slp' \slp} +  64m^4 \newl
&& -4 \trof{\slp\slk\slp'\slp} + 8m^2\trof{\slp\slk}  -4\trof{ \slp'\slk\slp\slp} + 8m^2 \trof{\slk\slp}\newl
&=& 4\trof{\slk\slp'\slk\slp}-32m^2(pp')+  64m^4 -4m^2\trof{\slk\slp'} + 16m^2\trof{\slp\slk}  -4 m^2 \trof{\slp'\slk} \newl
&=& 32(pk)(p'k)-32m^2(pp')+  64m^4 +64m^2(pk)   -32 m^2(p'k)  \newl
&=& 32(pk)(p'k) + 32m^4 +32m^2(pk)  
\eea
}
\ech
\end{frame}

\begin{frame}
\chtitle{计算$K_4$}
\bch
只要把$k$和$-k'$互换即得到$K_4$
{\scriptsize
\be
K_4 = 32(pk')(p'k') + 32m^4 - 32m^2(pk') 
\ee
}
\ech
\end{frame}


\begin{frame}
\chtitle{计算$K_2$}
\bch
{\tiny
\bea
K_2 &=& \mathrm{Tr}\left[(\gamma^\nu\slk\gamma^\mu+2\gamma^\mu p^\nu)(\slp'+m)(\gamma_\nu\slk'\gamma_\mu - 2 \gamma_\nu p_\mu)(\slp+m)\right] \newl
&=& \trof{\gamma^\nu\slk\gamma^\mu\slp'\gamma_\nu\slk'\gamma_\mu\slp} + m^2 \trof{\gamma^\nu\slk\gamma^\mu\gamma_\nu\slk'\gamma_\mu} -4 \trof{\gamma^\mu p^\nu\slp'\gamma_\nu p_\mu\slp} - 4m^2\trof{\gamma^\mu p^\nu\gamma_\nu p_\mu} \newl
 && -2 \trof{\gamma^\nu\slk\gamma^\mu\slp'\gamma_\nu p_\mu\slp} -2m^2  \trof{\gamma^\nu\slk\gamma^\mu\gamma_\nu p_\mu} + 2\trof{\gamma^\mu p^\nu\slp'\gamma_\nu\slk'\gamma_\mu\slp} +2m^2 \trof{\gamma^\mu p^\nu\gamma_\nu\slk'\gamma_\mu} \newl
&=& -2\trof{ \slp'\gamma^\mu\slk \slk'\gamma_\mu\slp} + 4m^2 \trof{k^\mu\slk'\gamma_\mu} -4 \trof{\slp\slp'\slp \slp} - 4m^2\trof{\slp^2} \newl
 && +4 \trof{\slp'\gamma^\mu\slk p_\mu\slp} -8m^2  \trof{k^\mu p_\mu} + 2\trof{\gamma^\mu \slp'\slp\slk'\gamma_\mu\slp} +8m^2 \trof{ \slp\slk'} \newl
&=& -8(kk')\trof{ \slp'\slp} + 4m^2 \trof{\slk'\slk} -4m^2 \trof{\slp\slp'} - 16m^4 \newl
 && +4 \trof{\slp'\slp\slk\slp} -32m^2(pk) -4\trof{ \slk'\slp\slp'\slp} +32m^2 (pk') \newl
&=& -32(kk')(pp') + 16m^2 (kk') -16m^2(pp') - 16m^4 \newl
 && + 32(pp')(pk) - 16m^2(p'k) -32m^2(pk) -32(pp')(pk') + 16m^2(p'k') +32m^2 (pk') \newl
&=& 32(pp')(pk)-32(kk')(pp') -32(pp')(pk') - 32m^4  -16m^2(pk)  +16m^2 (pk') \newl
&=&   - 32m^4  -16m^2(pk)  +16m^2 (pk')
 \eea
}
\ech
\end{frame}

\begin{frame}
\chtitle{计算$K_3$}
\bch
同样,在$K_2$的结果中把$k$和$-k'$互换即得到$K_3$
{\scriptsize
\be
K_3 =  - 32m^4  +16m^2(pk')  -16m^2 (pk) = K_2
\ee
}
\ech
\end{frame}

\begin{frame}
\chtitle{总和}
\bch
最后总和为
{\scriptsize
\be
K = 8\left[\frac{pk'}{pk} + \frac{pk}{pk'} + 2m^2\left(\frac{1}{pk}-\frac{1}{pk'}\right) + m^4\left(\frac{1}{pk}-\frac{1}{pk'}\right)^2\right]
\ee

即
\be
|\calM|^2 = \frac{q^4}{8\omega_p\omega_p'\omega_k\omega_k'}\left[\frac{pk'}{pk} + \frac{pk}{pk'} + 2m^2\left(\frac{1}{pk}-\frac{1}{pk'}\right) + m^4\left(\frac{1}{pk}-\frac{1}{pk'}\right)^2\right]
\ee

}
\ech
\end{frame}

\end{document}


