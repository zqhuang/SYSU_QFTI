\documentclass[CJK]{beamer}
\usepackage{CJKutf8}
\usepackage{beamerthemesplit}
\usetheme{Malmoe}
\useoutertheme[footline=authortitle]{miniframes}
\usepackage{amsmath}
\usepackage{amssymb}
\usepackage{graphicx}
\usepackage{color}
\graphicspath{{figures/}}
\def \bch {\begin{CJK}{UTF8}{gbsn}}
\def \ech {\end{CJK}}
\def \bex {\begin{minipage}{0.3\textwidth}\includegraphics[width=1in]{jugelizi.png}\end{minipage}\begin{minipage}{0.6\textwidth}}
\def \eex {\end{minipage}}
\def \chtitle#1 {\frametitle{\bch #1 \ech}}
\def \skipline { {\vskip 0.1in}}
\def \langr {\mathcal{L}}
\def \hamil {\mathcal{H}}
\def \vecx {\mathbf{x}}
\def \veck {\mathbf{k}}
\def \vecp {\mathbf{p}}
\def \hatphi {\hat{\phi}}
\def \hatq {\hat{q}}
\def \hatpi  {\hat{\pi}}
\def \vel {\upsilon}
\def \Dint {\mathcal{D}}
\def \adag {{\hat{a}^\dagger}}
\def \hata {\hat{a}}
\def \hatN {\hat{N}}
\def \hatH {\hat{H}}
\def \nket { {| n \rangle}}
\def \bran { {\langle n |}}

\title{Quantum Field Theory I \\ Lesson 11 - Spinor and Free Field Summary}
\author{}
\date{}


\begin{document}

\begin{frame}
 
\begin{center}
\begin{Large}
\bch
量子场论 I 

{\vskip 0.3in}

第十一课 旋量场和自由场总结

\ech
\end{Large}
\end{center}

\vskip 0.2in

\bch
课件下载
\ech
https://github.com/zqhuang/SYSU\_QFTI

\end{frame}


\begin{frame}
\chtitle{进一步讨论$u_{\veck,s}$和$\upsilon_{\veck,s}$的性质}
\bch
我们来回忆下Dirac方程在傅立叶空间的正频解和负频解$u_{\veck,s}e^{-\ii \omega t}$, $\upsilon_{\veck,s}e^{\ii \omega t}$ (这里$\omega = \sqrt{\veck^2+m^2}$, 自旋$s=\pm 1/2$)。
\skipline

延续上两节课的记号,沿$\veck$方向的自旋本征态记为$\zeta_{\veck,s}$,并定义
$\theta_{\veck,s} = 2s \tan^{-1}\frac{|\veck|}{\omega+m}$。我们已经解出了:
\begin{equation}
u_{\veck,s} = 
\bmat{c}
\zeta_{\veck,s} \cos\theta_{\veck,s} \\
\zeta_{\veck,s} \sin\theta_{\veck,s} \\
\emat,\ \ \ 
\upsilon_{\veck,s} = 
\bmat{c}
\zeta_{\veck,s} \sin\theta_{\veck,s} \\
-\zeta_{\veck,s} \cos\theta_{\veck,s} \\
\emat \nonumber
\end{equation}
记四维动量$k = (\omega, \veck)$,证明
\begin{itemize}
\item{$\zeta_{-\veck,s} = \zeta_{\veck,-s}$, $\theta_{-\veck,s} = \theta_{\veck,s}$, $\theta_{\veck,-s} = - \theta_{\veck,s}$ }
\item{利用$u$, $\upsilon$的显式表达式,直接证明$\bar{u}_{\veck,s}\upsilon_{-\veck,s'} = 0$, $\bar{\upsilon}_{-\veck,s} u_{\veck,s'} = 0$}
\end{itemize}
\ech
\end{frame}

\begin{frame}
\chtitle{进一步讨论$u_{\veck,s}$和$\upsilon_{\veck,s}$的性质}
\bch
事实上,对$\bar{u}$和$\upsilon$以及$\bar{\upsilon}$和$u$的正交性,存在更优美的证明方法:
\begin{itemize}
\item{证明$\gamma^0\slashed{k} = \slashed{k}^\dagger \gamma^0$}
\item{证明$\slashed{k}$的本征值为$m$的两个本征态为$u_{\veck,s}$ ($s=\pm 1/2$)。}
\item{证明$\slashed{k}$的本征值为$-m$的两个本征态为$\upsilon_{-\veck,s}$ ($s=\pm 1/2$)。}
\item{利用前三题结论证明:若$m>0$,则$\bar{u}_{\veck,s}\upsilon_{-\veck,s'} = 0$, $\bar{\upsilon}_{-\veck,s} u_{\veck,s'} = 0$}
\end{itemize}
\ech
\end{frame}


\begin{frame}
\chtitle{进一步讨论$u_{\veck,s}$和$\upsilon_{\veck,s}$的性质}
\bch
前两节课里我们还得到过
$$\bar{u}_{\veck,s} u_{\veck,s'} = \frac{m}{\omega} \delta_{ss'},\ \ \bar{\upsilon}_{-\veck,s} \upsilon_{-\veck,s'} = -\frac{m}{\omega} \delta_{ss'}$$
(注意我们已经代入了$\cos{2\theta_{\veck,s}} = m/\omega$)。

\skipline
若$m > 0$,证明$u_{\veck,s},\ \upsilon_{-\veck,s}$ ($s=\pm 1/2$)这四个旋量线性独立,并由此证明
$$\sum_s u_{\veck,s}\bar{u}_{\veck,s} = \frac{\slashed{k}+m}{2\omega}$$
$$\sum_s \upsilon_{-\veck,s}\bar{\upsilon}_{-\veck,s} = \frac{\slashed{k}-m}{2\omega}$$

\ech
\end{frame}


\begin{frame}
\chtitle{一些题外话}
\bch
事实上,我们前面已经提过的正反粒子投影算符:
$$P_+ \equiv \frac{\slashed{k}+m}{2m} = \frac{\omega}{m}\sum_s u_{\veck,s}\bar{u}_{\veck,s} $$
$$P_-\equiv \frac{-\slashed{k}+m}{2m} = -\frac{\omega}{m}\sum_s \upsilon_{-\veck,s}\bar{\upsilon}_{-\veck,s} $$
显然满足
$$P_+\, u_{\veck,s} = u_{\veck,s},\ \ P_+\, \upsilon_{-\veck,s} = 0$$
$$P_-\, u_{\veck,s} = 0,\ \ P_-\, \upsilon_{-\veck,s} = \upsilon_{-\veck,s}$$
\skipline

阅读教材时注意教材上的归一化不同,并且教材上的$\upsilon(\veck,\xi)$对应这里的$\upsilon_{-\veck,s}$
\ech
\end{frame}

\begin{frame}
\chtitle{再总结下我之前在干嘛……}
\bch
前面的结果
$$\sum_s u_{\veck,s}\bar{u}_{\veck,s} = \frac{\slashed{k}+m}{2\omega}$$
$$\sum_s \upsilon_{-\veck,s}\bar{\upsilon}_{-\veck,s} = \frac{\slashed{k}-m}{2\omega}$$
两边乘以$\gamma^0$还能得到
$$\sum_s u_{\veck,s}u^\dagger_{\veck,s} = \frac{(\slashed{k}+m)\gamma^0}{2\omega}$$
$$\sum_s \upsilon_{-\veck,s}\upsilon^\dagger_{-\veck,s} = \frac{(\slashed{k}-m)\gamma^0}{2\omega}$$
这几个等式为我们计算旋量场的反对易算子做好了准备。

\ech
\end{frame}

\begin{frame}
\chtitle{旋量场的实空间表达式}
\bch
利用上节课得到的旋量场的傅立叶空间解:
$$\hat\psi_{\veck}(t) = \frac{1}{\sqrt{d^3\veck}} \sum_{s} \left(u_{\veck,s}e^{-\ii \omega t}\hata_{\veck,s}|_{t=0} + \upsilon_{\veck,s}e^{\ii \omega t}\bdag_{-\veck,s}|_{t=0}\right)$$
若不致引起混淆一般省略不写$|_{t=0}$标记。进行傅立叶反变换:
$$\hat\psi(\vecx,t) = \frac{1}{(2\pi)^{3/2}}\int \sqrt{d^3\veck} \sum_s \left(u_{\veck,s}e^{-\ii(\omega t-\veck\cdot\vecx)} \hata_{\veck,s} + \upsilon_{\veck,s}e^{\ii(\omega t+\veck\cdot\vecx)} \bdag_{-\veck,s}\right)$$
把第二项中的$\veck$换成$-\veck$(因为对所有$\veck$求和所以这样替换是允许的),并写成四维形式:
$$\hat\psi(x) = \frac{1}{(2\pi)^{3/2}}\int \sqrt{d^3\veck} \sum_s \left(u_{\veck,s}e^{-\ii k_\mu x^\mu} \hata_{\veck,s} + \upsilon_{-\veck,s}e^{\ii k_\mu x^\mu} \bdag_{\veck,s}\right)$$
这里我们按通常习惯规定了$k^\mu = (\omega, \veck)$。
\ech
\end{frame}

\begin{frame}
\chtitle{旋量场的反对易}
\bch
由于产生算符互相反对易,湮灭算符互相反对易。而不同自由度的产生或湮灭算符也互相反对易。显然有
$$\{\hat\psi_\alpha(x), \hat\psi_\beta(x')\} = \{\hat\psi^\dagger_\alpha(x), \hat\psi^\dagger_\beta(x')\} = 0$$
时空任意两点的$\psi$和$\psi^\dagger$算符的反对易子为
{\scriptsize
\bea
&& \{ \hat\psi_\alpha(x), \hat\psi^\dagger_\beta(x')\} \newl
&=& \frac{1}{(2\pi)^3} \int d^3\veck \sum_s\left((u_{\veck,s})_\alpha (u^\dagger_{\veck,s})_\beta e^{-\ii k_\mu (x^\mu - x'^\mu)}  + (\upsilon_{-\veck,s})_{\alpha}(\upsilon^\dagger_{-\veck,s})_{\beta} e^{\ii k_\mu (x^\mu -x'^\mu) } \right) \newl
&=& \frac{1}{(2\pi)^3} \int \frac{d^3\veck}{2\omega} \left((\slashed{k}+m)\gamma^0 e^{-\ii k_\mu (x^\mu - x'^\mu)} + (\slashed{k}-m)\gamma^0 e^{\ii k_\mu (x^\mu -x'^\mu) } \right)_{\alpha\beta} 
\eea
}
\ech
\end{frame}

\begin{frame}
\chtitle{旋量场的的反对易}
\bch
如果上式中取$t' = t $,则
{\scriptsize
\bea
&& \{ \hat\psi_\alpha(\vecx,t), \hat\psi^\dagger_\beta(\vecx',t)\} \newl
&=& \frac{1}{(2\pi)^3} \int \frac{d^3\veck}{2\omega}\left((\slashed{k}+m)\gamma^0 e^{\ii \veck\cdot(\vecx-\vecx')} + (\slashed{k}-m)\gamma^0 e^{-\ii \veck\cdot(\vecx-\vecx') } \right)_{\alpha\beta} \newl
&=& \frac{1}{(2\pi)^3} \int \frac{d^3\veck}{2\omega}\left((\omega\gamma^0 - k^j\gamma^j+m)\gamma^0 e^{\ii \veck\cdot(\vecx-\vecx')} + (\omega \gamma^0 + k^j\gamma^j-m)\gamma^0 e^{\ii \veck\cdot(\vecx-\vecx') } \right)_{\alpha\beta} \newl
\eea
}
因为是对全空间积分,在上面第二项中我们做了$\veck \rightarrow -\veck$的替换。再利用第三课末尾介绍的积分公式,我们得到
{\scriptsize
\be
\{ \hat\psi_\alpha(\vecx,t), \hat\psi^\dagger_\beta(\vecx',t)\} = \frac{1}{(2\pi)^3} \int d^3\veck \left(e^{\ii \veck\cdot(\vecx-\vecx')} \right)_{\alpha\beta} = \delta(\vecx-\vecx')\delta_{\alpha\beta}
\ee
}

\ech
\end{frame}

\begin{frame}
\chtitle{旋量场的反对易}
\bch
现在我们考虑$\psi$和$\bar{\psi}$的反对易:

{\scriptsize
\bea
&& \{ \hat\psi_\alpha(x), \hat{\bar{\psi}}_\beta(x')\} \newl
&=& \frac{1}{(2\pi)^3} \int d^3\veck \sum_s\left((u_{\veck,s})_\alpha (\bar{u}_{\veck,s})_\beta e^{-\ii k_\mu (x^\mu - x'^\mu)}  + (\upsilon_{-\veck,s})_{\alpha}(\bar{\upsilon}_{-\veck,s})_{\beta} e^{\ii k_\mu (x^\mu -x'^\mu) } \right) \newl
&=& \frac{1}{(2\pi)^3} \int \frac{d^3\veck}{2\omega} \left((\slashed{k}+m) e^{-\ii k_\mu (x^\mu - x'^\mu)} + (\slashed{k}-m) e^{\ii k_\mu (x^\mu -x'^\mu) } \right)_{\alpha\beta} \newl
&=& (i\slashed{\partial}+m)_{\alpha\beta}\,\frac{1}{(2\pi)^3} \int \frac{d^3\veck}{2\omega} \left( e^{-\ii k_\mu (x^\mu - x'^\mu)} - e^{\ii k_\mu (x^\mu -x'^\mu) } \right)
\eea
}

是不是有些眼熟?请和之前作业里推导过的标量场的两点对易函数比较。

\ech
\end{frame}



\begin{frame}
\chtitle{一些容易混淆的符号的澄清}
\bch
$|n\rangle$符号的滥用:
\begin{itemize}
\item{上节课我们讲到对空穴数算符的本征态$|n\rangle$,$b^\dagger|n\rangle \propto |n-1\rangle$,$b|n\rangle \propto |n+1\rangle$或者为零。注意不要把空穴数算符的本征态$|n\rangle$和粒子数算符的本征态$|n\rangle$混淆。“空穴数本征态”只是我们逻辑演绎过程中的一个中间产物,今后我们只会讨论粒子数算符$\bdag\hatb$的本征态。对粒子数算符的本征态$|n\rangle$,$b^\dagger |n\rangle \propto |n+1\rangle$或者为零,$b|n\rangle =\sqrt{n} |n-1\rangle$。}
\item{同样,不要把不同自由度的粒子数算符的本征态$|n\rangle$混为一谈。}
\end{itemize}
\ech
\end{frame}

\begin{frame}
\chtitle{一些容易混淆的符号的澄清}
\bch
$k$符号的滥用:

\begin{itemize}
\item{在推导 Dirac方程的一般解时我们允许了$k_0 = \pm \omega$的数学解,这只是数学推导过程,不代表我们物理上允许$k_0 = -\omega$的四维动量存在。在其他不作特殊说明的情况下,我们写四维动量$k$时总默认$k_0=\omega$。}
\end{itemize}

\ech
\end{frame}

\begin{frame}
\chtitle{自由场总结:综述}
\bch
迄今为止,我们把实标量场,复标量场,矢量场,和旋量场进行了量子化(写成了一堆产生算符和湮灭算符的和)。最核心的内容是
\begin{itemize}
\item{场的量子化的最后结果}
\item{谐振子的产生湮灭算符的性质}
\item{自旋为1/2的费米子的产生湮灭算符的性质。}
\end{itemize}
这些将成为我们将来计算散射振幅的核心工具。
\ech
\end{frame}

\begin{frame}
\chtitle{自由场总结:实标量场的量子化结果}
\bch

傅立叶空间:
$$\hatphi(\veck) = \frac{\hata_{\veck}+\adag_{-\veck}}{\sqrt{2\omega\, d^3\veck}}$$


\skipline
实空间:
$$\hatphi(x) = \frac{1}{(2\pi)^{3/2}} \int \sqrt{\frac{d^3\veck}{2\omega}} \left(\hata_{\veck} e^{-ik_\mu x^\mu} + \adag_{\veck}e^{ik_\mu x^\mu}\right) $$

\skipline
{\small
注意:傅立叶空间表达式中算符 $\hata$, $\adag$均为任意$t$时刻算符(自带了$e^{\mp\ii\omega t}$的因子)。实空间表达式中算符$\hata$, $\adag$为$t=0$时刻算符,因子$e^{\mp\ii\omega t}$则放到了$e^{\mp \ii k_\mu x^\mu}$中。
}
\ech
\end{frame}

\begin{frame}
\chtitle{自由场总结: 复标量场的量子化结果}
\bch
傅立叶空间:
$$\hatphi(\veck) = \frac{\hata_{\veck}+\bdag_{-\veck}}{\sqrt{2\omega\, d^3\veck}}$$


\skipline
实空间:
$$\hatphi(x) = \frac{1}{(2\pi)^{3/2}} \int \sqrt{\frac{d^3\veck}{2\omega}} \left(\hata_{\veck} e^{-ik_\mu x^\mu} + \bdag_{\veck}e^{ik_\mu x^\mu}\right) $$
\ech
\end{frame}

\begin{frame}
\chtitle{自由场总结:$U(1)$规范场(零质量矢量场)量子化最后结果}
\bch

傅立叶空间:
$$ \hat{\vecA}_{\veck} = \sum_{s=\pm 1} \vece_{\veck,s}\left(\frac{\hata_{\veck,s} + \adag_{-\veck,s}}{\sqrt{2 |\veck| d^3\veck}}\right)$$。


\skipline
实空间:
$$\hat{\vecA}(x) = \frac{1}{(2\pi)^{3/2}} \int \sqrt{\frac{d^3\veck}{2|\veck|}} \sum_{s=\pm 1}\left(\hata_{\veck s} \mathbf{e}_{\veck s} e^{-ik_\mu x^\mu} + \adag_{\veck s} \mathbf{e}^*_{\veck s} e^{ik_\mu x^\mu}\right) $$
\ech
\end{frame}

\begin{frame}
\chtitle{自由场总结:旋量场量子化最后结果}
\bch
傅立叶空间:
$$\hat\psi_{\veck}(t) = \sum_{s=\pm 1/2} \frac{u_{\veck,s}\hata_{\veck,s} + \upsilon_{\veck,s}\bdag_{-\veck,s}}{\sqrt{d^3\veck}} $$

实空间:
$$\hat\psi(x) = \frac{1}{(2\pi)^{3/2}}\int \sqrt{d^3\veck} \sum_{s=\pm 1/2} \left(u_{\veck,s}e^{-\ii k_\mu x^\mu} \hata_{\veck,s} + \upsilon_{-\veck,s}e^{\ii k_\mu x^\mu} \bdag_{\veck,s}\right)$$
\ech
\end{frame}

\begin{frame}
\chtitle{自由场总结:玻色子产生湮灭算符的性质}
\bch
自旋为整数的玻色子:产生算符和产生算符总是对易。湮灭算符和湮灭算符总是对易。不同自由度(例如对应不同$\veck$或者不同自旋$s$)的任何两个产生或者湮灭算符对易。

同一自由度
$$[\hata, \adag] = 1$$

\skipline
粒子数算符$\hat{N}=\adag\hata$存在且仅存在本征值为任意非负整数$n$的本征态$|n\rangle$
$$\hata|n\rangle = \sqrt{n}|n-1\rangle $$
$$\adag|n\rangle = \sqrt{n+1}|n+1\rangle$$

\ech
\end{frame}

\begin{frame}
\chtitle{自由场总结:费米子产生湮灭算符的性质}
\bch
自旋为半整数的费米子:产生算符和产生算符总是反对易。湮灭算符和湮灭算符总是反对易。不同自由度(例如对应不同$\veck$或者不同自旋$s$)的任何两个产生或者湮灭算符反对易。

同一自由度
$$\{\hata, \adag\} = 1$$

\skipline
粒子数算符$\hat{N}=\adag\hata$存在且仅存在本征值$n=0,1$的本征态$|n\rangle$
$$\hata|0 \rangle = 0,\ \hata|1\rangle = |0\rangle $$
$$\adag|0 \rangle = |1\rangle, \  \adag|1\rangle = 0$$
 
\ech
\end{frame}

\begin{frame}
\chtitle{自由场总结:其他零星知识}
\bch
其它比较重要的知识有:
\begin{itemize}
\item{张量的写法和运用,度规和指标升降。}
\item{自然单位制,量纲分析。}
\item{经典作用量原理和Euler-Lagrange方程}
\item{Noether定理的推导和运用。}
\item{实空间的二次项的积分等于傅立叶空间的同样二次项的积分(这是自由场理论的数学根基)以及数学公式$\frac{1}{(2\pi)^3}\int d^3\vecx\,e^{i\veck\cdot\vecx} = \delta(\veck)$}
\item{$\gamma$矩阵的基本性质。旋量变换矩阵$\Lambda$的定义和性质。}
\item{Dirac方程经典解$u$和$\upsilon$的性质}
\end{itemize}
\ech
\end{frame}

\begin{frame}
\chtitle{讨论:关于11月10号换课的安排}
\bch
11月10号那节课我会在昆明开会,请选择
\begin{itemize}
\item[A]{一节破课而已,别上了。}
\item[B]{一节破课而已,别上了。}
\item[C]{一节破课而已,别上了。}
\end{itemize}
\ech
\end{frame}

\begin{frame}
\chtitle{讨论:关于11月10号换课的安排}
\bch
噢弄错了,请重新选择
\begin{itemize}
\item[A]{一节破课而已,别上了。}
\item[B]{换课到周六白天}
\item[C]{布置一个脑洞大开的作业,你们自己讨论。该作业共5题计10分(其他8次正常作业仍计20分),不限讨论不限搜索不限抄书不限请教他人,写下你认为最好的解,期末考试前一周交。}
\end{itemize}
\ech
\end{frame}

\end{document}


