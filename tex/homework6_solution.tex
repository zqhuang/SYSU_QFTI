\documentclass[CJK]{beamer}
\usepackage{CJKutf8}
\usepackage{beamerthemesplit}
\usetheme{Malmoe}
\useoutertheme[footline=authortitle]{miniframes}
\usepackage{amsmath}
\usepackage{amssymb}
\usepackage{graphicx}
\usepackage{color}
\graphicspath{{figures/}}
\def \bch {\begin{CJK}{UTF8}{gbsn}}
\def \ech {\end{CJK}}
\def \bex {\begin{minipage}{0.3\textwidth}\includegraphics[width=1in]{jugelizi.png}\end{minipage}\begin{minipage}{0.6\textwidth}}
\def \eex {\end{minipage}}
\def \chtitle#1 {\frametitle{\bch #1 \ech}}
\def \skipline { {\vskip 0.1in}}
\def \langr {\mathcal{L}}
\def \hamil {\mathcal{H}}
\def \vecx {\mathbf{x}}
\def \veck {\mathbf{k}}
\def \vecp {\mathbf{p}}
\def \hatphi {\hat{\phi}}
\def \hatq {\hat{q}}
\def \hatpi  {\hat{\pi}}
\def \vel {\upsilon}
\def \Dint {\mathcal{D}}
\def \adag {{\hat{a}^\dagger}}
\def \hata {\hat{a}}
\def \hatN {\hat{N}}
\def \hatH {\hat{H}}
\def \nket { {| n \rangle}}
\def \bran { {\langle n |}}

\title{Quantum Field Theory I \\ Homework 6 solution}
  \author{}
  \date{}


\begin{document}

\begin{frame}
 
\begin{center}
\begin{Large}
\bch
量子场论 I 

{\vskip 0.3in}

第六次课后作业参考答案
\skipline
\skipline

如发现参考答案有错误请不吝告知(微信zhiqihuang或邮箱huangzhq25@sysu.edu.cn)
\ech
\end{Large}
\end{center}

\vskip 0.2in

\bch
课件下载
\ech
https://github.com/zqhuang/SYSU\_QFTI

\end{frame}


\begin{frame}
\chtitle{第1题 题目和思路}
\bch
{\small
证明下述矩阵的迹为零:
\begin{itemize}
\item{$\slashed{a}\slashed{b}\gamma^0$}
\item{$\slashed{a}\slashed{b}\gamma^5$}
\item{$\slashed{a}\slashed{b}\slashed{c}$}
\item{$\slashed{a}\slashed{b}\slashed{c}\gamma^5$}
\item{$\slashed{a}\slashed{b}\slashed{c}\slashed{d}-\slashed{d}\slashed{c}\slashed{b}\slashed{a}$}
\end{itemize}

\skipline
思路:用$\gamma$矩阵的零迹定理和Feynman符号迹的倒排定理
}


\ech
\end{frame}

\begin{frame}
\chtitle{第1题解答}
\bch
{\small
\begin{itemize}
\item{$\slashed{a}\slashed{b}\gamma^0$展开式的每一项都是$\gamma^\mu\gamma^\nu\gamma^0$的形式($0\le \mu,\nu\le 3$)。根据零迹定理(排除$\gamma^5$),$\gamma^\mu\gamma^\nu\gamma^0$的迹为零。}
\item{$\slashed{a}\slashed{b}\gamma^5$的展开式每一项都是$\gamma^\mu\gamma^\nu\gamma^5$的形式($0\le \mu,\nu\le 3$)。 根据零迹定理(取某个$\alpha \ne \mu,\nu, 5$, 排除$\gamma^\alpha$) $\gamma^\mu\gamma^\nu\gamma^5$的迹为零}
\item{$\slashed{a}\slashed{b}\slashed{c}$的展开式每一项都是$\gamma^\mu\gamma^\nu\gamma^\lambda$的形式($0\le \mu,\nu,\lambda\le 3$)。根据零迹定理(排除$\gamma^5$),$\gamma^\mu\gamma^\nu\gamma^\lambda$的迹为零。}
\item{$\slashed{a}\slashed{b}\slashed{c}\gamma^5$展开式的每一项是$\gamma^\mu\gamma^\nu\gamma^\lambda\gamma^5$的形式($0\le \mu,\nu, \lambda \le 3$)。根据零迹定理(排除$\gamma^5$),$\gamma^\mu\gamma^\nu\gamma^\lambda\gamma^5$的迹为零。}
\item{根据Feynman符号的迹的倒排定理,$\slashed{a}\slashed{b}\slashed{c}\slashed{d}-\slashed{d}\slashed{c}\slashed{b}\slashed{a}$的迹为零。}
\end{itemize}

}


\ech
\end{frame}

\begin{frame}
\chtitle{第2题 题目和思路}
\bch
证明下述恒等式:
\begin{itemize}
\item{$\slashed{a}\slashed{b} + \slashed{b}\slashed{a} = 2 ab$}
\item{$\trof{\slashed{a}\gamma^\mu} = 4 a^\mu$}
\end{itemize}

\skipline
思路:利用Feynman符号和$\gamma$矩阵的定义的定义即可证明
\ech
\end{frame}

\begin{frame}
\chtitle{第2题解答}
\bch
\begin{itemize}
\item{$\slashed{a}\slashed{b} + \slashed{b}\slashed{a} = a_\mu b_\nu (\gamma^\mu\gamma^\nu+\gamma^\nu\gamma^\mu) = 2 g^{\mu\nu}a_\mu b_\nu = 2 ab$}
\item{利用迹可以轮换乘积次序的性质,$\trof{\gamma^\nu\gamma^\mu} =\trof{\gamma^\mu\gamma^\nu} =\frac{1}{2}\trof{\gamma^\mu\gamma^\nu + \gamma^\nu\gamma^\mu} = g^{\mu\nu}\trof{I_{4\times 4}} = 4 g^{\mu\nu}$

所以 $\trof{\slashed{a}\gamma^\mu} = a_\nu \trof{\gamma^\nu\gamma^\mu} =  4 a_\nu g^{\mu\nu} = 4 a^\mu$}
\end{itemize}

\ech
\end{frame}

\begin{frame}
\chtitle{第3题 题目和思路}
\bch 
把
$$\trof{\slashed{a}\gamma^\mu\slashed{b}\slashed{c}\slashed{d}\gamma_\mu}$$
化简为只含矢量内积的最简形式。

\skipline
思路:利用$\gamma$矩阵形式下标矩阵的性质和Feynman符号迹的展开定理
\ech
\end{frame}

\begin{frame}
\chtitle{第3题 第一种解答}
\bch
利用$\gamma^\mu\gamma^\alpha\gamma^\beta\gamma^\lambda\gamma_\mu = -2 \gamma^\lambda\gamma^\beta\gamma^\alpha$即可得到$\trof{\gamma^\mu\slashed{b}\slashed{c}\slashed{d}\gamma_\mu} = -2\slashed{d}\slashed{c}\slashed{b}$,所以
$$\trof{\slashed{a}\gamma^\mu\slashed{b}\slashed{c}\slashed{d}\gamma_\mu} = -2\trof{\slashed{a}\slashed{d}\slashed{c}\slashed{b}} = -8 (ad)(bc)   +8 (ac)(bd) -8 (ab)(cd) $$
\ech
\end{frame}


\begin{frame}
\chtitle{第3题 第二种解答}
\bch
利用$\gamma_\mu\gamma^\alpha \gamma^\mu = -2 \gamma^\alpha$即可得到$\trof{\gamma^\mu\slashed{a}\gamma_\mu} = -2\slashed{a}$,所以
\bea
\trof{\slashed{a}\gamma^\mu\slashed{b}\slashed{c}\slashed{d}\gamma_\mu} &=& \trof{\gamma_\mu\slashed{a}\gamma^\mu\slashed{b}\slashed{c}\slashed{d}} \newl
&=& -2 \trof{\slashed{a}\slashed{b}\slashed{c}\slashed{d}} \newl
&=& -8(ab)(cd)   + 8(ac)(bd) -8(ad)(bc)
\eea
\ech
\end{frame}


\begin{frame}
\chtitle{第4题 题目和思路}
\bch
设$p$为电子的四维动量,$k$为光子的四维动量,化简
$$\trof{ (\slashed{p}+m)\gamma^\mu\frac{1}{\slashed{p}+\slashed{k}-m}\gamma_\mu}$$

\skipline
思路:利用第19课所讲的技巧
 
\ech
\end{frame}

\begin{frame}
\chtitle{第4题 第一种解答}
\bch
{\scriptsize
\bea
 \trof{ (\slashed{p}+m)\gamma^\mu\frac{1}{\slashed{p}+\slashed{k}-m}\gamma_\mu} &= &\trof{ (\slashed{p}+m)\gamma^\mu\frac{\slashed{p}+\slashed{k}+m}{(p+k)^2-m}\gamma_\mu} \newl
&=& \frac{1}{2pk}\trof{ (\slashed{p}+m)\gamma^\mu(\slashed{p}+\slashed{k}+m)\gamma_\mu} 
\eea
利用$\trof{(\slp+m)\gamma^\mu(\slp+m) } = 2p^\mu(\slp+m)$,上式等于
\bea
&& \frac{1}{2pk}\trof{ (\slashed{p}+m)(2p^\mu\gamma_\mu + \gamma^\mu\slashed{k}\gamma_\mu)} \newl
&=& \frac{1}{2pk}\trof{ (\slashed{p}+m)(2\slashed{p} - 2\slashed{k})} \newl
&=& \frac{1}{pk}\trof{\slashed{p}(\slashed{p} - \slashed{k}) } \newl
&=& \frac{4m^2}{pk} - 4
\eea
在最后一步我们利用了$\trof{\slp^2} = 4 p^2 = 4 m^2$和$\trof{\slp\slk}= 4pk$
 }
\ech
\end{frame}


\begin{frame}
\chtitle{第4题 第二种解答}
\bch
{\scriptsize
\be
 \trof{ (\slashed{p}+m)\gamma^\mu\frac{1}{\slashed{p}+\slashed{k}-m}\gamma_\mu} = \trof{\gamma_\mu (\slashed{p}+m)\gamma^\mu\frac{1}{\slashed{p}+\slashed{k}-m}} 
\ee
利用$\gamma_\mu \slp\gamma^\mu = -2\slp$和$\gamma_\mu\gamma^\mu = 4$,上式等于
\bea
\trof{ (-2\slashed{p}+4m)\frac{1}{\slp+\slk-m}} &=&\trof{ (-2\slashed{p}+4m)\frac{\slashed{p}+\slashed{k}+m}{(p+k)^2-m}} \newl
&=& \frac{1}{pk}\trof{ (-\slashed{p}+2m)(\slashed{p}+\slashed{k}+m)} 
\eea
展开上式,因为奇数个feynman符号的乘积的迹为零,只要保留偶次项。上式等于
\be
\frac{1}{pk}\trof{-\slashed{p}(\slashed{p}+\slashed{k}) + 2m^2} = \frac{1}{pk}\trof{m^2-\slashed{p}\slashed{k}) } = \frac{4m^2}{pk} - 4
\ee
 }
\ech
\end{frame}


\begin{frame}
\chtitle{第5题 题目和思路}
\bch
在Compton散射过程中,取电子初始状态为静止,入射光子沿$z$轴方向的参照系。限定入射光子四维动量为$k^\mu=(\omega, 0, 0, \omega)$;出射光子能量为$\omega'$,方向限定在某固定方向$\vecn$附近的立体角$d\Omega$内;$\vecn$与$z$轴的夹角为$\theta$。当$d\Omega$很小时,散射截面与$d\Omega$成正比,它们之间的比称为微分散射截面:$\frac{d\sigma}{d\Omega}$。试根据课上所求的散射概率计算该微分散射截面(写成$\omega$, $\omega'$,和$\theta$的函数)。

\skipline
思路:在限定范围内对末态进行积分
\ech
\end{frame}

\begin{frame}
\chtitle{第5题 解答}
\bch
{\scriptsize

在课上的结果
\be
|\calM|^2 = \frac{q^4}{8\omega_p\omega_p'\omega_k\omega_k'}\left[\frac{pk'}{pk} + \frac{pk}{pk'} + 2m^2\left(\frac{1}{pk}-\frac{1}{pk'}\right) + m^4\left(\frac{1}{pk}-\frac{1}{pk'}\right)^2\right]
\ee
中代入$pk = m \omega$, $pk' = m \omega'$,得到
$$|\calM|^2 = \frac{q^4}{8 m (m+\omega-\omega')\omega\omega'}\left[\frac{\omega'}{\omega} + \frac{\omega}{\omega'} + 2m\left(\frac{1}{\omega}-\frac{1}{\omega'}\right) + m^2\left(\frac{1}{\omega}-\frac{1}{\omega'}\right)^2\right]$$
由能量动量守恒可以得到
$$m\left(\frac{1}{\omega}-\frac{1}{\omega'}\right) = \cos\theta - 1$$
所以
$$|\calM|^2 = \frac{q^4}{8 m (m+\omega-\omega')\omega\omega'}\left[\frac{\omega'}{\omega} + \frac{\omega}{\omega'} - \sin^2\theta\right]$$
}

\ech
\end{frame}

\begin{frame}
\chtitle{第5题 解答(续)}
\bch
{\scriptsize
同第14课的推导方法,但不对末态$d\Omega$积分:
\be
\frac{d\sigma}{d\Omega} = \frac{1}{(2\pi)^2} \int \omega'^2 d\omega' \int d^3\vecp' |\calM|^2\delta(p+k-p'-k')
\ee
注意因末态粒子不同,不需要除以$2!$因子。入射速度$\upsilon = 1$所以也可以略去。

对$\vecp'$积分,得到
\be
\frac{d\sigma}{d\Omega} =\frac{1}{(2\pi)^2} \int \omega'^2 d\omega' |\calM|^2\delta(m+\omega-\omega'-\sqrt{m^2+(\omega^2+\omega'^2-2\omega\omega'\cos\theta)}) 
\ee
对$\omega'$进行积分,利用复合$\delta$函数的积分公式:
$$\int g(x) \delta(f(x)) dx = \sum_{x^*:f(x^*)=0} g(x^*)\frac{1}{|f'(x^*)|}$$
和

\be
\left\vert\frac{d\left(m+\omega-\omega'-\sqrt{m^2+(\omega^2+\omega'^2-2\omega\omega'\cos\theta)} \right)}{d\omega'} \right\vert  = 1 + \frac{\omega'-\omega\cos\theta}{\sqrt{m^2+(\omega^2+\omega'^2-2\omega\omega'\cos\theta)}}
\ee
}
\ech
\end{frame}

\begin{frame}
\chtitle{第5题 解答(续)}
\bch
{\scriptsize
以及能量守恒 $\sqrt{m^2+(\omega^2+\omega'^2-2\omega\omega'\cos\theta)} = m+\omega-\omega'$,我们得到:
\bea
\frac{d\sigma}{d\Omega} &=& \frac{1}{(2\pi)^2}  \omega'^2  |\calM|^2 \frac{1}{1+\frac{\omega'-\omega\cos\theta}{m+\omega-\omega'}} \newl
&=& \frac{1}{(2\pi)^2}  \omega'^2  |\calM|^2 \frac{m+\omega-\omega'}{m+\omega(1-\cos\theta)} \newl
&=& \frac{1}{(2\pi)^2}  \omega'^2  |\calM|^2 \frac{m+\omega-\omega'}{m+\omega m(\frac{1}{\omega'}-\frac{1}{\omega})} \newl
&=& \frac{1}{(2\pi)^2}  \omega'^2  |\calM|^2 \frac{\omega'(m+\omega-\omega')}{m\omega} \newl
&=& \frac{q^4}{32\pi^2}  \frac{\omega'^2}{\omega^2m^2}\left[\frac{\omega'}{\omega} + \frac{\omega}{\omega'} - \sin^2\theta\right]   \newl
\eea


}

\ech
\end{frame}

\end{document}
