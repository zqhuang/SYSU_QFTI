\documentclass[CJK]{beamer}
\usepackage{CJKutf8}
\usepackage{beamerthemesplit}
\usetheme{Malmoe}
\useoutertheme[footline=authortitle]{miniframes}
\usepackage{amsmath}
\usepackage{amssymb}
\usepackage{graphicx}
\usepackage{color}
\graphicspath{{figures/}}
\def \bch {\begin{CJK}{UTF8}{gbsn}}
\def \ech {\end{CJK}}
\def \bex {\begin{minipage}{0.3\textwidth}\includegraphics[width=1in]{jugelizi.png}\end{minipage}\begin{minipage}{0.6\textwidth}}
\def \eex {\end{minipage}}
\def \chtitle#1 {\frametitle{\bch #1 \ech}}
\def \skipline { {\vskip 0.1in}}
\def \langr {\mathcal{L}}
\def \hamil {\mathcal{H}}
\def \vecx {\mathbf{x}}
\def \veck {\mathbf{k}}
\def \vecp {\mathbf{p}}
\def \hatphi {\hat{\phi}}
\def \hatq {\hat{q}}
\def \hatpi  {\hat{\pi}}
\def \vel {\upsilon}
\def \Dint {\mathcal{D}}
\def \adag {{\hat{a}^\dagger}}
\def \hata {\hat{a}}
\def \hatN {\hat{N}}
\def \hatH {\hat{H}}
\def \nket { {| n \rangle}}
\def \bran { {\langle n |}}

\title{Quantum Field Theory I \\ Lesson 01 - Introduction}
  \author{}
  \date{}


\begin{document}

\begin{frame}
 
\begin{center}
\begin{Large}
\bch
量子场论 I \\
第一课:坐标系和张量场简介

{\vskip 0.3in}

授课: 黄志琦

\ech
\end{Large}
\end{center}

\vskip 0.2in

\bch
教材:《简明量子场论》,王正行,北京大学出版社
\ech

\bch
课件下载
\ech
https://github.com/zqhuang/SYSU\_QFTI

\end{frame}

\begin{frame}
\frametitle{\bch 评分 \ech (Evaluation)}
\bch
总分 = 作业30分 + 课堂表现20分  + 期末考试50分
\ech
\end{frame}

\begin{frame}
\frametitle{\bch 互相简单介绍 \ech}
\bch
自我介绍:
\begin{itemize}
\item{名字}
\item{物理基础:高等数学,线性代数,理论力学,狭义相对论,量子力学这五门课的学习情况(这些是什么鬼?学过大部分忘了?学过大部分还会?精通?)}
\end{itemize}
\ech
\end{frame}

\begin{frame}
  \chtitle{第一章要掌握的内容}
\bch   
\begin{itemize}
\item{自然单位制和量纲}
\item{坐标,度规和张量}
\item{作用量和拉氏密度的概念}
\item{Euler-Lagrange方程的推导}
\item{Noether定理的推导}
\end{itemize}

\ech
\end{frame}

\begin{frame}
\frametitle{\bch 自然单位制和量纲 \ech}
\bch 自然单位制 \ech $c = \hbar  = 1$

\bch
长度量纲 = 时间量纲 = 质量量纲$^{-1}$ = 能量量纲$^{-1}$ 
\ech

{\vskip 0.2in}
\begin{minipage}{0.3\textwidth}
\includegraphics[width=1in]{jugelizi.png}
\end{minipage}
\begin{minipage}{0.6\textwidth}
\bch
力的量纲 \\
= 质量量纲 $\times$ 长度量纲/时间量纲$^2$ \\
= 质量量纲$^2$
\ech
\end{minipage}

\end{frame}

\begin{frame}
\frametitle{\bch 思考题\ech}
\bch
\addfig{0.5}{think.jpg}

下面物理量的量纲是质量的多少次方?

速度$v$

角速度$\omega$

角动量$L$

加速度$a$

能量密度$\rho$

压强$p$

牛顿引力常数$G$

\ech
\end{frame}


\begin{frame}
\frametitle{\bch 思考题 \ech}
\bch
$1\SIkg$的倒数是多少$\SIcm$?

\bcenter
\lfig{1}{blackq.jpg}

根本不知道你在说什么.jpg
\ecenter
\ech
\end{frame}



\begin{frame}
\frametitle{\bch 时空坐标的标记以及一些约定 \ech}
\bch
\begin{itemize}
\item{时空用坐标$(x^0, x^1, x^2, x^3)$来标记,其中$x^0$为时间坐标,有时也写作$t$;省略上标的坐标$x$是$(x^0, x^1, x^2, x^3)$的简写。}
\item{希腊字母指标$\mu, \nu, \rho, \sigma,\ldots$ 默认跑遍0,1,2,3}
\item{拉丁字母指标$i, j,  k, l,\ldots$ 默认跑遍1,2,3}
\item{黑体表示三维空间矢量,例如 $\mathbf{dx} \equiv (dx^1, dx^2, dx^3)$}
\end{itemize}
\ech
\end{frame}


\begin{frame}
\frametitle{\bch 时空坐标的标记以及一些约定 \ech}
\bch
\begin{itemize}
\item{爱因斯坦求和规则:重复出现的指标默认求和。例如:$dx^\mu dx_{\mu}$若无特殊说明等价于$\sum_{\mu = 0}^3 dx^\mu dx_{\mu}$} 
\item{空间坐标的偏微分$\frac{\partial}{\partial x^\mu}$经常简写成$\partial_\mu$. 省略下标的偏微分符号$\partial$是$(\partial_0,\partial_1,\partial_2,\partial_3)$的简写。算符$\nabla$则仅是空间偏微分$(\partial_1, \partial_2, \partial_3)$的简写。}
\item{克罗内克符号(Kronecker delta) $\delta^\mu_{\ \nu}$当$\mu = \nu$时为1,否则为0.}
\end{itemize}
\ech

\end{frame}


\begin{frame}
\chtitle{度规(metric)}
\bch 时空间隔元的定义:

$ds^2 = g_{\mu\nu} dx^\mu dx^\nu$

\ech
\end{frame}


\begin{frame}
\frametitle{\bch 度规 (metric) \ech}
\bch 时空间隔元:\ech
\bch 狭义相对论的平直时空,也叫闵氏空间(Minkowski spacetime)的度规 \ech

\begin{equation} 
g_{\mu\nu} =  \left( \begin{array}{rrrr} 1 & 0 & 0 & 0 \\ 0 & -1 & 0 & 0 \\ 0 & 0 & -1 & 0 \\ 0 & 0 & 0 & -1 \end{array} \right) \nonumber
\end{equation}

\bch 这样时空间隔元可以简写成 $ds^2 = dt^2 - \mathbf{dx}^2$ \ech

\end{frame}


\begin{frame}
\frametitle{\bch 一般性张量(tensor)的定义 \ech}
\bch
一般性张量在坐标变换$x \rightarrow \tilde{x}$下的变化:


$$\tilde{T}^{\mu\nu\ldots}_{\ \ \alpha\beta\ldots} = \left(\frac{\partial \tilde{x}^\mu}{\partial x^\rho}\frac{\partial \tilde{x}^\nu}{\partial x^\sigma} \frac{\partial x^\gamma }{\partial \tilde{x}^\alpha}\frac{\partial x^\lambda}{\partial \tilde{x}^\beta} \ldots \right)T^{\rho\sigma\ldots}_{\ \ \gamma\lambda\ldots} $$

\skiplines

(上个学年很多同学看到这里就退课了\bye)
\ech
\end{frame}

\begin{frame}
\frametitle{\bch 思考题 \ech}
\bch
如果$\phi$是一个标量(0阶张量,在坐标变换下不变),时空度规为$g_{\mu\nu}$,时空度规的逆矩阵为$g^{\mu\nu}$。试证明:
\begin{itemize}
\item{任何常数都是标量}
\item{时空微元$dx^\mu$是张量(一阶张量,也称为矢量)}
\item{$g_{\mu\nu}$本身是张量 (二阶张量)}
\item{$\partial_\mu\phi$是张量(矢量)}
\item{克罗内克符号$\delta^{\mu}_{\ \nu}$是张量}
\item{度规的逆矩阵$g^{\mu\nu}$是张量}
\item{如果定义$dx_{\mu} \equiv g_{\mu\nu} dx^{\nu}$,则$dx_\mu$也是张量}
\item{如果定义$\partial^\mu\phi \equiv g^{\mu\nu}\partial_\mu\phi$,则$\partial^\mu\phi$是张量}
\item{如果$T^{\mu\nu}$是二阶张量,则$T^{\mu}_{\ \mu}$是标量}
\item{如果$A^{\mu}$和$B^{\mu}$都是矢量,则$A^\mu B^\nu$是二阶张量}
\end{itemize}

\ech
\end{frame}


\begin{frame}
\frametitle{\bch 张量指标的升降\ech}
\bch
通过前面的讨论,我们发现张量的坐标可以通过度规$g_{\mu\nu}$和其逆矩阵$g^{\mu\nu}$来升降
\ech
{\vskip 0.1in}

\begin{minipage}{0.3\textwidth}
\includegraphics[width=1in]{jugelizi.png}
\end{minipage}
\begin{minipage}{0.6\textwidth}
$$F_{\mu\nu} = g_{\mu\rho}g_{\nu\sigma}F^{\rho\sigma}$$
$$T^{\mu}_{\ \nu\rho} = g^{\mu\sigma} T_{\sigma\nu\rho}$$
$$g^{\mu}_{\ \nu} = g^{\mu\rho}g_{\rho\nu} = \delta^{\mu}_{\ \nu}$$
\end{minipage}
\end{frame}


\begin{frame}
\frametitle{\bch 张量的直积 \ech}
\bch
通过前面的讨论,我们发现一个m阶张量和一个n阶张量可以直接相乘生成m+n阶张量。
\ech
{\vskip 0.1in}

\begin{minipage}{0.3\textwidth}
\includegraphics[width=1in]{jugelizi.png}
\end{minipage}
\begin{minipage}{0.6\textwidth}
\bch
$dx^\mu dx^\nu$是二阶张量。

$g_{\mu\nu}g^{\rho\sigma}$ 是四阶张量
\ech
\end{minipage}
\end{frame}



\begin{frame}
\frametitle{\bch 张量指标的收缩\ech}
\bch
通过前面的讨论,我们发现张量的上下指标可以收缩,产生低两阶的张量。
\ech
{\vskip 0.1in}

\begin{minipage}{0.3\textwidth}
\includegraphics[width=1in]{jugelizi.png}
\end{minipage}
\begin{minipage}{0.6\textwidth}
$$dx^\mu dx_\mu = ds^2$$
$$g^{\mu}_{\ \mu} = 4$$
$$g^{\mu\rho}g_{\rho\nu} = \delta^{\mu}_{\ \nu}$$
\end{minipage}
\end{frame}



\begin{frame}
\frametitle{\bch课堂互动\ech}
\bch
\addfig{1}{think2.jpg}
如果$\phi$是标量,那么$\partial_\mu\partial_\nu\phi$是张量吗?
\ech
\end{frame}

\begin{frame}
  \chtitle{课后作业 (9月19日周二课后交)}
  \bch
  \bitem
\item[1]{自然单位制下力矩的量纲是什么(=质量的多少次方)?}
\item[2]{某散射过程的截面为$\sigma = 10^{-3}/m_W^2$, 其中$m_W \approx 80 \mathrm{GeV}$是$W^{\pm}$的粒子质量。试换算出以$\mathrm{cm}^2$为单位的截面值。}
\item[3]{设四维时空度规为$g_{\mu\nu}$,试化简$g_{\mu\nu}g^{\mu\nu}$。}
  \eitem
  \ech
\end{frame}

\end{document}
