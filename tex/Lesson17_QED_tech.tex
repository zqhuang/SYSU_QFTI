\documentclass[CJK]{beamer}
\usepackage{CJKutf8}
\usepackage{beamerthemesplit}
\usetheme{Malmoe}
\useoutertheme[footline=authortitle]{miniframes}
\usepackage{amsmath}
\usepackage{amssymb}
\usepackage{graphicx}
\usepackage{color}
\graphicspath{{figures/}}
\def \bch {\begin{CJK}{UTF8}{gbsn}}
\def \ech {\end{CJK}}
\def \bex {\begin{minipage}{0.3\textwidth}\includegraphics[width=1in]{jugelizi.png}\end{minipage}\begin{minipage}{0.6\textwidth}}
\def \eex {\end{minipage}}
\def \chtitle#1 {\frametitle{\bch #1 \ech}}
\def \skipline { {\vskip 0.1in}}
\def \langr {\mathcal{L}}
\def \hamil {\mathcal{H}}
\def \vecx {\mathbf{x}}
\def \veck {\mathbf{k}}
\def \vecp {\mathbf{p}}
\def \hatphi {\hat{\phi}}
\def \hatq {\hat{q}}
\def \hatpi  {\hat{\pi}}
\def \vel {\upsilon}
\def \Dint {\mathcal{D}}
\def \adag {{\hat{a}^\dagger}}
\def \hata {\hat{a}}
\def \hatN {\hat{N}}
\def \hatH {\hat{H}}
\def \nket { {| n \rangle}}
\def \bran { {\langle n |}}

\title{Quantum Field Theory I \\ Lesson 17 - QED Technique}
\author{}
\date{}


\begin{document}

\begin{frame}
 
\begin{center}
\begin{Large}
\bch
量子场论 I 

{\vskip 0.3in}

第十七课 量子电动力学(QED)计算技巧

\ech
\end{Large}
\end{center}

\vskip 0.2in

\bch
课件下载
\ech
https://github.com/zqhuang/SYSU\_QFTI

\end{frame}

\begin{frame}
\chtitle{矩阵的迹}
\bch
{\small
先来回忆矩阵的迹的一些基本性质。
\begin{itemize}
\item{迹运算是线性的: $\trof{A+B} = \trof{A} + \trof{B}$}
\item{迹运算里的矩阵乘积可以轮转次序,例如$\trof{AB} =\trof{BA}$,$\trof{ABCD} = \trof{CDAB} = \trof{BCDA}$等}
\item{相似变换保持迹不变: $\trof{P^{-1}AP} = \trof{A}$}
\item{矩阵的迹是它所有本征值之和。}
\item{正定矩阵的行列式的对数等于该矩阵的对数的迹:$\ln(\det{A}) = \trof{\ln A}$ }
\end{itemize}

\skipline
QED的大量计算技巧就是围绕矩阵的迹展开的。例如,由于$\trof{\gamma^\mu\gamma^\nu} = \trof{\gamma^\nu\gamma^\mu}$,所以
$$\trof{\gamma^\mu\gamma^\nu} = \frac{1}{2}\trof{\{\gamma^\mu,\gamma^\nu\}} = g^{\mu\nu} \trof{I_{4\times 4}}  = 4 g^{\mu\nu}$$

}
\ech
\end{frame}


\begin{frame}
\chtitle{$\gamma$矩阵零迹定理}
\bch
如果在若干个$\gamma$矩阵(允许包括$\gamma^5$)的乘积里排除掉某个$\gamma^\nu$ ($\nu$允许为$0,1,2,3,5$中任选定的一个)后还剩下奇数个$\gamma$矩阵因子,则原乘积的迹为零。
\skipline

{\scriptsize
证明: 设$\gamma^{\mu_1}\gamma^{\mu_2}\ldots\gamma^{\mu_n}$中排除掉所有$\gamma^\nu$后还剩下奇数个$\gamma$矩阵,则因这些剩下的$\gamma$矩阵都与$\gamma^\nu$反对易,所以把$\gamma^\nu$从右边换到左边总共产生了奇数次$-1$因子。即
$$\gamma^{\mu_1}\gamma^{\mu_2}\ldots\gamma^{\mu_n} \gamma^\nu = -\gamma^\nu\gamma^{\mu_1}\gamma^{\mu_2}\ldots\gamma^{\mu_n} $$
右边同乘以$(\gamma^\nu)^{-1}$ 得到
$$\gamma^{\mu_1}\gamma^{\mu_2}\ldots\gamma^{\mu_n} = -\gamma^\nu\gamma^{\mu_1}\gamma^{\mu_2}\ldots\gamma^{\mu_n} (\gamma^\nu)^{-1}$$
两边取迹,然后利用相似变换下矩阵的迹不变,有
$$\trof{\gamma^{\mu_1}\gamma^{\mu_2}\ldots\gamma^{\mu_n} } = - \trof{\gamma^{\mu_1}\gamma^{\mu_2}\ldots\gamma^{\mu_n} }$$
即
$$\trof{\gamma^{\mu_1}\gamma^{\mu_2}\ldots\gamma^{\mu_n} } = 0$$

}

\ech
\end{frame}

\begin{frame}
\chtitle{$\gamma$矩阵零迹定理}
\bch
零迹定理的一些特例:

\begin{itemize}
\item{一个$\gamma$矩阵的迹为零}
\item{两个不同$\gamma$矩阵的乘积的迹为零}
\item{三个任意$\gamma$矩阵的乘积的迹为零}
\item{任意奇数个不含$\gamma^5$的$\gamma$矩阵的乘积的迹为零}
\item{任意奇数个Feynman符号的乘积的迹为零}
\end{itemize}
\ech
\end{frame}

\begin{frame}
\chtitle{Feynman符号的迹的展开定理}
\bch
{\small
设有$n$个矢量$u_1$, $u_2$, $\ldots$, $u_n$,
$$\trof{\prod_{i=1}^n\slashed{u}_i} = \sum_{k=2}^n (-1)^k (u_1 u_k)\trof{\prod_{j=2}^{k-1}\slashed{u}_j \prod_{j=k+1}^{n}\slashed{u}_j }$$
其中$u_1 u_k$表示$u_1$和$u_k$的内积。}

\skipline

{\small 利用这个定理可以求解任意Feynman符号的乘积的迹:
$$\trof{\slashed{u}_1\slashed{u_2}} = (u_1u_2)\trof{I_{4\times 4}} = 4u_1u_2$$ 
\bea
\trof{\slashed{u}_1\slashed{u}_2\slashed{u}_3\slashed{u}_4} &=& (u_1u_2)\trof{\slashed{u}_3\slashed{u}_4} - (u_1u_3)\trof{\slashed{u}_2\slashed{u}_4}+(u_1u_4)\trof{\slashed{u}_2\slashed{u}_3} \newl
&=& 4(u_1u_2)(u_3u_4) - 4(u_1u_3)(u_2u_4)+4(u_1u_4)(u_2u_3) 
\eea
}
\ech
\end{frame}


\begin{frame}
\chtitle{Feynman符号的迹的展开定理}
\bch
{\small
证明:若$n$为奇数,则根据零迹定理,等式两边都是零,不证自明。下面考虑$n$为偶数的情况。
首先
{\scriptsize
$$\slashed{A}\slashed{B} = A_\mu B_\nu \gamma^\mu\gamma^\nu =A_\mu B_\nu (2g^{\mu\nu}-\gamma^\nu\gamma^\mu) = 2 AB -  \slashed{B}\slashed{A} $$}
依次令$A = u_1$, $B = u_i$ ($i=2,3,\ldots, n$)即可把$\slashed{u}_1$轮换到乘积的最后去。最后再利用迹的性质把$\slashed{u}_1$轮换回来。
{\scriptsize
\bea
&& \trof{\slashed{u}_1 \slashed{u}_2 \ldots\slashed{u}_n} \newl
&=& 2 u_1u_2 \trof{\slashed{u}_3 \ldots\slashed{u}_n} - \trof{\slashed{u}_2 \slashed{u}_1\slashed{u}_3 \ldots\slashed{u}_n}\newl
&=& 2 u_1u_2 \trof{\slashed{u}_3 \ldots\slashed{u}_n} - 2(u_1u_3)\trof{\slashed{u}_2 \slashed{u}_4\ldots\slashed{u}_n} + \trof{\slashed{u}_2 \slashed{u}_3\slashed{u}_1\slashed{u}_4\ldots\slashed{u}_n}\newl
&=& \ldots\newl
&=& 2 \sum_{k=2}^n (-1)^k (u_1 u_k)\trof{\prod_{j=2}^{k-1}\slashed{u}_j \prod_{j=k+1}^{n}\slashed{u}_j } - \trof{ \slashed{u}_2 \ldots\slashed{u}_n\slashed{u}_1}\newl
&=& 2 \sum_{k=2}^n (-1)^k (u_1 u_k)\trof{\prod_{j=2}^{k-1}\slashed{u}_j \prod_{j=k+1}^{n}\slashed{u}_j } - \trof{ \slashed{u}_1\slashed{u}_2 \ldots\slashed{u}_n}
\eea
}
故$n$为偶数情况也得证。
}
\ech
\end{frame}


\begin{frame}
\chtitle{Feynman符号的迹的倒排定理}
\bch
{\small
任意$n$个矢量$u_1$, $u_2$, $\ldots$, $u_n$的Feynman符号的乘积的迹满足倒排定理:
$$\trof{\slashed{u}_1\slashed{u}_2\slashed{u}_3\ldots\slashed{u}_n} = \trof{\slashed{u}_n\slashed{u}_{n-1}\ldots\slashed{u}_2\slashed{u}_1} $$
\skipline

证明:当$n$为奇数时两边都是零,故不证自明。

当$n$为偶数时,用归纳法证明。$n=2$时显然成立。假设命题对$n-2$成立。把左边按展开定理展开;右边先利用迹的性质把$\slashed{u}_1$移到左边,然后按展开定理展开,再利用归纳假设即得到和左边展开一样的结果。
}
\ech
\end{frame}


\begin{frame}
\chtitle{$\gamma$矩阵和Feynman符号混合的一些常见情况}
\bch
{\small
\begin{itemize}
\item{$\trof{\slashed{A}\gamma^\mu} = 4 A^\mu$

证明:

$\trof{\slashed{A}\gamma^\mu} = A_\nu\trof{\gamma^\nu\gamma^\mu} = 4 A_\nu g^{\mu\nu} = 4 A^\mu $}
\item{$\trof{\slashed{A}\slashed{B}\slashed{C}\gamma^\mu} = 4(AB)C^\mu - 4(AC)B^\mu + 4(BC)A^\mu$

证明:

设$\trof{\slashed{A}\slashed{B}\slashed{C}\gamma^\mu} = X^\mu$,则对任意矢量$D$有
\bea
X^\mu D_\mu &=& \trof{\slashed{A}\slashed{B}\slashed{C}\slashed{D}} \newl
&=& 4(AB)(CD) - 4(AC)(BD)+ 4(AD)(BC) \newl
&=& \left[4(AB)C^\mu - 4(AC)B^\mu + 4(BC)A^\mu\right] D_\mu
\eea
由$D_\mu$的任意性即得
$X^\mu= 4(AB)C^\mu - 4(AC)B^\mu + 4(BC)A^\mu$
}
\end{itemize}
}
\ech
\end{frame}

\begin{frame}
\chtitle{$\gamma$矩阵的形式下标}
\bch
{\small
虽然$\gamma^\mu$并不是矢量,但我们可以形式得把它的指标降下去,定义:
$$\gamma_\mu \equiv g_{\mu\nu}\gamma^\nu$$
试证明:
\begin{itemize}
\item{$\gamma^\mu\gamma_\mu = 4$ }
\item{$\gamma^\mu\slashed{A}\gamma_\mu = -2\slashed{A}$ }
\end{itemize}

}
\ech
\end{frame}

\begin{frame}
\bch
学完这些,感觉我们小学数学又变强了。之后我们要拿这些武器来对付Dirac场。
\skiplines

下面转入对付$U(1)$规范场的武器。
\ech
\end{frame}

\begin{frame}
\chtitle{光子外线的替换规则}
\bch
{\small
QED相互作用为$qA_\mu j^\mu$,其中$j^\mu$为守恒电荷流。有光子外线的Feynman图的散射振幅写成$\calM_\mu = \left(\vece_{\veck,s}\right)_\mu^* J^\mu$或者$\calM_\mu = \left(\vece_{\veck,s}\right)_\mu J^\mu$的形式,则$J^\mu$可以看成$k$空间的守恒流,它满足
$k_\mu J^\mu = 0$。在以$\veck$方向为$z$轴的坐标系里,$k_\mu =(|\veck|, 0, 0, -|\veck|)$, 从而有$J^0 = J^3$。

{\scriptsize 注:上述结论的理论严格证明留到以后有时间再说。今天我们以掌握计算技术为首要目标。}

现在考虑初态或者末态自旋未知,需要把散射概率(散射振幅的平方)对所有自旋求和的情况。

$$|\calM^2| = \sum_{s=\pm 1} \left(\vece_{\veck,s}\right)^*_\mu \left(\vece_{\veck,s}\right)_\nu (J^\mu)^*J^\nu $$
利用$\vece_{\veck,\pm 1} = \vecn_1 \pm \ii \vecn_2$,即有
$$|\calM^2| = |J^1|^2 + |J^2|^2 =  |J^1|^2 + |J^2|^2 + |J^3|^2 - |J_0|^2 = -g_{\mu\nu}J^\mu J^\nu $$

由此我们得出:
{\bf 可以把外线的$\sum_{s} \left(\vece_{\veck,s}\right)^*_\mu \left(\vece_{\veck,s}\right)_\nu$替换为$-g_{\mu\nu}$}
}

\ech

\end{frame}

\begin{frame}
\chtitle{Compton散射}
\bch
下面我们尝试来完成Compton散射的散射振幅的计算:
{\tiny
\bea
 |\calM|^2 &=& \frac{q^4}{16 \omega_\gamma\omega_\gamma'} \sum_{\rm spins}\left\vert\bar{u}_{p_e'} \slashed{\vece}^*_{p_\gamma'}\left(\frac{\ii}{\slashed{p}_e+\slashed{p}_\gamma-m}+\frac{\ii}{\slashed{p}_e-\slashed{p}_\gamma'-m}\right)\slashed{\vece}_{p_\gamma} u_{p_e}   \right\vert^2 \newl
&=& \frac{q^4}{16 \omega_\gamma\omega_\gamma'} \sum_{\rm spins}(\vece_{p_\gamma'})_\mu (\vece^*_{p_\gamma})_\nu (\vece^*_{p_\gamma'})_\alpha (\vece_{p_\gamma})_\beta  \left(\bar{u}_{p_e'} \gamma^\mu\left(\frac{\ii}{\slashed{p}_e+\slashed{p}_\gamma-m}+\frac{\ii}{\slashed{p}_e-\slashed{p}_\gamma'-m}\right) \gamma^\nu u_{p_e}  \right)^\dagger \newl
&& \times\left(\bar{u}_{p_e'} \gamma^\alpha \left(\frac{\ii}{\slashed{p}_e+\slashed{p}_\gamma-m}+\frac{\ii}{\slashed{p}_e-\slashed{p}_\gamma'-m}\right)\gamma^\beta u_{p_e}  \right) \newl
&=& \frac{q^4}{16 \omega_\gamma\omega_\gamma'} \sum_{\rm spins} (-g_{\mu\alpha})(- g_{\nu\beta} )\left(\bar{u}_{p_e'} \gamma^\mu\left(\frac{\ii}{\slashed{p}_e+\slashed{p}_\gamma-m}+\frac{\ii}{\slashed{p}_e-\slashed{p}_\gamma'-m}\right) \gamma^\nu u_{p_e}  \right)^\dagger \newl
&& \times\left(\bar{u}_{p_e'} \gamma^\alpha \left(\frac{\ii}{\slashed{p}_e+\slashed{p}_\gamma-m}+\frac{\ii}{\slashed{p}_e-\slashed{p}_\gamma'-m}\right)\gamma^\beta u_{p_e}  \right) \newl
&=& \frac{q^4}{16 \omega_\gamma\omega_\gamma'} \sum_{\rm spins} \left(\bar{u}_{p_e} \gamma^\nu \left(\frac{-\ii}{\slashed{p}_e+\slashed{p}_\gamma-m}+\frac{-\ii}{\slashed{p}_e-\slashed{p}_\gamma'-m}\right)\gamma^\mu   u_{p_e'}\right) \newl
&& \times\left(\bar{u}_{p_e'} \gamma_\mu \left(\frac{\ii}{\slashed{p}_e+\slashed{p}_\gamma-m}+\frac{\ii}{\slashed{p}_e-\slashed{p}_\gamma'-m}\right)\gamma_\nu u_{p_e}  \right)
\eea
}
\ech
\end{frame}

\begin{frame}
\chtitle{Compton散射}
\bch
上面的自旋求和项可以写成求迹的形式($1\times 1$矩阵的迹等于自身)
{\tiny
\be
\sum_{\rm spins} \trof{\bar{u}_{p_e} \gamma^\nu \left(\frac{-\ii}{\slashed{p}_e+\slashed{p}_\gamma-m}+\frac{-\ii}{\slashed{p}_e-\slashed{p}_\gamma'-m}\right)\gamma^\mu   u_{p_e'}\bar{u}_{p_e'} \gamma_\mu \left(\frac{\ii}{\slashed{p}_e+\slashed{p}_\gamma-m}+\frac{\ii}{\slashed{p}_e-\slashed{p}_\gamma'-m}\right)\gamma_\nu u_{p_e} } 
\ee
}
利用迹可以轮换乘积的性质,把最右边的$\gamma^\nu u_{p_e}$移到左边,并利用$u\bar{u}$的自旋求和性质(见第11课前半部分),得到这个和为
{\tiny
\bea
 \trof{\gamma_\nu\frac{\slashed{p}_e+m}{2\omega_e} \gamma^\nu \left(\frac{-\ii}{\slashed{p}_e+\slashed{p}_\gamma-m}+\frac{-\ii}{\slashed{p}_e-\slashed{p}_\gamma'-m}\right)\gamma^\mu  \frac{\slashed{p}_{e'}+m}{2\omega_{e'}} \gamma_\mu \left(\frac{\ii}{\slashed{p}_e+\slashed{p}_\gamma-m}+\frac{\ii}{\slashed{p}_e-\slashed{p}_\gamma'-m}\right) } 
\eea
}
这样我们顺利得到了电子的$1/\omega$因子,把它们都提取到外面,定义
{\scriptsize 
$$|\calM|^2 = \frac{q^4}{16 \omega_\gamma\omega_\gamma'\omega_e\omega_{e'}} K$$}
再利用$\gamma$矩阵形式下标的性质,得到
{\tiny
$$K =  \trof{(-\slashed{p}_e+2m) \left(\frac{1}{\slashed{p}_e+\slashed{p}_\gamma-m}+\frac{1}{\slashed{p}_e-\slashed{p}_\gamma'-m}\right)(-\slashed{p}_{e'}+2m)\left(\frac{1}{\slashed{p}_e+\slashed{p}_\gamma-m}+\frac{1}{\slashed{p}_e-\slashed{p}_\gamma'-m}\right)}$$
}

\ech
\end{frame}

\begin{frame}
\chtitle{Compton散射}
\bch
{\scriptsize
\bea
K &=&  \trace\left[(\slashed{p}_e-2m) \left(\frac{\slashed{p}_e+\slashed{p}_\gamma+m}{(p_e+p_\gamma)^2-m^2}+\frac{\slashed{p}_e-\slashed{p}_\gamma'+m}{(p_e-p_\gamma')^2-m^2}\right)\right. \newl
&& \left. \times (\slashed{p}_{e'}-2m) \left(\frac{\slashed{p}_e+\slashed{p}_\gamma+m}{(p_e+p_\gamma)^2-m^2}+\frac{\slashed{p}_e-\slashed{p}_\gamma'+m}{(p_e-p_\gamma')^2-m^2}\right)\right]
\eea
}
然后再利用Feynman符号乘积的迹的展开公式把上式展开成几十项的和…
\ech
\end{frame}

\begin{frame}
\chtitle{无力吐槽}
\bch
下节课我们继续…
\ech
\end{frame}

\end{document}


