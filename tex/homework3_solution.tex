\documentclass[CJK]{beamer}
\usepackage{CJKutf8}
\usepackage{beamerthemesplit}
\usetheme{Malmoe}
\useoutertheme[footline=authortitle]{miniframes}
\usepackage{amsmath}
\usepackage{amssymb}
\usepackage{graphicx}
\usepackage{color}
\graphicspath{{figures/}}
\def \bch {\begin{CJK}{UTF8}{gbsn}}
\def \ech {\end{CJK}}
\def \bex {\begin{minipage}{0.3\textwidth}\includegraphics[width=1in]{jugelizi.png}\end{minipage}\begin{minipage}{0.6\textwidth}}
\def \eex {\end{minipage}}
\def \chtitle#1 {\frametitle{\bch #1 \ech}}
\def \skipline { {\vskip 0.1in}}
\def \langr {\mathcal{L}}
\def \hamil {\mathcal{H}}
\def \vecx {\mathbf{x}}
\def \veck {\mathbf{k}}
\def \vecp {\mathbf{p}}
\def \hatphi {\hat{\phi}}
\def \hatq {\hat{q}}
\def \hatpi  {\hat{\pi}}
\def \vel {\upsilon}
\def \Dint {\mathcal{D}}
\def \adag {{\hat{a}^\dagger}}
\def \hata {\hat{a}}
\def \hatN {\hat{N}}
\def \hatH {\hat{H}}
\def \nket { {| n \rangle}}
\def \bran { {\langle n |}}

\title{Quantum Field Theory I \\ Homework 3 solution}
  \author{}
  \date{}


\begin{document}

\begin{frame}
 
\begin{center}
\begin{Large}
\bch
量子场论 I 

{\vskip 0.3in}

第三次课后作业参考答案
\skipline
\skipline

如发现参考答案有错误请不吝告知(微信zhiqihuang或邮箱huangzhq25@sysu.edu.cn)
\ech
\end{Large}
\end{center}

\vskip 0.2in

\bch
课件下载
\ech
https://github.com/zqhuang/SYSU\_QFTI

\end{frame}

\begin{frame}
\chtitle{先乱入一下傅立叶变换的一些符号(据说你们经常搞错…)}
\bch
傅立叶变换
$$\partial_\mu \rightarrow -\ii k_\mu $$
$$\partial^\mu \rightarrow -\ii k^\mu $$
$$\veck \equiv (k^1, k^2, k^3) = (-k_1, -k_2, -k_3)$$

所有加粗的三维矢量字母都是上标,例如$\vecA$, $\vecx$等。而梯度符号则是
$$\nabla \equiv (\frac{\partial}{\partial x^1}, \frac{\partial}{\partial x^2}, \frac{\partial}{\partial x^3}) = (\partial_1, \partial_2, \partial_3)$$ 
因为默认上标的$\vecx$跑到了分母,所以$\nabla$是默认下标,所以有:
$$\nabla \rightarrow -\ii(k_1, k_2, k_3) = \ii(k^1,k^2,k^3) = \ii \veck$$

\ech
\end{frame}

\begin{frame}
\chtitle{第1题:题目和思路}
\bch
题目:{\small 证明自由$U(1)$规范场$A^\mu$在库仑规范$A^0=0, \nabla\cdot \vecA=0$下的Hamilton密度为
$$\hamil = \frac{1}{2}|\dot\vecA|^2 + \frac{1}{2}|\nabla\times \vecA|^2$$
并证明在傅立叶空间Hamilton量可以写成
$$H = \int d^3\veck\left[\frac{1}{2}|\dot\vecA|^2 + \frac{1}{2}|\veck \times \vecA|^2\right]$$}

\skipline
思路:老套路不需要思考了……
\ech
\end{frame}

\begin{frame}
\chtitle{第1题解答}
\bch
在库仑规范下,
$$F_0^i = \partial_0 A^i,\ F^0_i = \partial^0 A_i = -\partial_0 A^i$$ 
$$F_{21}= \partial_2 A_1 - \partial_1 A_2 = - \partial_2 A^1 + \partial_1 A^2 = (\nabla \times \vecA)_3$$
同理有$F_{13} = (\nabla\times \vecA)_2$, $F_{32} = (\nabla\times \vecA)_1$,都代入$\lagr = -\frac{1}{4}F^{\mu\nu}F_{\mu\nu}$就得到
$$\lagr = \frac{1}{2}|\dot\vecA|^2 - \frac{1}{2}|\nabla\times \vecA|^2$$
于是$A^i$对应的广义动量密度$\pi_i = \dot A^i$,然后就有
$$\hamil = \pi_i\dot{A^i} - \lagr = \frac{1}{2}|\dot\vecA|^2 + \frac{1}{2}|\nabla\times \vecA|^2$$
\ech
\end{frame}

\begin{frame}
\chtitle{第1题解答(续)}
\bch
$H$等于$\hamil$的全空间积分:
$$H = \int d^3\vecx \,\frac{1}{2}\left[|\dot\vecA|^2 + |\nabla\times \vecA|^2\right]$$
然后利用实空间二次项的积分等于傅立叶空间的同样二次项的积分,即得到
$$H = \int d^3\veck\left[\frac{1}{2}|\dot\vecA|^2 + \frac{1}{2}|\veck \times \vecA|^2\right]$$


\ech
\end{frame}


\begin{frame}
\chtitle{第2题: 题目和思路}
\bch
题目:{\small
利用上题的Hamilton量的表达式以及我们在课上得到的$\hat{\vecA}$在傅立叶空间的解:
$$\hat{\vecA} = \frac{1}{\sqrt{2|\veck|d^3\veck}}\sum_{s=\pm 1} \vece_{\veck,s}\left(\hata_{\veck, s} + \adag_{-\veck, s}\right)$$
证明Hamilton量的算符表达式为
$$\hat{H} = \sum_{\veck} \sum_{s=\pm 1}(\hat{N}_{\veck,s}+1/2)|\veck|$$
其中 $\hat{N}_{\veck, s} \equiv \adag_{\veck,s}\hata_{\veck, s}$是动量为$\veck$,自旋为$s$的粒子数算符。}
\skipline
思路:直接代入计算,利用产生湮灭算符的对易关系。
\ech
\end{frame}

\begin{frame}
\chtitle{第2题解答}
\bch
由
$$\hat{\vecA} = \frac{1}{\sqrt{2|\veck|d^3\veck}}\sum_{s=\pm 1} \vece_{\veck,s}\left(\hata_{\veck, s} + \adag_{-\veck, s}\right)$$
{\scriptsize (抱歉一开始出题时这个式子打错了,由此引起的问题均不扣分)} 可得到
$$\hat{\vecA^\dagger} = \frac{1}{\sqrt{2|\veck|d^3\veck}}\sum_{s=\pm 1} \vece^\dagger_{\veck,s}\left(\adag_{\veck, s} + \hata_{-\veck, s}\right)$$
再利用$d\hata/dt = -\ii |\veck| \hata$, $d\adag/dt = \ii|\veck| \adag$,有
$$\frac{d\hat{\vecA}}{dt} = \frac{-\ii\sqrt{|\veck|}}{\sqrt{2d^3\veck}}\sum_{s=\pm 1} \vece_{\veck,s}\left(\hata_{\veck, s} - \adag_{-\veck, s}\right)$$
$$\frac{d\hat{\vecA^\dagger}}{dt} = \frac{\ii\sqrt{|\veck|}}{\sqrt{2 d^3\veck}}\sum_{s=\pm 1} \vece^\dagger_{\veck,s}\left(\adag_{\veck, s} - \hata_{-\veck, s}\right)$$
\ech
\end{frame}

\begin{frame}
\chtitle{第2题解答(续)}
\bch
利用$\vece_{\veck,s}$ ($s=\pm 1$)的正交归一性,得到
\be
\frac{d^3\veck}{2}\left\vert\frac{d\hat\vecA}{dt}\right\vert^2 =\frac{d^3\veck}{2}\frac{d\hat\vecA^\dagger}{dt}\frac{d\hat\vecA}{dt} =\frac{|\veck|}{4}\sum_s \left(\hata_{\veck, s} - \adag_{-\veck, s}\right)\left(\adag_{\veck, s} - \hata_{-\veck, s}\right) 
\ee
再利用$\frac{\veck}{|\veck|}\times\vece_{\veck,s}$ ($s=\pm 1$)的正交归一性,得到
\bea
 \frac{d^3\veck}{2}|\veck\times \hat{\vecA}|^2 &=& \frac{|k|^2 d^3\veck}{2}\left(\frac{\veck}{|\veck|}\times \hat{\vecA}^\dagger\right)\left(\frac{\veck}{|\veck|}\times \hat{\vecA}\right) \newl
&=& \frac{|\veck|}{4}\sum_s\left(\adag_{\veck, s} + \hata_{-\veck, s}\right)\left(\hata_{\veck, s} + \adag_{-\veck, s}\right)
\eea
上面两个结果相加并对$\veck$求和即得:
$$\hat{H} = \sum_{\veck,s} \frac{|\veck|}{4}\left(\hata_{\veck,s}\adag_{\veck,s} 
+ \adag_{\veck,s}\hata_{\veck,s}+ \hata_{-\veck,s}\adag_{-\veck,s} + \adag_{-\veck,s}\hata_{-\veck,s}\right)$$

\ech
\end{frame}

\begin{frame}
\chtitle{第2题解答(续)}
\bch

因为是对所有$\veck$求和,可以把后面两项的$-\veck$换成$\veck$。这样后两项和前两项贡献相同, 就得到
$$\hat{H} = \sum_{\veck,s} \frac{|\veck|}{2}\left(\hata_{\veck,s}\adag_{\veck,s}  + \adag_{\veck,s}\hata_{\veck,s}\right)$$
再利用对易关系$\hata_{\veck,s}\adag_{\veck,s}  = \adag_{\veck,s}\hata_{\veck,s} + 1$,即得到
$$\hat{H} = \sum_{\veck,s} |\veck|\left(\adag_{\veck,s}\hata_{\veck,s}+\frac{1}{2}\right)$$



\ech
\end{frame}


\begin{frame}
\chtitle{第3题题目和思路}
\bch
题目:{\small
证明自由$U(1)$规范场$A^\mu$在库仑规范$A^0=0, \nabla\cdot \vecA=0$下的动量
$$P^i = \int d^3\vecx\, (\partial^0\vecA \cdot \partial^i\vecA)$$
是守恒量。并把它写成傅立叶空间的积分。
}

\skipline
\skipline
思路:默默祭出必杀技Noether定理
\ech
\end{frame}

\begin{frame}
\chtitle{第3题解答}
\bch
考虑沿第一个空间方向的空间整体(包括空间里的场)平移$x^\mu \rightarrow x^\mu+\epsilon \delta^\mu_1$,
平移后坐标为$x^\mu$的点在平移前的坐标则为$x^\mu-\epsilon \delta^\mu_1$。
$$\frac{\delta A^i}{\delta \epsilon} = -  \partial_1 A^i$$
拉氏密度改变量$\delta \lagr = -  \epsilon \partial_1 \lagr = \epsilon \partial_\mu(-\delta^\mu_1 \lagr)$。
所以Noether定理里可以取$F^\mu = -\delta^\mu_1\lagr$。

守恒流为:
$$j^\mu = F^\mu - \frac{\delta A^i}{\delta \epsilon}\frac{\partial\lagr}{\partial(\partial_\mu A^i)} =  -\delta^\mu_1\lagr -  \partial_1 A^i F^\mu_{\ i} $$
这里我们用到了课上推导过的结论$\frac{\partial \lagr}{\partial F_{\mu\nu}} = - F^{\mu\nu}$。
最后,$j^0$的全空间积分即为守恒量(沿第一个空间方向的动量):
$$P_1 = \int d^3\vecx\, \left(\partial_1 A^i \partial^0 A_i\right)=\int d^3\vecx\, \left( \partial^0 \vecA\cdot \partial_1 \vecA \right)$$
\ech
\end{frame}


\begin{frame}
\chtitle{第3题解答(续)}
\bch
上式我们用到了$F^0_{\ i} = \partial^0 A_i$。
重复同样的步骤,并把$P$的指标升上去,,我们可以得到对$i =1,2,3$均有
$$P^i = \int d^3\vecx\, \left(\partial^0 \vecA\cdot \partial^i \vecA \right)$$

因为这里的二次项不是对称的,在傅立叶空间有两种等价的方式来写这个积分:
$$P^i = \int d^3\veck\, (-\ii k^i) \left(\partial^0 \vecA^\dagger  \cdot \vecA \right)=\int d^3\veck\, (\ii k^i) \left( \vecA^\dagger  \cdot \partial^0 \vecA \right)$$
或更简洁地写成对称的形式:
$$\mathbf{P} = \int d^3\veck \, \frac{\ii\veck}{2}\left( \vecA^\dagger  \cdot \partial^0\vecA-\partial^0 \vecA^\dagger  \cdot \vecA \right)$$
\ech
\end{frame}

\begin{frame}
\chtitle{第4题题目和思路}
\bch
题目:{\small
利用上题的动量$P^i$的表达式以及我们在课上得到的$\hat{\vecA}$在傅立叶空间的解:
$$\hat{\vecA} = \frac{1}{\sqrt{2|\veck|d^3\veck}}\sum_{s=\pm 1} \vece_{\veck,s}\left(\hata_{\veck, s} + \adag_{-\veck, s}\right)$$
证明动量$\vecP \equiv (P^1, P^2, P^3)$的算符表达式为
$$\hat{\vecP} = \sum_{\veck}\sum_{s = \pm 1} \veck \hat{N}_{\veck, s}$$
其中 $\hat{N}_{\veck, s} \equiv \adag_{\veck,s}\hata_{\veck, s}$是动量为$\veck$,自旋为$s$的粒子数算符。
}

\skipline
思路:直接代入计算。
\ech
\end{frame}

\begin{frame}
\chtitle{第4题解答}
\bch
由
$$\hat{\vecA} = \frac{1}{\sqrt{2|\veck|d^3\veck}}\sum_{s=\pm 1} \vece_{\veck,s}\left(\hata_{\veck, s} + \adag_{-\veck, s}\right)$$
{\scriptsize (抱歉一开始出题时这个式子也打错了,由此引起的问题均不扣分)} 可得到
$$\hat{\vecA^\dagger} = \frac{1}{\sqrt{2|\veck|d^3\veck}}\sum_{s=\pm 1} \vece^\dagger_{\veck,s}\left(\adag_{\veck, s} + \hata_{-\veck, s}\right)$$
再利用$d\hata/dt = -\ii |\veck| \hata$, $d\adag/dt = \ii|\veck| \adag$,有
$$\partial^0\hat{\vecA}= \frac{-\ii\sqrt{|\veck|}}{\sqrt{2d^3\veck}}\sum_{s=\pm 1} \vece_{\veck,s}\left(\hata_{\veck, s} - \adag_{-\veck, s}\right)$$
$$\partial^0\hat{\vecA^\dagger} = \frac{\ii\sqrt{|\veck|}}{\sqrt{2 d^3\veck}}\sum_{s=\pm 1} \vece^\dagger_{\veck,s}\left(\adag_{\veck, s} - \hata_{-\veck, s}\right)$$

\ech
\end{frame}

\begin{frame}
\chtitle{第4题解答(续)}
\bch
{\scriptsize
\bea
\hat{\mathbf{P}}&=&\int d^3\veck\,\frac{\ii\veck}{2} \left(\hat\vecA^\dagger  \cdot \partial^0\hat\vecA -\partial^0 \hat\vecA^\dagger  \cdot \hat\vecA\right) \newl
&=&\frac{1}{4}\sum_{\veck,s}\veck\left[\left(\adag_{\veck, s} + \hata_{-\veck, s}\right)\left(\hata_{\veck, s} - \adag_{-\veck, s}\right) + \left(\adag_{\veck, s} - \hata_{-\veck, s}\right)\left(\hata_{\veck, s} + \adag_{-\veck, s}\right)\right] \newl
&=&\frac{1}{2}\sum_{\veck,s}\veck\left(\adag_{\veck, s}\hata_{\veck, s} -\hata_{-\veck,s}\adag_{-\veck, s}\right)\newl
&=&\frac{1}{2}\sum_{\veck,s}\veck \hat{N}_{\veck,s} + \frac{1}{2}\sum_{\veck,s}(-\veck) \left(\hat{N}_{-\veck,s}+1\right) \newl 
&=&\frac{1}{2}\sum_{\veck,s}\veck \hat{N}_{\veck,s} + \frac{1}{2}\sum_{\veck,s}\veck \left(\hat{N}_{\veck,s}+1\right) \newl
&=&\sum_{\veck,s}\veck \hat{N}_{\veck,s} + \sum_{\veck,s} \veck \newl
&=&\sum_{\veck,s}\veck \hat{N}_{\veck,s}
\eea

注:“真空动量”$\sum_{\veck,s}\veck$两两反向的动量抵消,总和为零。
}
\ech
\end{frame}

\begin{frame}
\chtitle{第5题题目和思路}
\bch
题目:{\small
对旋量$\psi$,证明$\bar\psi\slashed{\partial}\psi$是洛仑兹变换下的标量。
}

\skipline

思路:利用$\Lambda$矩阵的性质$\Lambda^\dagger = \gamma^0\Lambda^{-1}\gamma^0$。
\ech
\end{frame}


\begin{frame}
\chtitle{第5题解答}
\bch
{\scriptsize
设有洛仑兹变换$x^\mu \rightarrow a^\mu_{\ \nu} x^\nu$,对应的旋量变换矩阵为$\Lambda$矩阵。课上我们已经证明了$\Lambda^\dagger = \gamma^0\Lambda^{-1}\gamma^0$(或者两边右乘$\gamma^0$得到$\Lambda^\dagger \gamma^0= \gamma^0\Lambda^{-1}$),再者根据定义有$\Lambda^{-1}\gamma^\mu\Lambda = a^\mu_{\ \lambda}\gamma^\lambda$。我们利用这两条性质以及洛仑兹变换矩阵的正交性质$a^\mu_{\ \lambda}a_\mu^{\ \nu}=\delta^\nu_\lambda$来完成证明:

$\bar\psi\slashed{\partial}\psi$在洛仑兹变换下成为(注意洛仑兹变换的变换矩阵不依赖于坐标,偏导符号只作用到$\psi$上): 
\bea
&& (\psi^\dagger\Lambda^\dagger\gamma^0)(\gamma^\mu\partial_\mu)(\Lambda\psi) \newl
&=& (\psi^\dagger\gamma^0\Lambda^{-1})(\gamma^\mu a_\mu^{\ \nu}\partial_\nu)(\Lambda\psi) \newl
&=& (\bar\psi \Lambda^{-1}\gamma^\mu\Lambda \Lambda^{-1} a_\mu^{\ \nu}\partial_\nu) (\Lambda\psi) \newl
&=& \bar\psi a^\mu_{\ \lambda}\gamma^\lambda \Lambda^{-1} a_\mu^{\ \nu}\Lambda \partial_\nu \psi \newl
&=& \bar\psi a^\mu_{\ \lambda}a_\mu^{\ \nu}\gamma^\lambda \partial_\nu \psi \newl 
&=& \bar\psi \delta^\nu_\lambda\gamma^\lambda \partial_\nu \psi \newl
&=& \bar\psi \gamma^\nu \partial_\nu \psi \newl
&=& \bar\psi \slashed{\partial} \psi 
\eea
}
\ech
\end{frame}

\end{document}
