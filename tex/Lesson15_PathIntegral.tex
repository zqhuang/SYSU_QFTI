\documentclass[CJK]{beamer}
\usepackage{CJKutf8}
\usepackage{beamerthemesplit}
\usetheme{Malmoe}
\useoutertheme[footline=authortitle]{miniframes}
\usepackage{amsmath}
\usepackage{amssymb}
\usepackage{graphicx}
\usepackage{color}
\graphicspath{{figures/}}
\def \bch {\begin{CJK}{UTF8}{gbsn}}
\def \ech {\end{CJK}}
\def \bex {\begin{minipage}{0.3\textwidth}\includegraphics[width=1in]{jugelizi.png}\end{minipage}\begin{minipage}{0.6\textwidth}}
\def \eex {\end{minipage}}
\def \chtitle#1 {\frametitle{\bch #1 \ech}}
\def \skipline { {\vskip 0.1in}}
\def \langr {\mathcal{L}}
\def \hamil {\mathcal{H}}
\def \vecx {\mathbf{x}}
\def \veck {\mathbf{k}}
\def \vecp {\mathbf{p}}
\def \hatphi {\hat{\phi}}
\def \hatq {\hat{q}}
\def \hatpi  {\hat{\pi}}
\def \vel {\upsilon}
\def \Dint {\mathcal{D}}
\def \adag {{\hat{a}^\dagger}}
\def \hata {\hat{a}}
\def \hatN {\hat{N}}
\def \hatH {\hat{H}}
\def \nket { {| n \rangle}}
\def \bran { {\langle n |}}

\title{Quantum Field Theory I \\ Lesson 15 - Path Integral}
\author{}
\date{}


\begin{document}

\begin{frame}
 
\begin{center}
\begin{Large}
\bch
量子场论 I 

{\vskip 0.3in}

第十五课 路径积分:拉氏密度和传播子的联系

\ech
\end{Large}
\end{center}

\vskip 0.2in

\bch
课件下载
\ech
https://github.com/zqhuang/SYSU\_QFTI

\end{frame}



\begin{frame}
\chtitle{拉氏量和Feynman传播子}
\bch
{\small
考虑质量为m的自由场的作用量
$$S = \int d^4x\, \left(\frac{1}{2}\partial_\mu\phi\partial^\mu\phi - \frac{1}{2}m^2\phi^2\right)$$
利用四维散度的积分为零
$$S = \int d^4x\, \frac{1}{2}\phi(-\partial^2- m^2)\phi$$
夹在中间的算符在傅立叶空间为
$$-\partial^2 -m^2 \rightarrow k^2 - m^2$$
这里的$k^2-m^2$和Feynman传播子里的$k^2-m^2$有何联系?这是我们这节课要回答的问题。}
\ech
\end{frame}

\begin{frame}
\chtitle{课堂讨论}
\bch
如果记真空态为$|0\rangle$,试证明:
\begin{itemize}
\item{对于任何算符个数不为零的正则乘积$\rorder{\ldots}$,均有
$$\langle 0| \rorder{\ldots} | 0 \rangle = 0$$.}
\item{Feynman传播子
$$\contr{\hatphi(x)}{\hatphi(y)} =\langle 0 | \torder{\hatphi(x)\hatphi(y)} |0\rangle $$
}
\end{itemize}
所以Feynman传播子也可以看成真空态下$\torder{\hatphi(x)\hatphi(y)}$的平均值。
\ech
\end{frame}


\begin{frame}
\chtitle{路径积分基本定理}
\bch
{\small
事实上我们将一步到位地去求解任意$n$个编时排序的$\phi$的乘积的真空态平均值。为此引出下面的路径积分基本定理:
\skipline

对自由场$\phi$和$n$个四维时空点$x_1$, $x_2$, $\ldots$, $x_n$(注意这里下标仅用来区分这 $n$个点,不是张量指标)
$$\langle 0 | \torder{\hatphi(x_1)\hatphi(x_2)\ldots\hatphi(x_n)} |0\rangle = \lim_{\epsilon\rightarrow 0^+}\frac{\int \Dint\phi\, \phi(x_1)\phi(x_2)\ldots\phi(x_n) e^{\ii S - \int\frac{1}{2}\epsilon\phi^2d^4x} }{\int \Dint\phi \, e^{\ii S - \int \frac{1}{2}\epsilon\phi^2d^4x}}$$
\skipline}

{\scriptsize
注:把$x$看成场空间的自由度的标记,$\phi(x)$全体就构成了无穷维场空间的所有分量,其中的体积元$\Dint \phi$应理解为
$$\Dint\phi \equiv \prod_x d\phi(x)$$
在具体思考问题时,我们可以假想时空点是离散的且仅有有限$N$个,那么$\Dint\phi$就只是$N$维空间的积分元了。
}
\ech
\end{frame}


\begin{frame}
\chtitle{路径积分基本定理的证明}
\bch
根据量子作用量原理,末态概率振幅是初态概率振幅按每条连接末态和初态的路径以$e^{iS}\Dint\phi$的权重迭加的结果,其中作用量$S$仅从初态时刻积分到末态时刻。
\skipline

那么未指定边界条件的全空间积分$\int \Dint\phi \, e^{\ii S} $似乎并无确定意义。为此我们加上一个收敛因子:
$$\int \Dint\phi \, e^{\ii S - \int \frac{1}{2}\epsilon\phi^2d^4x}$$
这就相当于附加了场在无穷远处消失(否则无论多小的正数$\epsilon$都使积分$e^{\ii S-\int \frac{1}{2}\epsilon\phi^2d^4x}=0$)的边界条件。
\skipline

{\scriptsize 当然,如果你是个数学家,你一定会随手构造出20个在无穷远处不消失但积分$\int \phi^2d^4x$有限的例子并嘲笑物理学家的不严谨。事实上,这是数学家对物理学家的误解:当物理学家说“无穷远处消失”,并不是要纠结于要远处$\phi$处处为零,$\int\phi^2d^4x$在足够远处趋向于零就足够了。
}
\ech
\end{frame}


\begin{frame}
\chtitle{路径积分基本定理的证明}
\bch
{\small
问题又来了:因零点能的存在,要测量到有限的$\int\phi^2d^4x$是不可能的?

好在路径积分本来就描述的是一个未归一化的权重因子。也就是说,我们可以把$e^{-\frac{1}{2}\epsilon\int \langle 0 |\phi^2|0\rangle d^4x}$吸收到归一化因数里。那么,场从无穷远处消失的边界条件就退化为真空边界条件。也就是说:
\skipline}
 
{\bf $$\lim_{\epsilon\rightarrow 0^+}e^{\ii S-\int \frac{1}{2}\epsilon\phi^2d^4x}\Dint\phi$$是从初态为真空态到末态为真空态的路径的权重。}

因此路径积分
$$\lim_{\epsilon\rightarrow 0^+} \int \Dint\phi \,e^{\ii S-\int \frac{1}{2}\epsilon\phi^2d^4x}$$
就描述从$t = -\infty$的真空态到$t=\infty$的真空态的概率振幅的变化。

\ech
\end{frame}


\begin{frame}
\chtitle{路径积分基本定理的证明}
\bch
{\small
现在对给定的$n$个时空点$x_1$, $x_2$, $\ldots$, $x_n$。把这个路径积按时间拆分成$n+1$份,不妨设$-T < x_1^0 \le x_2^0 \le \ldots \le x_n^0<T$。在Sch\"odinger绘景中,记$x_i^0$时刻的态为 $|\phi_i\rangle$,记场算符为$\hatphi_S$,则有
}
{
\scriptsize
\bea
&& \lim_{\epsilon\rightarrow 0^+} \int \phi(x_1)\phi(x_2)\ldots\phi(x_n) \Dint\phi e^{\ii S-\int \frac{1}{2}\epsilon\phi^2d^4x} \newl
&& = N \int \Dint\phi_1\Dint\phi_2\Dint\ldots\Dint\phi_n\, \phi(x_1)\phi(x_2)\ldots\phi(x_n)  \newl
&&\times \langle 0|  e^{-\ii\hatH(T-x_n^0)} | \phi_n\rangle\langle \phi_n|e^{-\ii\hatH(x_{n}^0-x_{n-1}^0)} | \phi_{n-1}\rangle\ldots\langle\phi_2|e^{-\ii \hatH(x_2^0-x_1^0)} |\phi_1\rangle\langle \phi_1|e^{-\ii \hatH(x_1^0+T)}|0\rangle \newl
&=&  N \int \Dint\phi_1\Dint\phi_2\Dint\ldots\Dint\phi_n \newl
&& \times\, \langle 0 | e^{-\ii\hatH(T-x_n^0)} \hatphi_S(x_n)| \phi_n\rangle\langle \phi_n|e^{-\ii\hatH(x_{n}^0-x_{n-1}^0)} \hatphi_S(x_{n-1})| \phi_{n-1}\rangle \newl
&&\times \,\ldots \langle\phi_2|e^{-\ii \hatH(x_2^0-x_1^0)} \hatphi_S(x_1) |\phi_1\rangle\langle \phi_1|e^{-\ii \hatH(x_1^0+T)}|0\rangle \newl
&=& N \langle 0 | e^{-\ii\hatH(T-x_n^0)} \hatphi_S(x_n) e^{-\ii\hatH(x_{n}^0-x_{n-1}^0)} \ldots e^{-\ii \hatH(x_2^0-x_1^0)} \hatphi_S(x_1) e^{-\ii \hatH(x_1^0+T)}|0\rangle  \newl
&=&N\langle 0 | \hatphi(x_n)\hatphi(x_{n-1})\ldots\hatphi(x_2)\hatphi(x_1) | 0\rangle \ \ (\rm Heisenberg\  Picture)
\eea
其中$N$是路径积分的待定的归一化因子,在最后一行,我们变换回了Heisenberg绘景。
}

\ech
\end{frame}

\begin{frame}
\chtitle{路径积分基本定理的证明}
\bch
{\small
一般地对任意时间排序则有 
\bea
&& \langle 0|\torder{\hatphi(x_1)\hatphi(x_2)\ldots\hatphi(x_n)}|0\rangle \newl
&=& \frac{1}{N} \lim_{\epsilon\rightarrow 0^+} \int \phi(x_1)\phi(x_2)\ldots\phi(x_n) \Dint\phi e^{\ii S-\int \frac{1}{2}\epsilon\phi^2d^4x}
\eea
取$n=0$ 的特殊情形,就得到归一化因子
$$ N = \lim_{\epsilon\rightarrow 0^+} \int \Dint\phi e^{\ii S-\int \frac{1}{2}\epsilon\phi^2d^4x}$$
从而定理得证。
}
\ech
\end{frame}

\begin{frame}
\chtitle{路径积分方法的计算}
\bch
{\small
注意
$$ i S -\int d^4x\, \frac{1}{2}\epsilon\phi^2 = -\frac{1}{2}\int d^4x\,\phi \hat{C}^{-1}\phi $$
其中
$$\hat{C} \equiv \ii(-\partial^2-m^2+\ii\epsilon)^{-1}$$

取$n=2$的情形,就得到Feynman传播子
\bea
&&\contr{\hatphi(x)}{\hatphi(y)} \newl
&=& \langle 0 | \torder{\hatphi(x)\hatphi(y)} | 0\rangle \newl
&=&\lim_{\epsilon\rightarrow 0^+}\frac{\int \Dint\phi \,\phi(x)\phi(y) e^{-\frac{1}{2}\int d^4x\, \phi \hat{C}^{-1} \phi}}{\int \Dint\phi \, e^{-\frac{1}{2}\int d^4x\, \phi \hat{C}^{-1} \phi}}
\eea
}
\ech
\end{frame}

\begin{frame}
\chtitle{数学准备知识:多维空间高斯积分}
\bch
{\small
\begin{itemize}
\item{对$C>0$和任意实数$J$, 证明: 
$$\int_{-\infty}^\infty dx\,e^{-\frac{1}{2} x C^{-1}x + Jx } = \sqrt{2\pi C}e^{\frac{1}{2}JCJ}$$}
\item{
对$n\times n$的正定实对称矩阵$C$和$n$维向量$J$, 证明对$n$维实向量$x$的全空间$n$重积分: 
$$\int d^nx\, e^{-\frac{1}{2} x^T C^{-1}x + J^Tx } = (2\pi)^{n/2}\sqrt{ \det{C}}e^{\frac{1}{2}J^TCJ}$$
{\scriptsize 其中$T$表示矩阵转置。}}
\item{
对$n\times n$的正定实对称矩阵$C$和$n$维实向量$J=(J_1, J_2, \ldots, J_n)^T$, 证明对$n$维实向量$x=(x_1,x_2,\ldots, x_n)^T$的全空间$n$重积分: 
$$\frac{\int d^nx\, x_{i_1}x_{i_2}\ldots x_{i_m} e^{-\frac{1}{2} x^T C^{-1}x }}{\int d^nx\, e^{-\frac{1}{2} x^T C^{-1}x  }} = \left.\left(\frac{\partial^n}{\partial J_{i_1}\partial J_{i_2}\ldots\partial J_{i_m}} e^{\frac{1}{2}J^TCJ}\right)\right\vert_{J=0}$$
}
\end{itemize}
}
\ech
\end{frame}

\begin{frame}
\chtitle{有限到无限的过渡}
\bch
{\small
现在我们假设只有有限个时空点$x_1$, $x_2$, $\ldots$, $x_n$,对应的$\phi = (\phi(x_1)\sqrt{d^4x}, \phi(x_2)\sqrt{d^4x}, \ldots, \phi(x_n)d^4x)^T$构成了$n$维空间向量,利用上面的数学定理,
\bea
&&\contr{\hatphi(x_i)}{\hatphi(x_j)} d^4x\newl
&=&\left.\left(\lim_{\epsilon\rightarrow 0^+} \frac{\partial^2}{\partial(J_i)\partial(J_j)} e^{\frac{1}{2} J^TCJ}\right)\right\vert_{J=0} \newl
&=& \lim_{\epsilon\rightarrow 0^+} C_{ij} \newl
&=& \lim_{\epsilon\rightarrow 0^+} \sum_x (\delta(x-x_i) d^4x) C (\delta(x-x_j)d^4x)
\eea
}

\skipline
所以,从路径积分的观点看, Feynman传播子只是自由场关联矩阵$C$的一个矩阵元而已。

\ech
\end{frame}

\begin{frame}
\chtitle{转换到傅立叶空间}
\bch
{\small
两边约去$d^4x$,即有
\bea
&&\contr{\hatphi(x_i)}{\hatphi(x_j)} \newl
&=& \lim_{\epsilon\rightarrow 0^+} \int d^4x\, \delta(x-x_i)C \delta(x-x_j)\newl
&=& \lim_{\epsilon\rightarrow 0^+} \int d^4k \, \frac{e^{-\ii kx_i}}{(2\pi)^2} \frac{\ii}{k^2-m^2+\ii\epsilon} \frac{e^{\ii k x_j}}{(2\pi)^2}\newl
&=& \lim_{\epsilon\rightarrow 0^+} \int \frac{d^4k}{(2\pi)^4} \, \frac{\ii}{k^2-m^2+\ii\epsilon}e^{-\ii k(x_i-x_j)}
\eea
{\scriptsize
注:在第二个等号中我们把实空间的二次项积分化为了傅立叶空间的二次项积分。其中用到了$\delta(x-x_j)$的四维傅立叶变换为
$$\frac{1}{(2\pi)^2}\int \delta(x-x_j)e^{\ii k x}d^4x = \frac{1}{(2\pi)^2}e^{\ii k x_j}$$
}

\skipline

}
\ech
\end{frame}

\begin{frame}
\chtitle{后记}
\bch
{\small
\begin{itemize}
\item{自由场的关联矩阵$C$并不是正定实对称矩阵。这里我们对多元复变函数做了一个解析延拓。}
\item{利用路径积分基本定理和多维空间高斯积分的计算公式,实际上我们可以直接求出$n$个场算符的编时乘积的真空平均值,这使得我们多了一种计算散射振幅的办法}
\end{itemize}
这节课我们要掌握的两个核心结论是:
$$\langle 0 | \torder{\hatphi(x_1)\hatphi(x_2)\ldots\hatphi(x_n)} |0\rangle = \lim_{\epsilon\rightarrow 0^+}\frac{\int \Dint\phi\, \phi(x_1)\phi(x_2)\ldots\phi(x_n) e^{\ii S - \int\frac{1}{2}\epsilon\phi^2d^4x} }{\int \Dint\phi \, e^{\ii S - \int \frac{1}{2}\epsilon\phi^2d^4x}}$$
$$\frac{\int d^nx\, x_{i_1}x_{i_2}\ldots x_{i_m} e^{-\frac{1}{2} x^T C^{-1}x }}{\int d^nx\, e^{-\frac{1}{2} x^T C^{-1}x  }} = \left.\left(\frac{\partial^n}{\partial J_{i_1}\partial J_{i_2}\ldots\partial J_{i_m}} e^{\frac{1}{2}J^TCJ}\right)\right\vert_{J=0}$$

}
\ech
\end{frame}


\end{document}


