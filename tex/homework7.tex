\documentclass[CJK]{beamer}
\usepackage{CJKutf8}
\usepackage{beamerthemesplit}
\usetheme{Malmoe}
\useoutertheme[footline=authortitle]{miniframes}
\usepackage{amsmath}
\usepackage{amssymb}
\usepackage{graphicx}
\usepackage{color}
\graphicspath{{figures/}}
\def \bch {\begin{CJK}{UTF8}{gbsn}}
\def \ech {\end{CJK}}
\def \bex {\begin{minipage}{0.3\textwidth}\includegraphics[width=1in]{jugelizi.png}\end{minipage}\begin{minipage}{0.6\textwidth}}
\def \eex {\end{minipage}}
\def \chtitle#1 {\frametitle{\bch #1 \ech}}
\def \skipline { {\vskip 0.1in}}
\def \langr {\mathcal{L}}
\def \hamil {\mathcal{H}}
\def \vecx {\mathbf{x}}
\def \veck {\mathbf{k}}
\def \vecp {\mathbf{p}}
\def \hatphi {\hat{\phi}}
\def \hatq {\hat{q}}
\def \hatpi  {\hat{\pi}}
\def \vel {\upsilon}
\def \Dint {\mathcal{D}}
\def \adag {{\hat{a}^\dagger}}
\def \hata {\hat{a}}
\def \hatN {\hat{N}}
\def \hatH {\hat{H}}
\def \nket { {| n \rangle}}
\def \bran { {\langle n |}}

\title{Quantum Field Theory I \\ Homework 7}
  \author{}
  \date{}


\begin{document}

\begin{frame}
 
\begin{center}
\begin{Large}
\bch
量子场论 I 

{\vskip 0.3in}

第七次 课后作业 (5题共2.5分, 模拟期末考试题)

交作业时间: 12月12日,星期一,13:30pm

\ech
\end{Large}
\end{center}

\vskip 0.2in

\bch
课件下载
\ech
https://github.com/zqhuang/SYSU\_QFTI

\end{frame}


\begin{frame}
\chtitle{第1题(0.5分)}
\bch
如果不改变时空的度规,是否可能从真空中产生出一对正负粒子?为什么?
\ech
\end{frame}

\begin{frame}
\chtitle{第2题(0.5分)}
\bch
阐述什么是自由场。为什么自由场量子化之前先要对场进行傅立叶变换?
\ech
\end{frame}

\begin{frame}
\chtitle{第3题(0.5分)}
\bch
在课上我们把自由实标量场量子化为
$$ \hatphi(x) = \frac{1}{(2\pi)^{3/2}} \int \sqrt{\frac{d^3\veck}{2\omega}} \left(\hata_{\veck} e^{-ik_\mu x^\mu} + \adag_{\veck}e^{ik_\mu x^\mu}\right) $$

考虑两个实标量场$\phi$和$\psi$,拉氏密度为
$$\lagr = \frac{1}{2}\partial^\mu\phi\partial_\mu\phi + \frac{1}{2}\partial^\mu\psi\partial_\mu\psi - \frac{1}{2}m^2(\phi^2+\psi^2+\phi\psi) $$
把$\phi$和$\psi$都量子化 (写成独立的产生湮灭算符的线性迭加)。
\ech
\end{frame}


\begin{frame}
\chtitle{第4题(0.5分)}
\bch
设$p$为电子的四维动量,$k$为光子的四维动量,$m$为电子质量,求下列矩阵的迹
\begin{itemize}
\item{$\slashed{k}$}
\item{$\slashed{p}\gamma^5$}
\item{$\slashed{p}\slashed{k}\frac{1}{\slashed{p}+\slashed{k}-m}$}
\item{$\slashed{k}(\slashed{p}+\slashed{k}+m)^3\frac{1}{\slashed{p}+\slashed{k}-m}(\slashed{p}-\slashed{k}+m)^2\slashed{k}$}
\item{$(\slashed{p}+m)\gamma^\mu\frac{1}{\slashed{p}+\slashed{k} -m}\slashed{k}\gamma_\mu\slashed{k}$}
\end{itemize}
\ech
\end{frame}


\begin{frame}
\chtitle{第5题(0.5分)}
\bch
{\small
电子和$\mu$子可以分别看成独立的两个Dirac场$\psi_e$,$\psi_\mu$的粒子。和电磁场一起的拉氏密度为
$$\lagr = -\frac{1}{4}\Fup\Fdown + \bar{\psi}_e(\ii \slashed{D} - m)\psi_e + \bar{\psi}_\mu(\ii \slashed{D} - m)\psi_\mu$$
其中
$$ D_\mu = \partial_\mu + \ii q A_\mu$$

画出散射过程
$$e^+e^- \rightarrow \mu^+\mu^-$$
的非零的最低阶近似Feynman图。}
\ech
\end{frame}

\end{document}
