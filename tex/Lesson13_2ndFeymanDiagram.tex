\documentclass[CJK]{beamer}
\usepackage{CJKutf8}
\usepackage{beamerthemesplit}
\usetheme{Malmoe}
\useoutertheme[footline=authortitle]{miniframes}
\usepackage{amsmath}
\usepackage{amssymb}
\usepackage{graphicx}
\usepackage{color}
\graphicspath{{figures/}}
\def \bch {\begin{CJK}{UTF8}{gbsn}}
\def \ech {\end{CJK}}
\def \bex {\begin{minipage}{0.3\textwidth}\includegraphics[width=1in]{jugelizi.png}\end{minipage}\begin{minipage}{0.6\textwidth}}
\def \eex {\end{minipage}}
\def \chtitle#1 {\frametitle{\bch #1 \ech}}
\def \skipline { {\vskip 0.1in}}
\def \langr {\mathcal{L}}
\def \hamil {\mathcal{H}}
\def \vecx {\mathbf{x}}
\def \veck {\mathbf{k}}
\def \vecp {\mathbf{p}}
\def \hatphi {\hat{\phi}}
\def \hatq {\hat{q}}
\def \hatpi  {\hat{\pi}}
\def \vel {\upsilon}
\def \Dint {\mathcal{D}}
\def \adag {{\hat{a}^\dagger}}
\def \hata {\hat{a}}
\def \hatN {\hat{N}}
\def \hatH {\hat{H}}
\def \nket { {| n \rangle}}
\def \bran { {\langle n |}}

\title{Quantum Field Theory I \\ Lesson 13 - The Second Feynman diagram}
\author{}
\date{}


\begin{document}

\begin{frame}
 
\begin{center}
\begin{Large}
\bch
量子场论 I 

{\vskip 0.3in}

第十三课 第二个Feynman图

\ech
\end{Large}
\end{center}

\vskip 0.2in

\bch
课件下载
\ech
https://github.com/zqhuang/SYSU\_QFTI

\end{frame}


\begin{frame} 
\chtitle{三次项的相互作用} 
\bch
考虑拉氏密度为
$$\lagr = \frac{1}{2}\partial^\mu\phi\partial_\mu\phi - \frac{1}{2}m^2\phi^2 - \frac{g}{3!}\phi^3- \frac{\lambda}{4!}\phi^4$$
的实标量场。其中$\lambda>0$和$g$都是耦合常数。我们仅考虑它们很小($\lambda \ll 1$, $g\ll m$)的情况。

我们取如下的相互作用表象:
$$ H_0 = H_{\rm free},\ H_I = \int d^3\vecx\, \left(\frac{g}{3!}\phi^3 + \frac{\lambda}{4!} \phi^4\right)$$
其中$H_{\rm free} = \int d^3\vecx\, \frac{1}{2}(\dot\phi^2 + |\nabla \phi|^2 + m^2\phi^2)$是我们以前学过的自由场的Hamilton量。

\ech
\end{frame}

\begin{frame} 
\chtitle{继续上节课的热身问题} 
\bch
上节课的热身问题是:
\skipline

求两个动量为$\vecp_1$, $\vecp_2$的粒子发生散射,变为两个动量为$\vecp_3$, $\vecp_4$的粒子的概率幅。
$$\calM T/V \delta(p_1+p_2-p_3-p_4)d^4k= \langle \vecp_3, \vecp_4|  e^{-\ii\int_{-\infty}^\infty \hat{H}_I dt}|\vecp_1,\vecp_2\rangle$$
如上节课末尾所讲,我们已经把$\delta(p_1+p_2-p_3-p_4)d^4k$引入了左边的定义式。
\skipline

我们仍假设这四个动量$\vecp_1$, $\vecp_2$, $\vecp_3$, $\vecp_4$互不相同,即初态和末态粒子都是可以区分的。
\ech
\end{frame}

\begin{frame} 
\chtitle{一阶微扰展开} 
\bch
{\small 
我们做微扰展开,对带$\lambda$的项我们仍展开到一次,对带$g$的项我们要展开到二次。这是因为单个$g\phi^3$最多只能提供三个产生湮灭算符的乘积,无法湮灭两个粒子并产生两个新粒子。

$$e^{-\ii\int_{-\infty}^\infty \hat{H}_I dt}\approx 1-\ii\int d^4x\,\frac{g}{3!}\hat\phi(x)^3 -\frac{1}{2}\left(\int d^4x\,\frac{g}{3!}\hat\phi(x)^3\right)^2- \ii\int d^4x\, \frac{\lambda}{4!}\hat\phi^4 $$

第一项和第二项均没有贡献,第四项的贡献我们上节课算了。我们这节课来计算第三项的非零贡献:
$$ -\frac{g^2}{2(3!)^2} \int d^4x\, \int d^4y\,\langle \vecp_3, \vecp_4|  \hat\phi(x)^3 \hat\phi(y)^3 |\vecp_1,\vecp_2\rangle$$
}
\ech
\end{frame}

\begin{frame} 
\chtitle{有了新问题。} 
\bch
等等,这样直接写下来的式子还不完全正确。
\skipline

上节课中出现的算符是$\hatphi^4(x)$,不存在次序问题。现在出现了$\hatphi^3(x)\hatphi^3(y)$这样的项,它和$\hatphi^3(y)\hatphi^3(x)$不是等价的,到底该怎么排序呢?
\ech
\end{frame}

\begin{frame} 
\chtitle{化简} 
\bch
{\small
利用四维空间的积分公式:
$$\frac{1}{(2\pi)^4}\int d^4 x\, e^{-i(p_1+p_2-p_3-p_4)x} = \delta(p_1+p_2-p_3-p_4)$$
我们最终得到:
\be
\calM T/V \approx \frac{-\ii\lambda}{(2\pi)^2} \frac{(d^3\veck)^2}{4\sqrt{\omega_1\omega_2\omega_3\omega_4}} \delta(p_1+p_2-p_3-p_4)
\ee
注意到$d^3\veck = (2\pi)^3/V$(例如考虑一个边长为$L$的立方盒子),以及$d^4k = (2\pi/T)d^3\veck$,我们可以把上式写成
\be
\calM \approx -\ii\lambda \frac{1}{4\sqrt{\omega_1\omega_2\omega_3\omega_4}} \delta(p_1+p_2-p_3-p_4)d^4k
\ee
}
\ech
\end{frame}

\begin{frame} 
\chtitle{对结果的物理讨论} 
\bch
{\small
结果中的$\delta(p_1+p_2-p_3-p_4)d^4k$在$p_1+p_2-p_3-p_4=0$时为1,否则为零。这是散射问题能量动量守恒的自然结果。一般定义散射振幅时会把这个能量动量守恒因子排除在外,按这样的定义方法,最后结果就是:
\be
\calM \approx -\ii\lambda \frac{1}{4\sqrt{\omega_1\omega_2\omega_3\omega_4}} 
\ee
}
\ech
\end{frame}

\begin{frame} 
\chtitle{这个问题对应的Feynman图和Feynman规则} 
\bch
\begin{minipage}{0.45\textwidth}
\includegraphics[width=2in]{Feynman_lf4_tree.png}
\end{minipage}
\begin{minipage}{0.45\textwidth}
实标量场$\frac{\lambda}{4!}\phi^4$相互作用项的Feynman规则:
\begin{itemize}
\item{四线交叉顶角给出因子$-\ii\lambda$}
\item{每条外线给出因子$1/\sqrt{2\omega}$}
\end{itemize}
\end{minipage}

\ech
\end{frame}

\begin{frame}
\chtitle{选择题时间} 
\bch
我们算完了人生中第一个Feynman图,你的感想是:
\begin{itemize}
\item[A]{无敌是多么寂寞}
\item[B]{老师你故意找个最好算的图逗我们的吧!}
\item[C]{刚才发生了什么……}
\end{itemize}
\ech
\end{frame}

\begin{frame}
\chtitle{答案是B} 
\bch

下节课我们要动真格的了。
\ech
\end{frame}

\end{document}


