\documentclass[CJK]{beamer}
\usepackage{CJKutf8}
\usepackage{beamerthemesplit}
\usetheme{Malmoe}
\useoutertheme[footline=authortitle]{miniframes}
\usepackage{amsmath}
\usepackage{amssymb}
\usepackage{graphicx}
\usepackage{color}
\graphicspath{{figures/}}
\def \bch {\begin{CJK}{UTF8}{gbsn}}
\def \ech {\end{CJK}}
\def \bex {\begin{minipage}{0.3\textwidth}\includegraphics[width=1in]{jugelizi.png}\end{minipage}\begin{minipage}{0.6\textwidth}}
\def \eex {\end{minipage}}
\def \chtitle#1 {\frametitle{\bch #1 \ech}}
\def \skipline { {\vskip 0.1in}}
\def \langr {\mathcal{L}}
\def \hamil {\mathcal{H}}
\def \vecx {\mathbf{x}}
\def \veck {\mathbf{k}}
\def \vecp {\mathbf{p}}
\def \hatphi {\hat{\phi}}
\def \hatq {\hat{q}}
\def \hatpi  {\hat{\pi}}
\def \vel {\upsilon}
\def \Dint {\mathcal{D}}
\def \adag {{\hat{a}^\dagger}}
\def \hata {\hat{a}}
\def \hatN {\hat{N}}
\def \hatH {\hat{H}}
\def \nket { {| n \rangle}}
\def \bran { {\langle n |}}

\title{Quantum Field Theory I \\ Lesson 18 - Openmind I}
\author{}
\date{}


\begin{document}

\begin{frame}
 
\begin{center}
\begin{Large}
\bch
量子场论 I 

{\vskip 0.3in}

第十八课 脑洞大开系列(一)

\ech
\end{Large}
\end{center}

\vskip 0.2in

\bch
课件下载
\ech
https://github.com/zqhuang/SYSU\_QFTI

\end{frame}


\begin{frame}
\chtitle{解决物理问题的正确方法}
\bch
解决物理问题的正确方法:
\begin{itemize}
\item[1]{理解问题中的物理量的量纲和含义}
\item[2]{用对称性分析,量纲分析,简单的物理原则,或者取某些特殊情况估算答案的形式或数量级}
\item[3]{写下系统的运动方程并求解}
\item[4]{将答案和第2步做比较}
\end{itemize}

\skipline
步骤1,2,4往往被跳过,那是因为你解的题目有意被设计成书上例题的翻版。总是这样学习的话,遇到不熟悉的问题只会束手无策。

\ech
\end{frame}


\begin{frame}
\chtitle{例子1}
\bch
试估算地球表面的大气分子之间的势能大小。
\ech
\end{frame}


\begin{frame}
\chtitle{例子2}
\bch
一个半径约为$100$光年的球状星团中恒星的典型速度约为$100\SIkm/\SIs$。假设恒星的典型质量为太阳质量($\sim 10^{30}\SIkg$)。估算该球状星团中大约有多少颗恒星。
\ech
\end{frame}



\begin{frame}
\chtitle{例子3}
\bch
一维无限深势阱的宽度为$L$,试计算势阱中质量为$m$的粒子的基态能量。
\ech
\end{frame}



\begin{frame}
\chtitle{例子4}
\bch
黑洞可以看成一个把物质束缚在视界内的“势阱”
\begin{itemize}
\item{中性不旋转的黑洞的视界是什么形状}
\item{估算质量为$M$的中性不旋转的黑洞的视界大小}
\item{Hawking通过把量子理论在平直时空外推的方法计算出黑洞有辐射且有温度,试估算估算质量为$M$的中性不旋转的黑洞的温度(假设观测者在远处)}
\end{itemize}
\ech
\end{frame}

\begin{frame}
\chtitle{课堂讨论:例子5}
\bch
相对论里能量动量张量$T^{\mu\nu}$是对牛顿力学里的质量密度的推广。对静止的理想流体$T^{\mu\nu} = \mathrm{diag}(\rho, p, p, p)$,其中$\rho$是流体的能量密度,$p$是流体的压强。

牛顿引力理论里的质量密度就被推广为能量动量张量的迹$\mathrm{Tr}(T^{\mu\nu}) = \rho + 3p$。质量为$M$的大质量天体对附近物质的牛顿引力公式可以推广到
$$ F = - \int d^3\vecx \frac{GM(\rho + 3p) \vec{r}}{r^3}$$
光子通过太阳附近由于受到太阳引力而造成的偏折角分别为$\theta_\gamma$, 问用牛顿引力公式计算出的$\theta_\gamma$比实际小了多少?

\ech
\end{frame}

\begin{frame}
\chtitle{理科PhD学生第一守则}
\bch
拿到问题不知道该干嘛时,先做个傅立叶变换。
\ech
\end{frame}

\begin{frame}
\chtitle{养成到“对角化”空间考虑问题的习惯}
\bch

{\small任何拉氏量为二次型的系统都是“自由场”。把拉氏量对角化是透过迷雾看本质的有效方法。

\skipline
 第一守则的由来:当拉氏量里包含了空间偏导时,傅立叶变换往往能成功地把拉氏量对角化。}

\ech
\end{frame}

\begin{frame}
\chtitle{课堂讨论}
\bch
怎样把变量为$x^1, x^2,\ldots, x^n$的拉氏量
$$L = \frac{1}{2} \dot{x}^i K_{ij}\dot{x}^j - \frac{1}{2} (x^i-b^i) V_{ij} (x^j - b^j)$$
通过变量替换的方法变为$n$个独立谐振子的拉氏量之和? 
\ech
\end{frame}


\begin{frame}
\chtitle{例子6}
\bch
\bmini{0.6}
如图,无重力的空间中,质量为$m_1$和$m_2$的两个质点固定在弹性系数为$k$,长度为$L$的弹簧两端。如果在$t=0$时刻把弹簧压缩到$L/2$后释放,求两个质点之后的位置随时间的变化。
\emini
\bmini{0.35}
\includegraphics[width=1.8in]{doubleosc.png}
\emini
\ech
\end{frame}


\begin{frame}
\chtitle{例子7}
\bch
\bmini{0.6}
如图,无重力的空间中,质量为$m_1$和$m_2$的两个质点固定在弹性系数为$k$的弹簧两端。求该系统的量子零点能。

\skipline
思考:该结果和弹簧长度或者参考系有关吗?
\emini
\bmini{0.35}
\includegraphics[width=1.8in]{doubleosc.png}
\emini
\ech
\end{frame}

\end{document}


