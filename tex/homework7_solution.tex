\documentclass[CJK]{beamer}
\usepackage{CJKutf8}
\usepackage{beamerthemesplit}
\usetheme{Malmoe}
\useoutertheme[footline=authortitle]{miniframes}
\usepackage{amsmath}
\usepackage{amssymb}
\usepackage{graphicx}
\usepackage{color}
\graphicspath{{figures/}}
\def \bch {\begin{CJK}{UTF8}{gbsn}}
\def \ech {\end{CJK}}
\def \bex {\begin{minipage}{0.3\textwidth}\includegraphics[width=1in]{jugelizi.png}\end{minipage}\begin{minipage}{0.6\textwidth}}
\def \eex {\end{minipage}}
\def \chtitle#1 {\frametitle{\bch #1 \ech}}
\def \skipline { {\vskip 0.1in}}
\def \langr {\mathcal{L}}
\def \hamil {\mathcal{H}}
\def \vecx {\mathbf{x}}
\def \veck {\mathbf{k}}
\def \vecp {\mathbf{p}}
\def \hatphi {\hat{\phi}}
\def \hatq {\hat{q}}
\def \hatpi  {\hat{\pi}}
\def \vel {\upsilon}
\def \Dint {\mathcal{D}}
\def \adag {{\hat{a}^\dagger}}
\def \hata {\hat{a}}
\def \hatN {\hat{N}}
\def \hatH {\hat{H}}
\def \nket { {| n \rangle}}
\def \bran { {\langle n |}}

\title{Quantum Field Theory I \\ Homework 7 solution}
  \author{}
  \date{}


\begin{document}

\begin{frame}
 
\begin{center}
\begin{Large}
\bch
量子场论 I 

{\vskip 0.3in}

第七次课后作业参考答案
\skipline
\skipline

如发现参考答案有错误请不吝告知(微信zhiqihuang或邮箱huangzhq25@sysu.edu.cn)
\ech
\end{Large}
\end{center}

\vskip 0.2in

\bch
课件下载
\ech
https://github.com/zqhuang/SYSU\_QFTI

\end{frame}


\begin{frame}
\chtitle{第1题 }
\bch

{\small 
如果不改变时空的度规,是否可能从真空中产生出一对正负粒子?为什么?

\skiplines

解答: 不可能。因为真空态是能量最低态。产生任何粒子都违反能量守恒。
}
\ech
\end{frame}

\begin{frame}
\chtitle{第2题}
\bch
{
阐述什么是自由场。为什么自由场量子化之前先要对场进行傅立叶变换?

\skiplines

解答:自由场是可以把拉氏量写成由每个自由度单独的拉氏量之和,也就是拉氏量可以对角化的场。 (答只含场的二次项也正确。)

对场进行傅立叶变换是为了把拉氏量对角化。
}
\ech
\end{frame}

\begin{frame}
\chtitle{第3题 题目和思路}
\bch
{\small
题目:在课上我们把自由实标量场量子化为
$$ \hatphi(x) = \frac{1}{(2\pi)^{3/2}} \int \sqrt{\frac{d^3\veck}{2\omega}} \left(\hata_{\veck} e^{-ik_\mu x^\mu} + \adag_{\veck}e^{ik_\mu x^\mu}\right) $$

考虑两个实标量场$\phi$和$\psi$,拉氏密度为
$$\lagr = \frac{1}{2}\partial^\mu\phi\partial_\mu\phi + \frac{1}{2}\partial^\mu\psi\partial_\mu\psi - \frac{1}{2}m^2(\phi^2+\psi^2+\phi\psi) $$
把$\phi$和$\psi$都量子化 (写成独立的产生湮灭算符的线性迭加)。

\skipline
思路:先做简单的矩阵对角化即可把拉氏量写成独立实标量场之和,对独立的实标量场则可套用课上结论。
}
\ech
\end{frame}


\begin{frame}
\chtitle{第3题解答}
\bch
{\small
令$\chi = \frac{\phi+\psi}{\sqrt{2}}$, $\sigma = \frac{\phi-\psi}{\sqrt{2}}$则
$$\lagr = \frac{1}{2}\partial^\mu\chi\partial_\mu\chi + \frac{1}{2}\partial^\mu\sigma\partial_\mu\sigma -\frac{M_\chi^2}{2}\chi^2 -\frac{M_\sigma^2}{2}\sigma^2$$
其中$M_\chi = \sqrt{\frac{3}{2}}m$, $M_\sigma = \sqrt{\frac{1}{2}}m$。
于是
$$\chi = \frac{1}{(2\pi)^{3/2}} \int \sqrt{\frac{d^3\veck}{2\omega_\chi}} \left(\hata_{\veck,\chi} e^{-ik_\mu x^\mu} + \adag_{\veck,\chi}e^{ik_\mu x^\mu}\right) $$
$$\sigma = \frac{1}{(2\pi)^{3/2}} \int \sqrt{\frac{d^3\veck}{2\omega_\sigma}} \left(\hata_{\veck,\sigma} e^{-ik_\mu x^\mu} + \adag_{\veck,\sigma}e^{ik_\mu x^\mu}\right) $$
其中$\omega_\chi \equiv \sqrt{M_\chi^2+\veck^2}$, $\omega_\sigma^2 = \sqrt{M_\sigma^2+\veck^2}$。

再解出$\phi = \frac{\chi+\sigma}{\sqrt{2}}$, $\psi = \frac{\chi-\sigma}{\sqrt{2}}$ 即把$\phi$, $\psi$都表示成了独立的谐振子产生湮灭算符的线性组合。


}
\ech
\end{frame}


\begin{frame}
\chtitle{第4题 题目和思路}
\bch
{\small
题目:设$p$为电子的四维动量,$k$为光子的四维动量,$m$为电子质量,求下列矩阵的迹
\begin{itemize}
\item{$\slashed{k}$}
\item{$\slashed{p}\gamma^5$}
\item{$\slashed{p}\slashed{k}\frac{1}{\slashed{p}+\slashed{k}-m}$}
\item{$\slashed{k}(\slashed{p}+\slashed{k}+m)^3\frac{1}{\slashed{p}+\slashed{k}-m}(\slashed{p}-\slashed{k}+m)^2\slashed{k}$}
\item{$(\slashed{p}+m)\gamma^\mu\frac{1}{\slashed{p}+\slashed{k} -m}\slashed{k}\gamma_\mu\slashed{k}$}
\end{itemize}

\skipline

思路:利用$\slp^2 = p^2 = m^2$, $\slk^2 = k^2 = 0$以及课上学习的一些$\gamma$矩阵求迹技巧即可解答。
}

\ech
\end{frame}

\begin{frame}
\chtitle{第4题 解答}
\bch
{\tiny
利用$\slp^2 = p^2 = m^2$, $\slk^2 = k^2 = 0$,奇数个Feynman符号乘积的迹为零,$\trof{\slashed{p}\slashed{k}} = 4pk$,$\gamma^\mu\slashed{A}\gamma_\mu = -2\slashed{A}$, $\gamma^\mu\slashed{A}\slashed{B}\gamma_\mu = 4 AB$ 以及$\frac{1}{\slp+\slk-m} = \frac{\slp+\slk+m}{(p+k)^2-m^2} = \frac{\slp+\slk+m}{2pk}$,可以得到
\begin{itemize}
\item{$\trof{\slk} = 0$  }
\item{$\trof{\slp\gamma^5} = p_\mu\trof{\gamma^\mu\gamma^5} = 0$ (零迹定理,剔除$\gamma^5$) }
\item{$\trof{\slp\slk \frac{1}{\slashed{p}+\slashed{k}-m}} = \frac{1}{2pk}\trof{\slp\slk(\slp+\slk+m)} = \frac{m}{2pk}\trof{\slp\slk} = 2m$}
\item{$\trof{\slashed{k}(\slashed{p}+\slashed{k}+m)^3\frac{1}{\slashed{p}+\slashed{k}-m}(\slashed{p}-\slashed{k}+m)^2\slashed{k}} = \trof{\slashed{k}^2(\slashed{p}+\slashed{k}+m)^3\frac{1}{\slashed{p}+\slashed{k}-m}(\slashed{p}-\slashed{k}+m)^2} = 0$}
\item{$\trof{(\slashed{p}+m)\gamma^\mu\frac{1}{\slashed{p}+\slashed{k} -m}\slashed{k}\gamma_\mu\slashed{k}} = \frac{1}{2pk} \trof{(\slashed{p}+m)\gamma^\mu(\slashed{p}+\slashed{k} + m)\slashed{k}\gamma_\mu\slashed{k}} =  \frac{1}{2pk} \trof{(\slashed{p}+m) (4pk - 2m\slashed{k}) \slashed{k}} = 2 \trof{(\slashed{p}+m)\slk }  =2 \trof{\slp\slk } = 8 pk $ }
\end{itemize}
}
\ech
\end{frame}

\begin{frame}
\chtitle{第5题}
\bch
{\scriptsize
电子和$\mu$子可以分别看成独立的两个Dirac场$\psi_e$,$\psi_\mu$的粒子。和电磁场一起的拉氏密度为
$$\lagr = -\frac{1}{4}\Fup\Fdown + \bar{\psi}_e(\ii \slashed{D} - m_e)\psi_e + \bar{\psi}_\mu(\ii \slashed{D} - m_\mu)\psi_\mu$$
其中
$$ D_\mu = \partial_\mu + \ii q A_\mu$$

画出散射过程
$$e^+e^- \rightarrow \mu^+\mu^-$$
的非零的最低阶近似Feynman图。

\skipline

解答: 见课本143页图7-5
}


\ech
\end{frame}


\end{document}
