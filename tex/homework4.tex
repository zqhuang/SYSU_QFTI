\documentclass[CJK]{beamer}
\usepackage{CJKutf8}
\usepackage{beamerthemesplit}
\usetheme{Malmoe}
\useoutertheme[footline=authortitle]{miniframes}
\usepackage{amsmath}
\usepackage{amssymb}
\usepackage{graphicx}
\usepackage{color}
\graphicspath{{figures/}}
\def \bch {\begin{CJK}{UTF8}{gbsn}}
\def \ech {\end{CJK}}
\def \bex {\begin{minipage}{0.3\textwidth}\includegraphics[width=1in]{jugelizi.png}\end{minipage}\begin{minipage}{0.6\textwidth}}
\def \eex {\end{minipage}}
\def \chtitle#1 {\frametitle{\bch #1 \ech}}
\def \skipline { {\vskip 0.1in}}
\def \langr {\mathcal{L}}
\def \hamil {\mathcal{H}}
\def \vecx {\mathbf{x}}
\def \veck {\mathbf{k}}
\def \vecp {\mathbf{p}}
\def \hatphi {\hat{\phi}}
\def \hatq {\hat{q}}
\def \hatpi  {\hat{\pi}}
\def \vel {\upsilon}
\def \Dint {\mathcal{D}}
\def \adag {{\hat{a}^\dagger}}
\def \hata {\hat{a}}
\def \hatN {\hat{N}}
\def \hatH {\hat{H}}
\def \nket { {| n \rangle}}
\def \bran { {\langle n |}}

\title{Quantum Field Theory I \\ Homework 4}
  \author{}
  \date{}


\begin{document}

\begin{frame}
 
\begin{center}
\begin{Large}
\bch
量子场论 I 

{\vskip 0.3in}

第四次课后作业 (共八次,每次2.5分)

交作业时间: 11月7日,星期一,13:30pm

\ech
\end{Large}
\end{center}

\vskip 0.2in

\bch
课件下载
\ech
https://github.com/zqhuang/SYSU\_QFTI

\end{frame}

\begin{frame}
\chtitle{第1题(0.5分)}
\bch
对两个矢量$A^\mu$, $B^\mu$,证明在Dirac矩阵表示下:
$$\mathrm{Tr}(\slashed{A}\slashed{B}) = 4A^\mu B_\mu$$
其中$\mathrm{Tr}$表示矩阵求迹。
\ech
\end{frame}

\begin{frame}
\chtitle{第2题(0.5分)}
\bch
对旋量$\psi$证明$\bar{\psi}\gamma^\mu\gamma^\nu\gamma^\rho\psi$为洛仑兹变换下的三阶张量。
\ech
\end{frame}

\begin{frame}
\chtitle{第3题(0.5分)}
\bch
如果一个实标量场$\phi$和一个旋量场$\psi$有相互作用,拉氏密度为
$$\lagr = \frac{1}{2}\partial_\mu\phi\partial^\mu\phi - \frac{1}{2}m^2\phi^2 + g\phi^2\bar\psi\psi + \bar\psi(i\slashed{\partial}-m)\psi$$
其中$g$为耦合常数。

试推导$\phi$和$\psi$的运动方程。
\ech
\end{frame}


\begin{frame}
\chtitle{第4题(0.5分)}
\bch
设有三维动量$\veck = (k_x, k_y, k_z)$。请在以$z$方向自旋向上的态$|\uparrow\rangle$和自旋向下的态$|\downarrow\rangle$为基的表象里,写出沿$\veck$方向的电子自旋算符的矩阵表达式,并求它的所有本征值$s$和本征矢$\zeta_{\veck,s}$。
\ech
\end{frame}


\begin{frame}
\chtitle{第5题(0.5分)}
\bch
对$m=0$的旋量和非零的三维动量$\veck$,记相应的四维动量为$k$,证明$\slashed{k}$只有两个线性独立的本征态。 
\ech
\end{frame}



\end{document}
