\documentclass[CJK]{beamer}
\usepackage{CJKutf8}
\usepackage{beamerthemesplit}
\usetheme{Malmoe}
\useoutertheme[footline=authortitle]{miniframes}
\usepackage{amsmath}
\usepackage{amssymb}
\usepackage{graphicx}
\usepackage{color}
\graphicspath{{figures/}}
\def \bch {\begin{CJK}{UTF8}{gbsn}}
\def \ech {\end{CJK}}
\def \bex {\begin{minipage}{0.3\textwidth}\includegraphics[width=1in]{jugelizi.png}\end{minipage}\begin{minipage}{0.6\textwidth}}
\def \eex {\end{minipage}}
\def \chtitle#1 {\frametitle{\bch #1 \ech}}
\def \skipline { {\vskip 0.1in}}
\def \langr {\mathcal{L}}
\def \hamil {\mathcal{H}}
\def \vecx {\mathbf{x}}
\def \veck {\mathbf{k}}
\def \vecp {\mathbf{p}}
\def \hatphi {\hat{\phi}}
\def \hatq {\hat{q}}
\def \hatpi  {\hat{\pi}}
\def \vel {\upsilon}
\def \Dint {\mathcal{D}}
\def \adag {{\hat{a}^\dagger}}
\def \hata {\hat{a}}
\def \hatN {\hat{N}}
\def \hatH {\hat{H}}
\def \nket { {| n \rangle}}
\def \bran { {\langle n |}}

\title{Quantum Field Theory I \\ Openmind Homework}
  \author{}
  \date{}


\begin{document}

\begin{frame}
 
\begin{center}
\begin{Large}
\bch
量子场论 I 

{\vskip 0.3in}

脑洞大开作业 (10分,共5题每题2分,每题如有特别精彩巧妙的解法额外加1分bonus分, 但bonus分仅在期末总分不超过100分的情况下有效)

\skipline
允许搜索查阅讨论请教。允许使用任何文献(包括讲义)中的结论,但请指出出处。

\skipline
交作业时间: 期末考试前任何时间

\ech
\end{Large}
\end{center}

\vskip 0.2in

\bch
课件下载
\ech
https://github.com/zqhuang/SYSU\_QFTI

\end{frame}


\begin{frame}
\chtitle{第1题 二维空间的Casimir力(2分)}
\bch
假设“脑洞大开世界”是2维空间加1维时间的平直时空,时空度量元为
$$ds^2 = (dx^0)^2 - (dx^1)^2-(dx^2)^2$$
在这个世界里的“脑洞大开人”发现真空中的相距为$d$的很长的(长度$\gg d$)两根平行金属线之间有大小为$F$的相互作用力,他们认为这是两条金属线之间的真空能的改变引起的作用力,并把这种力称为Casimir力。现在问,当把金属线之间的距离变为$d/2$,Casimir力变为多大?

\ech
\end{frame}

\begin{frame}
\chtitle{第2题 只有两个自由度的“量子场”(2分)}
\bch
\begin{minipage}{0.3\textwidth}
\includegraphics[width=1.in]{shuangbai.png}
\end{minipage}
\begin{minipage}{0.5\textwidth}
如图,在一个长度为$\ell$,质量为$m$的理想刚性单摆下再悬挂一个相同的单摆。节点处都认为可以无阻力自由转动。本地重力常数为$g$。试求该系统的量子零点能。

\end{minipage}
\ech
\end{frame}

\begin{frame}
\chtitle{第3题 奇怪的一维量子场。(2分)}
\bch
\begin{minipage}{0.3\textwidth}
\includegraphics[width=0.5in]{duobai.png}
\end{minipage}
\begin{minipage}{0.5\textwidth}
把上题推广到$N$个长度为$\ell$,质量为$m$的理想刚性单摆首位连接挂起来(如左图给出了一个$N=5$的例子)。记系统的量子零点能为$E_N$。当$N\rightarrow \infty$时,所有自由度的平均零点能$E_N/N$趋向于一个常数,试计算这个常数(用$\ell$, $g$, $m$来表示)。

\skipline
{\scriptsize 
注:精确解的求解需要比较巧妙的手段。如果你找不到求解精确解的办法,请尽你所能地估算一个范围。
}
\end{minipage}

\ech
\end{frame}




\begin{frame}
\chtitle{第4题 环上的量子场(2分)}
\bch
假设“脑洞大开世界”为一个一维圆环加上一维时间,时空度量元为
$$ds^2 = dt^2 - R^2 d\theta^2$$
其中$R>0$为固定常数,$(t,\theta)$为时空坐标。在这个时空里的质量为$m$的实标量场$\phi(t,\theta)$满足周期性边界条件
$$\phi(t,\theta+2\pi) = \phi(t,\theta)$$
其自由场拉氏量为
$$L_{\rm free} = \int_0^{2\pi}d\theta\ \frac{1}{2}\left[\left(\frac{\partial\phi}{\partial t}\right)^2-\frac{1}{R^2}\left(\frac{\partial\phi}{\partial\theta}\right)^2-m^2\phi^2\right]$$
试把$\phi$场量子化。

\ech
\end{frame}


\begin{frame}
\chtitle{第5题 环上的量子场的散射(2分)}
\bch
给上题中的场$\phi$加一个自相互作用的拉氏量,
$$L = L_{\mathrm{free}} - \int_0^{2\pi}d\theta\ \frac{\lambda}{4}\phi^4$$
其中$\lambda\ll 1$为耦合常数。

试证明,这个场的两个粒子不可能发生散射变成和初态不同的两个粒子。

\ech
\end{frame}


\end{document}
