\documentclass[CJK]{beamer}
\usepackage{CJKutf8}
\usepackage{beamerthemesplit}
\usetheme{Malmoe}
\useoutertheme[footline=authortitle]{miniframes}
\usepackage{amsmath}
\usepackage{amssymb}
\usepackage{graphicx}
\usepackage{color}
\graphicspath{{figures/}}
\def \bch {\begin{CJK}{UTF8}{gbsn}}
\def \ech {\end{CJK}}
\def \bex {\begin{minipage}{0.3\textwidth}\includegraphics[width=1in]{jugelizi.png}\end{minipage}\begin{minipage}{0.6\textwidth}}
\def \eex {\end{minipage}}
\def \chtitle#1 {\frametitle{\bch #1 \ech}}
\def \skipline { {\vskip 0.1in}}
\def \langr {\mathcal{L}}
\def \hamil {\mathcal{H}}
\def \vecx {\mathbf{x}}
\def \veck {\mathbf{k}}
\def \vecp {\mathbf{p}}
\def \hatphi {\hat{\phi}}
\def \hatq {\hat{q}}
\def \hatpi  {\hat{\pi}}
\def \vel {\upsilon}
\def \Dint {\mathcal{D}}
\def \adag {{\hat{a}^\dagger}}
\def \hata {\hat{a}}
\def \hatN {\hat{N}}
\def \hatH {\hat{H}}
\def \nket { {| n \rangle}}
\def \bran { {\langle n |}}

\title{Quantum Field Theory I \\ Lesson 06 - Angular Momentum}
\author{}
\date{}


\begin{document}

\begin{frame}
 
\begin{center}
\begin{Large}
\bch
量子场论 I 

{\vskip 0.3in}

第五课 角动量理论

\ech
\end{Large}
\end{center}

\vskip 0.2in

\bch
课件下载
\ech
https://github.com/zqhuang/SYSU\_QFTI

\end{frame}



\begin{frame}
\chtitle{矩阵的函数简要回顾}
\bch
上节课末尾讲到,对任意的解析函数$f(x)$总能定义矩阵的函数$f(A)$。它有如下性质。
\begin{itemize}
\item{若$A$的本征值和本征矢为$(\lambda_1, \vecv_1)$; $(\lambda_2, \vecv_2)$;$\ldots$ 则$f(A)$有相同的本征矢,相应的本征值则变为$f(\lambda_1)$, $f(\lambda_2)$, $\ldots$}
\item{矩阵的函数满足复合函数规则,即:若解析函数$f$, $g$, $h$满足$h(x) = f(g(x))$,则对任意矩阵$A$有$h(A)=f(g(A))$。特别地,$(e^A)^n = e^{nA}$。}
\item{共轭转置和函数可以交换位置。即$\left(f(A)\right)^\dagger = f(A^\dagger)$。由此得出若$A$是厄米矩阵($A=A^\dagger$),则$f(A)$也是厄米矩阵。注意厄米矩阵的本征值都是实数。}
\item{若两个矩阵$A$, $B$对易,则它们的任意函数$f(A)$与$g(B)$也对易。特别地,$ e^{A}e^{B} = e^{B}e^{A}=e^{A+B}$。}
\end{itemize}

\ech
\end{frame}

\begin{frame}
\chtitle{三维空间转动算符}
\bch
任意三维矢量$\vecv$绕方向$\vecn$(我们约定方向矢总是归一化的,即$\vecn^2=1$)旋转无穷小角度$\epsilon$总是可以写成:
$$\vecv \rightarrow \vecv - \ii\epsilon \hatJn \vecv = e^{-\ii\epsilon\hatJn}\vecv\,.$$
其中$\hatJn$是$3\times 3$的转动矩阵,它只跟$\vecn$有关。为了防止同三维空间指标$i$混淆,以后我们用符号$\ii$来表示复数单位。

对于有限大小的转动角$\theta$,因为总是可以把转动划分成很多份的小转动,所以取足够大的$N$即有
$$\vecv\rightarrow (e^{-\ii(\theta/N)\hatJn})^N\vecv = e^{-\ii\theta \hatJn}\vecv$$
也就是说绕$\vecn$旋转$\theta$的操作仍然可以写成$e^{-\ii\theta \hatJn}$。算符$\hatJn$是对应于$\vecn$方向的转动算符,它不依赖于坐标系而存在。仅当我们需要写出$\hatJn$的具体矩阵表达式时,才需要选定一个坐标系。

\ech
\end{frame}

\begin{frame}
\chtitle{课堂互动}
\bch
对三个形成右手正交系的方向$\vecn_1$, $\vecn_2$, $\vecn_3$,对应的转动算符分别为$\hatj{1} $, $\hatj{2} $, $\hatj{3} $。试证明
\begin{itemize}
\item{对任意角度$\theta$,有$e^{-\ii\theta\hatj{3} }\vecn_1 =  \vecn_1 \cos\theta + \vecn_2 \sin\theta$, $e^{-\ii\theta\hatj{3} }\vecn_2 = \vecn_2 \cos\theta - \vecn_1 \sin\theta$, $e^{-\ii\theta\hatj{3}}\vecn_3 = \vecn_3$. }
\item{对任意角度$\theta$, 算符$e^{-\ii\theta\hatj{3} }$的三组本征矢量和本征值分别为$(\vecn_3, 1)$, $(\vecn_1+\ii\vecn_2, e^{-\ii\theta})$, $(\vecn_1-\ii\vecn_2, e^{\ii\theta})$。}
\item{根据上题结论,证明$\hatj{3}$的三组本征矢量和本征值分别$(\vecn_3, 0)$, $(\vecn_1+\ii\vecn_2, 1)$, $(\vecn_1-\ii\vecn_2, -1)$}.
\item{根据上题结论,证明$\hatj{3}\vecn_1=\ii\vecn_2$, $\hatj{3} \vecn_2 = -\ii\vecn_1$。通过置换$1,2,3$可以得到$\hatj{i}\vecn_j = \ii\epsilon_{ijk}\vecn_k$。其中$\epsilon_{ijk}$为完全反对称张量(当$ijk$为$123$的偶置换时为$\epsilon_{ijk}=1$,当$ijk$为$123$的奇置换时$\epsilon_{ijk}=-1$, $ijk$有重复指标时$\epsilon_{ijk} = 0$)。}
\item{$[\hatj{i}, \hatj{j} ] = \ii \epsilon_{ijk}\hatj{k} $}
\end{itemize}
\ech
\end{frame}



\begin{frame}
\chtitle{空间转动对称性和角动量守恒}
\bch
由于作用量是拉氏密度在全空间的积分,所以对固定的$\vecn$,整体空间的转动$e^{i\epsilon\hatJn}$保持作用量不变。对应的Noether定理中的守恒量就是角动量。

\skipline
首先,我们形式上把三维转动矩阵$\hatJn$拓写成四维张量$J^{\mu}_{\;\nu}$,规定$J^0_{\;\nu} = J^\mu_{\;0} = 0$,对$i$, $j$均不为零的情况$J^i_{\;j}$则是$\hatJn$矩阵的第$i$行第$j$列元素。(显然,这样定义的$J^\mu_{\;\nu}$并不是任意坐标变换下的张量,我们这里仅仅讨论三维空间的转动则没有问题。)由转动带来的坐标变动为
$$ \delta x^\mu =   \ii \epsilon J^\mu_{\;\nu} x^\nu$$

这样,对拉氏密度中的所有标量自由度$\phi$而言,$\phi$在转动后的$x^\mu$的位置的值,等于$\phi$在转动之前的坐标$x^\mu + \ii\epsilon J^\mu_{\;\nu} x^\nu$处的值。所以
$$\delta \phi = (-\ii \epsilon J^\mu_{\;\nu} x^\nu)\partial_\mu \phi$$

\ech
\end{frame}

\begin{frame}
\chtitle{空间转动对称性和角动量守恒}
\bch
于是我们得到
$$\frac{\delta \phi}{\delta\epsilon} = -\ii  J^\mu_{\;\nu} x^\nu \partial_\mu \phi$$.

\skipline
对矢量分量$A^\sigma$,则由于$A$本身也会发生旋转,所以多了额外一项。
$$\frac{\delta A^\sigma}{\delta\epsilon} = -\ii  J^\mu_{\;\nu} x^\nu\partial_\mu A^\sigma + \ii J^\sigma_{\;\nu}A^\nu$$.

\ech
\end{frame}

\begin{frame}
\chtitle{空间转动对称性和角动量守恒}
\bch
最后,拉氏密度作为一个标量,它的变化仅仅由坐标变动而引起:
$$ \delta \langr = (-\ii \epsilon J^\mu_{\;\nu} x^\nu)\partial_\mu \langr $$
另外,有
$$\partial_\mu(J^\mu_{\;\nu} x^\nu) = J^\mu_{\;\nu} \delta^nu_\mu = J^\mu_\mu = 0 $$
最后一个等号是因为$\hatJn$的迹(Trace)是所有本征值的和($=1+(-1)+0 = 0$).

所以我们得到
$$\delta \langr = \epsilon \partial_\mu (-\ii  J^\mu_{\;\nu} x^\nu \langr) $$

也就是Noether定理中的$F^\mu = -\ii  J^\mu_{\;\nu} x^\nu \langr$. 显然$F^0 = 0$.
\ech
\end{frame}

\begin{frame}
\chtitle{空间转动对称性和角动量守恒}
\bch
综上所述,我们得到Noether定理守恒荷密度:
$$j^0 =   \sum_q(\ii  J^\mu_{\;\nu} x^\nu \partial_\mu q) \frac{\partial \langr}{\partial(\partial_0 q)} - \ii J^\sigma_{\;\nu}A^\nu \frac{\partial \langr}{\partial(\partial_0 A^\sigma)}$$
其中 $q$取遍标量自由度和矢量所有分量自由度。因为$J^0_{\;\nu} = J^\mu_{\;0} = 0$,上式也可写成仅对空间坐标求和
$$j^0 =  -\ii  J^i_{\;j} x^j \sum_q(\partial_i q )\frac{\partial \langr}{\partial(\partial_0 q)} - \ii J^i_{\;j}A^j \frac{\partial \langr}{\partial(\partial_0 A^i)}$$
\ech
\end{frame}

\begin{frame}
\chtitle{空间转动对称性和角动量守恒}
\bch
最后,利用上节课的结论$\frac{\langr}{\partial(\partial_0 A^i)} = - F^0_{\; i}$,以及以前讨论过的空间平移带来的动量密度
$p_i = \sum_q(\partial_i q )\frac{\partial \langr}{\partial(\partial_0 q)} $,有
$$j^0 =  \ii  J^i_{\;j} (x^j p_i - F^0_{\;i} A^j)$$
注意到$F^0_{\;i}$即为“电场强度”$\mathbf{E}$, 第一项取共轭转置并注意到$\hatJn$是厄米矩阵,最终角动量密度可以写成 
$$j^0=-\ii(\vecx^\dagger \hatJn \vecp + \mathbf{E}^\dagger \hatJn \mathbf{A})$$
取$\vecn$沿$\vecn_3$轴方向,并根据$\hatj{3}\vecn_1 = i \vecn_2$, $\hatj{3}\vecn_2 = - i\vecn_1$,$\hatj{3}\vecn_3 = 0$, 上面的式子可以写成
$$j^0 = \vecx \times \vecp + \mathbf{E} \times \mathbf{A}$$
第一项为轨道角动量(包含了所有自由度),第二项是由于空间转动时矢量会随之变化而带来的“自旋”。

\ech
\end{frame}

\begin{frame}
\chtitle{}
\end{frame}

\end{document}
